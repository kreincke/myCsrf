% mycsrf License Include Module
%
% (c) Karsten Reincke, Frankfurt a.M. 2012, ff.
%
% This file is licensed under the Creative Commons Attribution 3.0 Germany
% License (http://creativecommons.org/licenses/by/3.0/de/): 
% For details see teh file LICENSE in the top directory
%

\footnote{\textbf{Dieser Text wird unter der XYZ Lizenz veröffentlicht.}
Hier können Bedingungen stehen, unter denen Sie Ihren Text weitergeben.
Gute Kandidaten wären z.B, die Creative Commons Lizenzen
\texttt{https://creativecommons.org/}. Traditionell ist auch die Formel:
\emph{Alle Rechte vorbehalten. Die Verwendung von Text und Bildern, auch
auszugsweise, bedürfen der schriftlichen Zustimmung.} Auf jeden Fall müssen Sie
(anschließend) -- gemäß der Lizenz \texttt{CC BY 3.0 DE}, unter der mycrsf
veröffentlicht ist -- angmessen auf die Ableitung von mycsrf verweisen.
\newline 
{ \tiny \itshape [Format abgeleitet vom \texttt{mind your Scholar Research
Framework} \copyright K. Reincke CC BY 3.0 DE http://mycsrf.fodina.de/)] }}

