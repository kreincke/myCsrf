% mycsrf cloak file
%
% (c) Karsten Reincke, Frankfurt a.M. 2010, 2011, ff.
%
% This file is licensed under the Creative Commons Attribution 3.0 Germany
% License (http://creativecommons.org/licenses/by/3.0/de/): 
% For details see teh file LICENSE in the top directory
%


\documentclass[
  DIV=calc,
  BCOR=5mm,
  11pt,
  headings=small,
  oneside,
  abstract=true,
  toc=bib,
  english,ngerman]{scrartcl}
  
%%% (1) general configurations %%%
\usepackage[utf8]{inputenc}

%%% (2) language specific configurations %%%
\usepackage[]{a4,babel}
\selectlanguage{ngerman}

% package for improving the grey value and the line feed handling
\usepackage{microtype}

%language specific quoting signs
%default for language emglish is american style of quotes
\usepackage[german=quotes]{csquotes}

% jurabib configuration
\usepackage[see]{jurabib}
\bibliographystyle{jurabib}
% mycsrf German jurabib configuration include module file 
%
% (c) Karsten Reincke, Frankfurt a.M. 2012, ff.
%
% This file is licensed under the Creative Commons Attribution 3.0 Germany
% License (http://creativecommons.org/licenses/by/3.0/de/): 
% For details see teh file LICENSE in the top directory

% the first time cite with all data, later with shorttitle
\jurabibsetup{citefull=first}

%%% (1) author / editor list configuration
%\jurabibsetup{authorformat=and} % uses 'und' instead of 'u.'
% therefore define your own abbreviated conjunction: 
% an 'and before last author explicetly written conjunction

% for authors in citations
\renewcommand*{\jbbtasep}{\ u.\ } % bta = between two authors sep
\renewcommand*{\jbbfsasep}{,\ } % bfsa = between first and second author sep
\renewcommand*{\jbbstasep}{\ u.\ }% bsta = between second and third author sep
% for editors in citations
\renewcommand*{\jbbtesep}{\ u.\ } % bta = between two authors sep
\renewcommand*{\jbbfsesep}{,\ } % bfsa = between first and second author sep
\renewcommand*{\jbbstesep}{\ u.\ }% bsta = between second and third author sep

% for authors in literature list
\renewcommand*{\bibbtasep}{\ u.\ } % bta = between two authors sep
\renewcommand*{\bibbfsasep}{,\ } % bfsa = between first and second author sep
\renewcommand*{\bibbstasep}{\ u.\ }% bsta = between second and third author sep
% for editors  in literature list
\renewcommand*{\bibbtesep}{\ u.\ } % bte = between two editors sep
\renewcommand*{\bibbfsesep}{,\ } % bfse = between first and second editor sep
\renewcommand*{\bibbstesep}{\ u.\ }% bste = between second and third editor sep

% use: name, forname, forname lastname u. forname lastname
\jurabibsetup{authorformat=firstnotreversed}
\jurabibsetup{authorformat=italic}

%%% (2) title configuration
% in every case print the title, let it be seperated from the 
% author by a colon and use the slanted font
\jurabibsetup{titleformat={all,colonsep}}
%\renewcommand*{\jbtitlefont}{\textit}

%%% (3) seperators in bib data
% separate bibliographical hints and page hints by a comma
\jurabibsetup{commabeforerest}

%%% (4) specific configuration of bibdata in quotes / footnote
% use a.a.O if possible
\jurabibsetup{ibidem=strict}
% replace ugly a.a.O. by ders., a.a.O. resp. ders., ebda.
% but if there are more than one author or girl writers?
\AddTo\bibsgerman{
  \renewcommand*{\ibidemname}{Ds.,\ a.a.O.}
  \renewcommand*{\ibidemmidname}{ds.,\ a.a.O.}
}
\renewcommand*{\samepageibidemname}{Ds.,\ ebda.}
\renewcommand*{\samepageibidemmidname}{ds.,\ ebda.}

%%% (5) specific configuration of bibdata in bibliography
% ever an in: before journal and collection/book-titles 

\renewcommand*{\bibjtsep}{in:\ }
\renewcommand*{\bibbtsep}{in:\ }

% ever a colon after author names 
\renewcommand*{\bibansep}{:\ }
% ever a semi colon after the title 
\renewcommand*{\bibatsep}{;\ }
% ever a comma before date/year
\renewcommand*{\bibbdsep}{,\ }

% let jurabib insert the S. and p. information
% no S. necessary in bib-files and in cites/footcites
\jurabibsetup{pages=format}

% use a compressed literature-list using a small line indent
\jurabibsetup{bibformat=compress}
\setlength{\jbbibhang}{1em}

% which follows the design of the cites and offers comments
\jurabibsetup{biblikecite}

% print annotations into bibliography
\jurabibsetup{annote}
\renewcommand*{\jbannoteformat}[1]{{ \itshape #1 }}

%refine the prefix of url download
\AddTo\bibsgerman{\renewcommand*{\urldatecomment}{Referenzdownload: }}

% we want to have the year of articles in brackets
\renewcommand*{\bibaldelim}{(}
\renewcommand*{\bibardelim}{)}

%Umformatierung des Reihentitels und der Reihennummer
\DeclareRobustCommand{\numberandseries}[2]{%
\unskip\unskip%,
\space\bibsnfont{(=~#2}%
\ifthenelse{\equal{#1}{}}{)}{, [Bd./Nr.]~#1)}%
}%

%Umformatierung Referenzverweises
\usepackage{xpatch}
\AfterFile{dejbbib.ldf}{%
  \xapptocmd{\bibsgerman}{%
     \def\inname{\ifjboxford in:\else\ifjbchicago in:\else in:\fi\fi}%
    \def\incollinname{\ifjboxford in:\else\ifjbchicago in:\else in:\fi\fi}%
  }{}{}%
}

% the field printed before ISBN, ISSN or URL is the bibfield note
% Hence: If you insert into the field note the type of the literature
% [ Print | [FreeWeb | BibWeb] / [ PDF | HTML ] ] then you now
% get entries like:
% Print: ISBN ....
% BibWeb / PDF => http...
% That's nice for dealing with electronic sources correctly
\DeclareRobustCommand{\jbissn}[1]{\unskip:\space ISSN #1}%
\DeclareRobustCommand{\jbisbn}[1]{\unskip:\space ISBN #1}%

\DeclareRobustCommand{\biburlprefix}{$\Rightarrow$ }
\DeclareRobustCommand{\biburlsuffix}{}



% language specific hyphenation
%mycsrfk Hyphenation Include Module text
%
% (c) Karsten Reincke, Frankfurt a.M. 2012, ff.
%
% This file is licensed under the Creative Commons Attribution 3.0 Germany
% License (http://creativecommons.org/licenses/by/3.0/de/): 
% For details see teh file LICENSE in the top directory
%


\hyphenation{ Mehr-stimmig-keit Musik-wissen-schaft-ler}



%%% (3) layout page configuration %%%

% select the visible parts of a page
% S.31: { plain|empty|headings|myheadings }
%\pagestyle{myheadings}
\pagestyle{headings}

% select the wished style of page-numbering
% S.32: { arabic,roman,Roman,alph,Alph }
\pagenumbering{arabic}
\setcounter{page}{1}

% select the wished distances using the general setlength order:
% S.34 { baselineskip| parskip | parindent }
% - general no indent for paragraphs
\setlength{\parindent}{0pt}
\setlength{\parskip}{1.2ex plus 0.2ex minus 0.2ex}


%%% (4) general package activation %%%
%\usepackage{utopia}
%\usepackage{courier}
%\usepackage{avant}
\usepackage[dvips]{epsfig}

% graphic
\usepackage{graphicx,color}
\usepackage{array}
\usepackage{shadow}
\usepackage{fancybox}

\usepackage{tikz}
\usetikzlibrary{arrows}
\usetikzlibrary{shapes,snakes}
\usetikzlibrary{positioning}
\usetikzlibrary{decorations.text}
\usetikzlibrary{trees}
\usetikzlibrary{matrix}

\usepackage{amsmath}
\usepackage{amsfonts}
\usepackage{amssymb}
\usepackage{wasysym}
\usepackage{chngcntr}


%- start(footnote-configuration)
\usepackage[marginal]{footmisc}
\deffootnote[1.5em]{1.5em}{1.5em}{\textsuperscript{\thefootnotemark)\ }}

% if document class: count footnotes from start to end

%- end(footnote-configuration)

% package for macking tables with broken lines
\usepackage{multirow}

%for using label as nameref
\usepackage{nameref}

%integrate nomenclature
% mycsrf  Deutsch Nomenclation Declaration Include Module 
%
% (c) Karsten Reincke, Frankfurt a.M. 2012, ff.
%
% This file is licensed under the Creative Commons Attribution 3.0 Germany
% License (http://creativecommons.org/licenses/by/3.0/de/): 
% For details see teh file LICENSE in the top directory

\usepackage[intoc]{nomencl}
\let\abbr\nomenclature
% Deutsche Überschrift
%\renewcommand{\nomname}{Abbreviations}
\renewcommand{\nomname}{Abkürzungen}

\setlength{\nomlabelwidth}{.20\hsize}
\renewcommand{\nomlabel}[1]{#1 \dotfill}
% reduce the line distance
\setlength{\nomitemsep}{-\parsep}
\makenomenclature


% depth of contents
\setcounter{secnumdepth}{5}
\setcounter{tocdepth}{5}

% Hyperlinks
\usepackage{hyperref}
\hypersetup{bookmarks=true,breaklinks=true,colorlinks=true,citecolor=blue,draft=false}

\begin{document}

%% use all entries of the bliography
\nocite{*}

%%-- start(titlepage)
\titlehead{Suche nach Sekundärliteratur}
\subject{Die Bibliothek der Universität Frankfurt am Main}
\title{Recherchebeispiel zum Thema 'Open Source License Management'}
\subtitle{Besonderheiten bei der Literatursuche u. -bestellung}

\author{K. Reincke\% mycsrf License Include Module
%
% (c) Karsten Reincke, Frankfurt a.M. 2012, ff.
%
% This file is licensed under the Creative Commons Attribution 3.0 Germany
% License (http://creativecommons.org/licenses/by/3.0/de/): 
% For details see teh file LICENSE in the top directory
%

\footnote{\textbf{This file is distributed under the terms of license XYZ}
Here, you can insert your conditions for using your text. Good examples
for such licenses are offered under \texttt{https://creativecommons.org/}. 
Traditionally it also possible to say : \emph{All rights reserved}.
In accordance to the license \texttt{CC BY 3.0 DE}, under which mycrsf
is released, you must finally point to mycsrf:
\newline 
{ \tiny \itshape [Format derived from \texttt{mind your Scholar Research
Framework} \copyright K. Reincke CC BY 3.0 DE http://fodina.de/mycsrf)] }}

}

%thanks entry cannot be combined with license footnote
%\thanks{den Autoren von KOMA-Script und denen von Jurabib}

\maketitle
%%-- end(titlepage)

\begin{abstract}
\noindent \itshape
Dieses Paper erläutert amnhand des Beispielthemas \emph{Open Source}, wie man in der 
\emph{Universitätsbibliothek Frankfurt a.M.}\footcite[s.][]{UbFaMHome}
entsprechende Sekundärliteratur findet, resp. deren Bibliotheksstandort
zwecks Ausleihe ermittelt.
\end{abstract}

%% no table of content in a snippet
\footnotesize
\tableofcontents

\normalsize

\section{Vorbemerkung}

Universitätsbibliotheken kaufen bei Verlagen Bücher und Zeitschriften, die sie
ihren 'Kunden' zur Einsicht bereitstellen. Manche Werke - insbesondere
elektronische Texte - dürfen Bibliotheken nur an Studenten oder
Universtitätsmitarbeiter aushändigen. Anderen dürfen diese Arbeiten nur
innerhalb des Bibliotheksgebäudes einsehen. Konsequenterweise müssen Bibliotheken
verifizieren, dass der Ausleihende ein Student, Universitätsmitarbeiter oder
externer Bibliotheksnutzer ist. Im Falle physischer Werke werden dazu bei der
Ausleihe die Bibliotheksausweise vorgelegt.

Bei dem Zugriff auf elektronische Werke über das Portal der Bibliothek von
außerhalb - also per Internet - gibt es statt des Ausweises nur den Login. Der
reicht als Legitimation (aus Sicht der Verlage) aber nicht aus:
zusätzlich muss sich der Ausleihende im Universitätsnetz befinden. Dies wird
über VPNs organisiert.

Externen Bibliotheksnutzern wird ein solches VPN aber nicht bereitgestellt.
Deshalb müssen sie sich gelegentlich physisch in das Universitätsgebäude begeben
und mittels die UB eigenen Rechner nach Literatur suchen und sie einsehen.

\section{Standort schon bekannter Werke ermitteln}

Ausgangspunkt sind hier die schon bekannten bibliographischen Daten eines
Werkes. Es gilt also 'nur', den entsprechenden Bibliotheksstandort zu ermitteln.

\subsection{Monographien und Lehrbücher}

Bei Büchern wird der Bibliotheksstandort über die Suche nach Autor, Titel etc.
ermittelt. Anhand der Angaben können die physischen Texte in die Ausleihe oder
den Lesesaal bestellt werden. Und die  Volltextversionen elektronischer Texte
können darüber eingesehen und/oder downgeloadet werden, sofern der Zugriff
zulässig ist.

In der UB gibt es 3 Einstiegspunkte für die Suche im Frankfurter
OPAC\footnote{Online Public Access Catalogue}:

\begin{itemize}
  \item Gehe zu {\ttfamily http://www.ub.uni-frankfurt.de/} und wähle dort
  im linken Menu \emph{Kataloge} an.
  \item Oder gehe direkt zu {\ttfamily http://suche.ub.uni-frankfurt.de/}.
  \item Oder gehe direkt zum älteren Frontend des
  Frankfurter OPAC Katalog {\ttfamily https://lbsopac.rz.uni-frankfurt.de/} (er
  zeigt in der Trefferliste auch an, ob es sich um eine elektronisches oder ein
  gedrucktes Werk handelt.)
\end{itemize}
  
Um diese zu nutzen, verfährt man wie folgt:

\begin{itemize}
  \item Logge Dich ein mit den Daten Deines Bibliotheksausweises
  \item Wähle - je nach Fokus - rechts oben den UB-Katalog oder rechts mittig
  den ebook-katalog an.
  \item Gib den Autornamen, Kernwörter des Titels oder Herausgebernamen ins
  lineare Suchfeld ein.
  \item Finde das gewünschte Buch in der Trefferliste
\end{itemize}

Im Erfolgsfall kann das gesuchte Werk in der Trefferliste angeklickt werden. Um
es real einsehen zu können, sind vier Fälle zu unterscheiden:
\begin{itemize}
  \item Liegt es gebunden vor und ist nicht vorbestellt, kann es im
  erscheinenden OPAC Formular direkt in die Buchausgabe der UB (ggfls. in den
  Lesesaal) bestellt werden und liegt dann dort (1 Woche) zur
  Einsicht/Ausleihe bereit. [\emph{verifiziert am} {\ttfamily 2011.07.01}]
  \item Liegt es gebunden vor und ist vorbestellt, kann es im erscheinenden OPAC
  Formular direkt vorbestellt werden. Nach der Benachrichtigung kann dan wie
  o.a. verfahren werden.  [\emph{verifiziert am} {\ttfamily 2011.07.08}]
  \item Liegt es als elektronische Resource vor, kann es im UB Netz direkt als
  Volltext eingesehen resp. als PDF auf einen Stick heruntergeladen werden. 
  [\emph{verifiziert am} {\ttfamily 2011.07.08}]
  \item Liegt es gebunden vor, ist aber nicht ausleihbar, so muss es direkt
  unter dem angegebenen Standort im Lesesaal / Freihandbestand eingesehen werden. 
  [\emph{verifiziert am} {\ttfamily 2011.07.01}]
\end{itemize}

Wenn das Werk (von außerhalb des UNI/UB-Netzes) nicht über den Katalog der
UB FaM zu finden ist, gibt es folgende Alternativen:

\begin{itemize}
  \item Gehe ins Gebäude der UB FaM an einen Rechner im UB-Netz, und wiederhole
  obiges Verfahren sicherheitshalber innerhalb des UB/UNI-Netzes. 
  \item Wähle
  insbesondere unter {\ttfamily http://suche.ub.uni-frankfurt.de/} rechts mittig
  die Datenbank ebooks (Books \ldots online) an. Oder nutze den allgemeinen
  Einstieg {\ttfamily http://www.ub.uni-frankfurt.de/}, wähle li. oben den
  Menueintrag \enquote{Datenbanken, E-Journals, E-Books} und recherchiere
  in den elektronischen Büchern.  [\emph{verifiziert am} {\ttfamily 2011.07.08}]
  \item Wähle unter {\ttfamily http://suche.ub.uni-frankfurt.de/} - sofern das
  gesuchte Buch weder in den gedruckten, noch in den elektronischen Werken zu
  finden ist - rechts oben satt des UB Kataloges den HeBiS-Katalog an oder gehe
  direkt zu {\ttfamily https://www.portal.hebis.de}, logge Dich dort mit Deinen
  UB FaM Daten ein und initiiere nach Erfolg ein Fernleihe.  [\emph{verifiziert
  am} {\ttfamily 2011.07.08}]
\end{itemize}


\subsection{Zeitschriftenartikel}

Sind die bibliographischen Angaben eines Zeitschriftenartikels bekannt,
\enquote{reduziert} sich die Recherche auf die Ermittlung des Standorts der
Zeitschrift selbst und das Holen (Kopieren) des Artikels. Herausforderung ist
hier, dass wesentlich mehr Zeitschriften nur elektronisch existieren.

\subsubsection{Zeitschriften}

Gedruckte / gebundene Zeitschriften als solche werden wie Bücher behandelt. Für
die Suche nach elektronische Zeitschriften gibt es eine
universitätsübergreifende Bibliothek, die \emph{EZB} [\enquote{Elektronische
Zeitschriftenbibliothek}]. Sie wird von allen Unis verlinkt, allerdings
jeweils unterschiedlich parametrisiert, sodass nur die von der UB gekauften
Zeitschriften auch über das je spezifische Netz als Volltext eingesehen werden
können\footnote{Grüne Ampel: Volltext frei im Internet, gelbe Ampel: UB
spezifisch, Volltext nur über UB)}.

Um die Zeitschrift zu finden, führt man Folgendes aus:


\begin{itemize}
  \item Verfahre wie bei einem gebundenen gedruckten Buch und suche über den UB
  FaM Katalog nach dem Zeitschriftentitel (nicht dem Titel/Autor des Artikels
  in der Zeitschrift)
  \item Gehe zu {\ttfamily http://www.ub.uni-frankfurt.de/} und dort per
  Menüeintrag \enquote{Datenbanken, E-Journals, E-Books} und den dabei
  erscheinenden Link \enquote{Elektronische Zeitschriftenbibliothek}. Hinter
  diesem Eintrag verbirgt sich der Link auf die EZB, der für UB FaM
  parametrisiert worden ist: {\ttfamily
  http://ezb.uni-regensburg.de/fl.phtml?bibid=UBFM}.
  \item Wähle in der EZB Dein Fachgebiet und in dessen Liste Deine
  Zeitschrift.
\end{itemize}

Wenn die Ampel der gefundenen elektronische Zeitschrift grün ist, ist die
Zeitschrift allgemein im Netz erhältlich. Wenn die Ampel gelb ist, kann sie nur
im Netz der UB eingesehen resp. als PDF im Volltext runtergeladen werden. Und
wenn sie rot ist, kann sie nur über Fernleihe bestellt werden.
  
Insgesamt gibt es im Erfolgsfall drei Möglichkeiten:
\begin{itemize}
  \item Liegt die Zeitschrift nur gebunden vor, muss der entsprechende Band -
  wie ein Buch - zur Ansicht in den Lesesaal bestellt werden.
  \item Liegt die Zeitschrift (auch) elektronisch vor und ist ihre Ampel im
  EZB grün, kann der entsprechende Band übers Internet frei eingesehen und der
  Artikel (ausserhalb des UB-Netzes) als PDF downgeloaded werden.
  \item Liegt die Zeitschrift (auch) elektronisch vor und ist ihre Ampel im
  EZB gelb, kann der entsprechende Band nur über das UB-Netz eingesehen und
  der Artikel als PDF downgeloaded werden.
\end{itemize}  
 

\subsubsection{Zeitschriftenartikel}
\begin{itemize}
  \item Bei einer gebunden Zeitschrift muss die entsprechende Ausgabe der
  Zeitschrift in den den Lesensaal bestellt werden, um den Artikel kopieren zu
  können.
  \item Bei elektronischen Artikeln kann er bei grüner oder gelber Ampel
  direkt eingesehen und als PDF downgeloaded werden.
\end{itemize}

\subsection{Sammlungen und Periodica}

Sammlungen und Periodica werden analog zu Büchern oder Zeitschriften behandelt.
Auch hier ist der Unterschied zwischen gebunden / gedruckten Werken und
elektronischen entsprechend zu beachten.

\section{Themenzentrierte Suche: relevante Literatur finden }

\subsection{Schlagwortsuche im Title, Abstract und den Keywords}
Der eine Weg zur thematische Suche führt über die Schlagwortsuche im Titel, im
Abstract und in den Keywords. Technisch unterscheidet sich diese nicht wirklich
von der Suche nach bekannten bibliographischen Daten im UB-Bestand. In beiden
Fällen wird auf den OPAC-Katalog zugegriffen. In das Suchfeld werden eben nur
thematische Suchbegriffe eingegeben, wobei ggf. die Detailssuche bemüht wird.
Hier noch einmal die Links auf die Katalogsuche:


\begin{itemize}
  \item books \& ebooks (UB-Katalog): {\ttfamily
  http://suche.ub.uni-frankfurt.de/}
  \item books \& ebooks (UB-Katalog): {\small \ttfamily
  https://lbsopac.rz.uni-frankfurt.de/}
  \item books \& ebooks: {\ttfamily
  https://lbsopac.rz.uni-frankfurt.de/}
  \item ebooks: {\ttfamily \tiny
  http://www.ub.uni-frankfurt.de/datenbanken/ebooks\_gesamt.html}
\end{itemize}

Gesucht wird in all diesen Fällen auf Metadaten. Das bedeutet konsequenterweise,
dass so nicht nach Zeitschriftenartikeln gesucht werden kann. Das wäre erst
möglich, wenn die UBs alle Artikel ihrer Zeitschriften einzelnd
verschlagwortet / erfasst hätten. Diesen Service liefern stattdessen
Literaturdatenbanken.

\subsection{Stichwortsuche in Literaturdatenbanken}

Literaturdatenbanken sind zumeist fachspezifisch, verschlagworten dafür
aber (auch) die einzelnen Artikel der Zeitschriften. Es gibt Datenbanklisten,
aus denen eine Datenbank auszuwählen ist, um darin zu suchen. Auch die
Datenbanken können frei im Netz zugänglich oder bibliotheksspezifisch sein. In
die Literaturdatenbanklisten kann über verschiedene Links eingestiegen werden\footnote{
Hier einige wichtige Literaturdatenbanken zum Fach Informatik:
\begin{itemize}
  \item ACM Digital Library / Association for Computing Machinery
  \item CiteSeer.IST Scientific Literature Digital Library
  \item Web of Knowledge \ldots\footnote{ \tiny vgl. {\ttfamily
  http://info.ub.uni-frankfurt.de/fach\_liste.html?fach=informatik }}
  \item arXiv.org e-Print archive (frei)
  \item CiteSeerX Beta 	(frei)
  \item Collection of Computer Science Bibliographies (frei)
  \item Computer Science Bibliography (frei)
  \item WTI-Frankfurt Datenbanken (ehem. FIZ Technik) (UB)
  \item FOLDOC : Free On-Line Dictionary of Computing (frei)
  \item IEEE Xplore / Electronic Library Online (IEL) (UB)
  \item INFODATA (frei)
  \item Journal Citation Reports (UB)
  \item MathSciNet (UB) \ldots
  \item Web of Science (UB)\ldots
\end{itemize}
Für das Web of Science siehe {\ttfamily 
http://rzblx10.uni-regensburg.de/dbinfo/dbliste.php?
bib\_id=tuda\&colors=63\&ocolors=40\&lett=f\&gebiete=30
}

}:

\begin{itemize}
  \item {\ttfamily http://www.ub.uni-frankfurt.de/} $\Rightarrow$ Menueintrag
  \emph{Datenbanken, \ldots}
  \item {\ttfamily http://www.ub.uni-frankfurt.de/banken.html} $\Rightarrow$ 
  \emph{Datenbanken}
  \item Oder direkt {\ttfamily http://info.ub.uni-frankfurt.de/}
\end{itemize}


\section{Fallen}

Es gibt Datenbanken, die über das Internet (CiteSeer, Web of Science, ACM
Digital Library) oder per Bib-Login (Hebis) {\bfseries in ihrer
Suchfunktionalität} frei erreichbar sind. Allerdings ist der \textbf{Download
des Volltextes} bei einigen oder vielen Werken dann doch nur \textbf{über das
Universitätsnetz} möglich. Deshalb lohnt sich der Einstieg bei einigen von ihnen
( ACM Digital Library, Web of Science, Hebis) direkt über das Universitätsnetz
(Vor-Ort-Recherche).

Erschwert wird der Zugang dadurch, dass für Gast-Rechercheure \textbf{bei der
Vor-Ort-Recherche} über das UB-Netz der \textbf{Durchgriff ins Internet
gesperrt} ist. Deshalb lohnt es sich bei mehrheitlich frei orientierten
Datanbanken (CiteSeer) eher, zuerst über das freie Internet zu recherchen und
nicht zugängliche Downloads gezielt über das Universitätsnetz noch einmal zu
suchen und dann downzuloaden.

\section{Beispielhafter Recherche Report zum Thema Open Source}

Die erste Tabelle listet die Kataloge der Unibibliothek Frankfurt auf, für deren
Nutzung diese zahlt und die deshalb für Nicht-Universitätsangehörige zumeist nur
im Netz der Uni-Bibliothek ausgewertet (und die Ergebnisse nur ebenso
eingesehen) werden können.

Die zweite Tabelle listet die über das Internet zugänglichen
Literaturdatenbanken auf, die ebenso über Portal der Universität angesteuert und
außerhalb des Uni-Netzes ausgewertet werden können. Die Ergebnisse können
wiederum elektronische Bücher/Artikel sein, die dann wieder nur im Uni-Netz
konkret eingesehen werden können\footnote{\underline{Zur Erinnerung:} Die
Elektronische Zeitschriften Bibliothek (und der Hebis-Katalog?) bieten nur eine
fachbereichsbezogene Suche nach Zeitschriftentiteln, nicht nach Artikeln in den
Zeitschriften. Dafür sind die einzelnen Datenbanken zuständig}!

\begin{table}
\scriptsize
\caption{Resourcen aus/über Frankfurter Universiätsbibliothek}
\begin{center}
\begin{tabular}[h]{|r|c|c|c||c||c|c|c|c||c|c|c|c|c|c|c|c||c|}
\hline
& \rotatebox{90}{$\clubsuit$ OPAC FaM}
& \rotatebox{90}{$\clubsuit$ \textit{ACM Digital Library}}
& \rotatebox{90}{$\clubsuit$ \textit{Web of Science}}
& \rotatebox{90}{$\spadesuit$ Hebis Portal}
& \rotatebox{90}{$\heartsuit$ \textit{CiteSeer.IST FaM}}
& \rotatebox{90}{$\heartsuit$ \textit{Web of Knowledge}}
& \rotatebox{90}{$\heartsuit$ EZB FaM Informatik}
& \rotatebox{90}{$\heartsuit$ EZB FaM Jura}
& \rotatebox{90}{$\diamondsuit$ \textit{IBZ (Int. Bibl. geistes-\&soz.-wis. Zeitschr.)~}} 
& \rotatebox{90}{$\diamondsuit$ \textit{Int. Philsophiocal Bibl.}} 
& \rotatebox{90}{$\diamondsuit$ \textit{Cambridge Journals Digital Archive}}
& \rotatebox{90}{$\diamondsuit$ \textit{Index to theses (GB/IR)}}
& \rotatebox{90}{$\diamondsuit$ \textit{Springer E-books (Comp. Science + Techn. \& Inf.)~}} 
& \rotatebox{90}{$\diamondsuit$ \textit{Juris Spectrum Datenbank}}
& \rotatebox{90}{$\diamondsuit$ \textit{Oldenbourg \& Akademie e-books (Inf., Phil)}} 
& \rotatebox{90}{$\diamondsuit$ \textit{Oxford Journals}}
& \rotatebox{90}{\itshape{???}}
\\
\hline \hline
Open Source Li[cen[c/s]e]zenz]
  & $\ast$ & $\ast$ & $\ast$ & $\ast$ & ? & $\circ$ 
  & $\circ$ & $\neg$ & ? & ? & ? & ?
  & ? & ? & ? & ? & ?\\
\hline
GNU Public Licen[c/s]e
  & $\star$ & $\star$ & $\circ$ & $\star$ & ? & $\odot$
  & $\neg$ & $\neg$ & ? & ? & ? & ?
  & ? & ? & ? & ? & ?\\
\hline
GNU Lizenz
  & $\circ$ & $\neg$ & $\neg$ & $\circ$ & ? & $\odot$
  & $\neg$ & $\neg$ & ? & ? & ? & ?
  & ? & ? & ? & ? & ?\\
\hline
Free Software Licen[c/s]e]
  & $\circ$ & $\circ$ & $\circ$ & $\circ$ & ? & $\neg$ 
  & $\neg$ & $\neg$ & ? & ? & ? & ?
  & ? & ? & ? & ? & ?\\
\hline
Freie Software Lizenz
  & $\circ$ & $\neg$ & $\neg$ & $\circ$ & ? & $\neg$ 
  & $\neg$ & $\neg$ & ? & ? & ? & ?
  & ? & ? & ? & ? & ?\\
\hline
OSI / Open Source Initiative
  & $\star$ & $\neg$ & $\circ$ & $\neg$ & ? & $\circ$ 
  & $\neg$ & $\neg$ & ? & ? & ? & ?
  & ? & ? & ? & ? & ?\\
\hline
FSF / Free Software Foundation
  & $\star$ & $\star$ & $\neg$ & $\star$ & ? & $\circ$ 
  & $\neg$ & $\neg$ & ? & ? & ? & ?
  & ? & ? & ? & ? & ?\\
\hline
Apache Li[cen[c/s]e]zenz]
  & $\ast$ & $\ast$ & $\neg$ &  $\circ$ & ? & $\odot$
  & $\neg$ & $\neg$ & ? & ? & ? & ?
  & ? & ? & ? & ? & ?\\
\hline
Apache Foundation
  & $\circ$ & $\neg$ & $\neg$ & $\neg$ & ? & $\odot$
  & $\neg$ & $\neg$ & ? & ? & ? & ?
  & ? & ? & ? & ? & ?\\
\hline
Eclipse Public Li[cen[c/s]e]zenz]
  & $\ast$ & $\circ$ & $\circ$ & $\circ$ & ? & $\odot$
  & $\neg$ & $\neg$ & ? & ? & ? & ?
  & ? & ? & ? & ? & ?\\
\hline 
FL/OSS FOSS
  & $\neg$ & $\star$ & $\circ$ & $\neg$ & ? & $\odot$
  & $\neg$ & $\neg$ & ? & ? & ? & ?
  & ? & ? & ? & ? & ?\\
\hline
Copyleft
  & $\circ$ & $\circ$ & $\circ$ & $\star$ & ? & $\odot$
  & $\neg$ & $\neg$ & ? & ? & ? & ?
  & ? & ? & ? & ? & ?\\
\hline
\hline$\odot$
GPL
  & $\circ$ & $\circ$ & $\circ$ & $\neg$ & ? & $\odot$
  & $\neg$ & $\neg$ & ? & ? & ? & ?
  & ? & ? & ? & ? & ?\\
\hline
EPL
  & $\neg$ & $\circ$ & $\neg$ & $\neg$ & ? & $\odot$
  & $\neg$ & $\neg$ & ? & ? & ? & ?
  & ? & ? & ? & ? & ?\\
\hline 
\hline
Open Source
  & $\star$ & $\ast$ & $\odot$ & $\odot$ & ? & $\circ$ 
  & $\circ$ & $\neg$ & ? & ? & ? & ?
  & ? & ? & ? & ? & ?\\
\hline
Free Software
  & $\circ$ & $\circ$ & $\odot$ & $\odot$ & ? & $\circ$
  & $\neg$ & $\neg$ & ? & ? & ? & ?
  & ? & ? & ? & ? & ?\\
\hline
Freie Software
  & $\circ$ & $\neg$ & $\neg$ & $\odot$ & ? & $\neg$
  & $\neg$ & $\neg$ & ? & ? & ? & ?
  & ? & ? & ? & ? & ?\\
\hline
GNU
  & $\circ$ & $\circ$ & $\odot$ & $\odot$ & ? & $\odot$
  & $\neg$ & $\neg$ & ? & ? & ? & ?
  & ? & ? & ? & ? & ?\\
\hline
Apache
  & $\circ$ & $\neg$ & $\neg$ & $\odot$ & ? & $\odot$
  & $\neg$ & $\neg$ & ? & ? & ? & ?
  & ? & ? & ? & ? & ?\\
\hline
Eclipse
  & $\neg$ & $\circ$ & $\odot$ & $\odot$ & ? & $\odot$
  & $\circ$ & $\neg$ & ? & ? & ? & ?
  & ? & ? & ? & ? & ?\\
\hline
\hline
OSI
  & $\ast$ & $\neg$ & - & - & ? & -
  & $\neg$ & $\neg$ & ? & ? & ? & ?
  & ? & ? & ? & ? & ?\\
\hline
FLOSS
  & - & - & - & - & ? & -
  & - & - & ? & ? & ? & ?
  & ? & ? & ? & ? & ?\\
\hline
OSS
  & $\ast$ & $\ast$ & - & - & ? & -
  & $\neg$ & $\neg$ & ? & ? & ? & ?
  & ? & ? & ? & ? & ?\\
\hline

\end{tabular}
\end{center}
\end{table}


\begin{table}
\small
\caption{Freie Internetresourcen (ggfls. redo UB-Portale DA oder FaM)}
\begin{center}
\begin{tabular}[h]{|r||c|c|c||c|c||c|c|c|c|c|c||c|c|c|}
\hline
& \rotatebox{90}{$\clubsuit$ \textit{CiteSeer}}
& \rotatebox{90}{$\clubsuit$ Amazon Book Store}
& \rotatebox{90}{$\clubsuit$ O'Reilly Book Store}
& \rotatebox{90}{$\spadesuit$ \textit{bibsonomy}}
& \rotatebox{90}{$\spadesuit$ \textit{Google Scholar}}
& \rotatebox{90}{$\heartsuit$ \textit{AarXiv.org [DA]}}
& \rotatebox{90}{$\heartsuit$ \textit{Coll. of Comp. Science Bibl. [DA]}}
& \rotatebox{90}{$\heartsuit$ \textit{Comp. Science Bibliography [DA]}}
& \rotatebox{90}{$\heartsuit$ \textit{(F)ree (O)n-(L)ine (D)ictionary (o)f (C)omputing [DA]}}
& \rotatebox{90}{$\heartsuit$ \textit{INFODATA [DA]}}
& \rotatebox{90}{$\heartsuit$ \textit{spires}}
& \rotatebox{90}{$\diamondsuit$ Hebis}
& \rotatebox{90}{$\diamondsuit$ \textit{Web of Science}}
& \rotatebox{90}{$\diamondsuit$ \textit{Web of Knowledge}}
\\
\hline \hline
Open Source Li[cen[c/s]e]zenz]
  & ? & ? & ? & ? & ? 
  & ? & ? & ? & ? & ? 
  & ? & ? & ? & ?\\
\hline
GNU Public Licen[c/s]e
  & ? & ? & ? & ? & ? 
  & ? & ? & ? & ? & ? 
  & ? & ? & ? & ?\\
\hline
GNU Lizenz
  & ? & ? & ? & ? & ? 
  & ? & ? & ? & ? & ? 
  & ? & ? & ? & ?\\
\hline
Free Software Licen[c/s]e]
  & ? & ? & ? & ? & ? 
  & ? & ? & ? & ? & ? 
  & ? & ? & ? & ?\\
\hline
Freie Software Lizenz
  & ? & ? & ? & ? & ? 
  & ? & ? & ? & ? & ? 
  & ? & ? & ? & ?\\
\hline
OSI / Open Source Initiative
  & ? & ? & ? & ? & ? 
  & ? & ? & ? & ? & ? 
  & ? & ? & ? & ?\\
\hline
FSF / Free Software Foundation
  & ? & ? & ? & ? & ? 
  & ? & ? & ? & ? & ? 
  & ? & ? & ? & ?\\
hline
Apache Li[cen[c/s]e]zenz]
  & ? & ? & ? & ? & ? 
  & ? & ? & ? & ? & ? 
  & ? & ? & ? & ?\\
\hline
Apache Foundation
  & ? & ? & ? & ? & ? 
  & ? & ? & ? & ? & ? 
  & ? & ? & ? & ?\\
\hline
Eclipse Public Li[cen[c/s]e]zenz]
  & ? & ? & ? & ? & ? 
  & ? & ? & ? & ? & ? 
  & ? & ? & ? & ?\\
\hline 
FL/OSS FOSS
  & ? & ? & ? & ? & ? 
  & ? & ? & ? & ? & ? 
  & ? & ? & ? & ?\\
\hline
Copyleft
  & ? & ? & ? & ? & ? 
  & ? & ? & ? & ? & ? 
  & ? & ? & ? & ?\\
\hline
\hline
GPL
  & ? & ? & ? & ? & ? 
  & ? & ? & ? & ? & ? 
  & ? & ? & ? & ?\\
\hline
EPL
  & ? & ? & ? & ? & ? 
  & ? & ? & ? & ? & ? 
  & ? & ? & ? & ?\\
\hline 
\hline
Open Source
  & ? & ? & ? & ? & ? 
  & ? & ? & ? & ? & ? 
  & ? & ? & ? & ?\\
\hline
Free Software
  & ? & ? & ? & ? & ? 
  & ? & ? & ? & ? & ? 
  & ? & ? & ? & ?\\
\hline
Freie Software
  & ? & ? & ? & ? & ? 
  & ? & ? & ? & ? & ? 
  & ? & ? & ? & ?\\
\hline
GNU
  & ? & ? & ? & ? & ? 
  & ? & ? & ? & ? & ? 
  & ? & ? & ? & ?\\
\hline
Apache
  & ? & ? & ? & ? & ? 
  & ? & ? & ? & ? & ? 
  & ? & ? & ? & ?\\
\hline
Eclipse
  & ? & ? & ? & ? & ? 
  & ? & ? & ? & ? & ? 
  & ? & ? & ? & ?\\
\hline
\hline
OSI
  & ? & ? & ? & ? & ? 
  & ? & ? & ? & ? & ? 
  & ? & ? & ? & ?\\
\hline
FLOSS
  & ? & ? & ? & ? & ? 
  & ? & ? & ? & ? & ? 
  & ? & ? & ? & ?\\
\hline
OSS
  & ? & ? & ? & ? & ? 
  & ? & ? & ? & ? & ? 
  & ? & ? & ? & ?\\
\hline
\end{tabular}
\end{center}
\end{table}
\newpage
\small
\bibliography{bib/literature}

\end{document}
