% mycsrf cloak file
%
% (c) Karsten Reincke, Frankfurt a.M. 2010, 2011, ff.
%
% This file is licensed under the Creative Commons Attribution 3.0 Germany
% License (http://creativecommons.org/licenses/by/3.0/de/): 
% For details see teh file LICENSE in the top directory
%
% select the document class
% S.26: [ 10pt|11pt|12pt onecolumn|twocolumn oneside|twoside notitlepage|titlepage final|draft
%         leqno fleqn openbib a4paper|a5paper|b5paper|letterpaper|legalpaper|executivepaper openrigth ]
% S.25: { article|report|book|letter ... }
%
% oder koma-skript S.10 + 16
\documentclass[
  DIV=calc,
  BCOR=5mm,
  11pt,
  headings=small,
  oneside,
  abstract=true,
  toc=bib,
  english,ngerman]{scrartcl}
  
%%% (1) general configurations %%%
\usepackage[utf8]{inputenc}

%%% (2) language specific configurations %%%
\usepackage[]{a4,babel}
\selectlanguage{ngerman}

% package for improving the grey value and the line feed handling
\usepackage{microtype}

%language specific quoting signs
\usepackage{csquotes}

% jurabib configuration
\usepackage[see]{jurabib}
\bibliographystyle{jurabib}
% mycsrf German jurabib configuration include module file 
%
% (c) Karsten Reincke, Frankfurt a.M. 2012, ff.
%
% This file is licensed under the Creative Commons Attribution 3.0 Germany
% License (http://creativecommons.org/licenses/by/3.0/de/): 
% For details see teh file LICENSE in the top directory

% the first time cite with all data, later with shorttitle
\jurabibsetup{citefull=first}

%%% (1) author / editor list configuration
%\jurabibsetup{authorformat=and} % uses 'und' instead of 'u.'
% therefore define your own abbreviated conjunction: 
% an 'and before last author explicetly written conjunction

% for authors in citations
\renewcommand*{\jbbtasep}{\ u.\ } % bta = between two authors sep
\renewcommand*{\jbbfsasep}{,\ } % bfsa = between first and second author sep
\renewcommand*{\jbbstasep}{\ u.\ }% bsta = between second and third author sep
% for editors in citations
\renewcommand*{\jbbtesep}{\ u.\ } % bta = between two authors sep
\renewcommand*{\jbbfsesep}{,\ } % bfsa = between first and second author sep
\renewcommand*{\jbbstesep}{\ u.\ }% bsta = between second and third author sep

% for authors in literature list
\renewcommand*{\bibbtasep}{\ u.\ } % bta = between two authors sep
\renewcommand*{\bibbfsasep}{,\ } % bfsa = between first and second author sep
\renewcommand*{\bibbstasep}{\ u.\ }% bsta = between second and third author sep
% for editors  in literature list
\renewcommand*{\bibbtesep}{\ u.\ } % bte = between two editors sep
\renewcommand*{\bibbfsesep}{,\ } % bfse = between first and second editor sep
\renewcommand*{\bibbstesep}{\ u.\ }% bste = between second and third editor sep

% use: name, forname, forname lastname u. forname lastname
\jurabibsetup{authorformat=firstnotreversed}
\jurabibsetup{authorformat=italic}

%%% (2) title configuration
% in every case print the title, let it be seperated from the 
% author by a colon and use the slanted font
\jurabibsetup{titleformat={all,colonsep}}
%\renewcommand*{\jbtitlefont}{\textit}

%%% (3) seperators in bib data
% separate bibliographical hints and page hints by a comma
\jurabibsetup{commabeforerest}

%%% (4) specific configuration of bibdata in quotes / footnote
% use a.a.O if possible
\jurabibsetup{ibidem=strict}
% replace ugly a.a.O. by ders., a.a.O. resp. ders., ebda.
% but if there are more than one author or girl writers?
\AddTo\bibsgerman{
  \renewcommand*{\ibidemname}{Ds.,\ a.a.O.}
  \renewcommand*{\ibidemmidname}{ds.,\ a.a.O.}
}
\renewcommand*{\samepageibidemname}{Ds.,\ ebda.}
\renewcommand*{\samepageibidemmidname}{ds.,\ ebda.}

%%% (5) specific configuration of bibdata in bibliography
% ever an in: before journal and collection/book-titles 

\renewcommand*{\bibjtsep}{in:\ }
\renewcommand*{\bibbtsep}{in:\ }

% ever a colon after author names 
\renewcommand*{\bibansep}{:\ }
% ever a semi colon after the title 
\renewcommand*{\bibatsep}{;\ }
% ever a comma before date/year
\renewcommand*{\bibbdsep}{,\ }

% let jurabib insert the S. and p. information
% no S. necessary in bib-files and in cites/footcites
\jurabibsetup{pages=format}

% use a compressed literature-list using a small line indent
\jurabibsetup{bibformat=compress}
\setlength{\jbbibhang}{1em}

% which follows the design of the cites and offers comments
\jurabibsetup{biblikecite}

% print annotations into bibliography
\jurabibsetup{annote}
\renewcommand*{\jbannoteformat}[1]{{ \itshape #1 }}

%refine the prefix of url download
\AddTo\bibsgerman{\renewcommand*{\urldatecomment}{Referenzdownload: }}

% we want to have the year of articles in brackets
\renewcommand*{\bibaldelim}{(}
\renewcommand*{\bibardelim}{)}

%Umformatierung des Reihentitels und der Reihennummer
\DeclareRobustCommand{\numberandseries}[2]{%
\unskip\unskip%,
\space\bibsnfont{(=~#2}%
\ifthenelse{\equal{#1}{}}{)}{, [Bd./Nr.]~#1)}%
}%

%Umformatierung Referenzverweises
\usepackage{xpatch}
\AfterFile{dejbbib.ldf}{%
  \xapptocmd{\bibsgerman}{%
     \def\inname{\ifjboxford in:\else\ifjbchicago in:\else in:\fi\fi}%
    \def\incollinname{\ifjboxford in:\else\ifjbchicago in:\else in:\fi\fi}%
  }{}{}%
}

% the field printed before ISBN, ISSN or URL is the bibfield note
% Hence: If you insert into the field note the type of the literature
% [ Print | [FreeWeb | BibWeb] / [ PDF | HTML ] ] then you now
% get entries like:
% Print: ISBN ....
% BibWeb / PDF => http...
% That's nice for dealing with electronic sources correctly
\DeclareRobustCommand{\jbissn}[1]{\unskip:\space ISSN #1}%
\DeclareRobustCommand{\jbisbn}[1]{\unskip:\space ISBN #1}%

\DeclareRobustCommand{\biburlprefix}{$\Rightarrow$ }
\DeclareRobustCommand{\biburlsuffix}{}



% language specific hyphenation
%mycsrfk Hyphenation Include Module text
%
% (c) Karsten Reincke, Frankfurt a.M. 2012, ff.
%
% This file is licensed under the Creative Commons Attribution 3.0 Germany
% License (http://creativecommons.org/licenses/by/3.0/de/): 
% For details see teh file LICENSE in the top directory
%


\hyphenation{ Mehr-stimmig-keit Musik-wissen-schaft-ler}



%%% (3) layout page configuration %%%

% select the visible parts of a page
% S.31: { plain|empty|headings|myheadings }
%\pagestyle{myheadings}
\pagestyle{headings}

% select the wished style of page-numbering
% S.32: { arabic,roman,Roman,alph,Alph }
\pagenumbering{arabic}
\setcounter{page}{1}

% select the wished distances using the general setlength order:
% S.34 { baselineskip| parskip | parindent }
% - general no indent for paragraphs
\setlength{\parindent}{0pt}
\setlength{\parskip}{1.2ex plus 0.2ex minus 0.2ex}


%%% (4) general package activation %%%
%\usepackage{utopia}
%\usepackage{courier}
%\usepackage{avant}
\usepackage[dvips]{epsfig}

% graphic
\usepackage{graphicx,color}
\usepackage{array}
\usepackage{shadow}
\usepackage{fancybox}

\usepackage{tikz}
\usetikzlibrary{arrows}
\usetikzlibrary{shapes,snakes}
\usetikzlibrary{positioning}
\usetikzlibrary{decorations.text}
\usetikzlibrary{trees}
\usetikzlibrary{matrix}

\usepackage{amsmath}
\usepackage{amsfonts}
\usepackage{amssymb}
\usepackage{wasysym}
\usepackage{chngcntr}


%- start(footnote-configuration)

\deffootnote[1.5em]{1.5em}{1.5em}{\textsuperscript{\thefootnotemark)\ }}

% if document class = book: count footnotes from start to end
% \counterwithout{footnote}{chapter}
%- end(footnote-configuration)

% package for macking tables with broken lines
\usepackage{multirow}

%for using label as nameref
\usepackage{nameref}

%integrate nomenclature
% mycsrf  Deutsch Nomenclation Declaration Include Module 
%
% (c) Karsten Reincke, Frankfurt a.M. 2012, ff.
%
% This file is licensed under the Creative Commons Attribution 3.0 Germany
% License (http://creativecommons.org/licenses/by/3.0/de/): 
% For details see teh file LICENSE in the top directory

\usepackage[intoc]{nomencl}
\let\abbr\nomenclature
% Deutsche Überschrift
%\renewcommand{\nomname}{Abbreviations}
\renewcommand{\nomname}{Abkürzungen}

\setlength{\nomlabelwidth}{.20\hsize}
\renewcommand{\nomlabel}[1]{#1 \dotfill}
% reduce the line distance
\setlength{\nomitemsep}{-\parsep}
\makenomenclature


% depth of contents
\setcounter{secnumdepth}{5}
\setcounter{tocdepth}{5}

% Hyperlinks
\usepackage{hyperref}
\hypersetup{bookmarks=true,breaklinks=true,colorlinks=true,citecolor=blue,draft=false}


\usepackage{musixtex}
% Unfortunately musixtex still uses outdated commands for
% establishing its own \bar command. Hence for enabling
% the use of musixtex we must 'redefine' these outdated commands:
\makeatletter
\DeclareOldFontCommand{\rm}{\normalfont\rmfamily}{\mathrm}
\DeclareOldFontCommand{\sf}{\normalfont\sffamily}{\mathsf}
\DeclareOldFontCommand{\tt}{\normalfont\ttfamily}{\mathtt}
\DeclareOldFontCommand{\bf}{\normalfont\bfseries}{\mathbf}
\DeclareOldFontCommand{\it}{\normalfont\itshape}{\mathit}
\DeclareOldFontCommand{\sl}{\normalfont\slshape}{\@nomath\sl}
\DeclareOldFontCommand{\sc}{\normalfont\scshape}{\@nomath\sc}
\makeatother 

\begin{document}

%% use all entries of the bliography
\nocite{*}

%%-- start(titlepage)
\titlehead{Musikwissenschaft}
\subject{Release \input{rel.inc}}
\title{mycsrf mit MusixTex}
\subtitle{Über die Einbettung von MusixTex in wissenschaftliche Texte}
\author{Karsten Reincke% mycsrf License Include Module
%
% (c) Karsten Reincke, Frankfurt a.M. 2012, ff.
%
% This file is licensed under the Creative Commons Attribution 3.0 Germany
% License (http://creativecommons.org/licenses/by/3.0/de/): 
% For details see teh file LICENSE in the top directory
%

\footnote{\textbf{This file is distributed under the terms of license XYZ}
Here, you can insert your conditions for using your text. Good examples
for such licenses are offered under \texttt{https://creativecommons.org/}. 
Traditionally it also possible to say : \emph{All rights reserved}.
In accordance to the license \texttt{CC BY 3.0 DE}, under which mycrsf
is released, you must finally point to mycsrf:
\newline 
{ \tiny \itshape [Format derived from \texttt{mind your Scholar Research
Framework} \copyright K. Reincke CC BY 3.0 DE http://fodina.de/mycsrf)] }}

}

%thanks entry cannot be combined with license footnote
%\thanks{den Autoren von KOMA-Script und denen von Jurabib}

\maketitle
%%-- end(titlepage)

\footnotesize
\tableofcontents

\normalsize

\section{Einleitung}
In dem Tutorial zur Nutzung von MusixTex wird mehrfach erwähnt, dass man in
einen LaTex-Text Noten als MusixTex basierten Code eher nicht einfügen wolle:
Schon das Vorwort sagt Anfängern, dass sie nicht mit diesem Tutorial beginnen
müssen, sondern sich lieber \emph{PMX} oder \emph{M-Tx} aneignen
sollte\footcite[vgl.][iii]{VogSimRyc2018a}. Selbstverständlich könne man
Texteditoren nutzen, um MusixTex-Befehle manuell in die Tex-Datei einzufügen.
Gleichwohl würden die meisten User es weniger anstrengend finden, die dabei
anstehenden Entscheidungen durch einen Präprozessor wie \emph{PMX} treffen zu
lassen\footcite[vgl.][1]{VogSimRyc2018a}. Wenn man allerdings normalen Text und
etwa Musikbeispiele in einem Dokument vereinen wolle, dann gäbe es kaum eine
bessere Erläutertung dazu als eben das genannte
Tutorial\footcite[vgl.][1]{VogSimRyc2018a}.

Wir teilen diese Einschätzung. Mehr noch: wir glauben sogar, dass für kleinere
Beispiele - wie etwa Kadenzen oder Motive - die direkte Einbettung von MusixTex
der produktivere Weg ist. Der folgende Abschnitte präsentiert einige auf diesen
Zweck ausgerichtete Beispiele.

\section{Beispiele}

\begin{music}
   \parindent10mm
   \instrumentnumber{1}               % a single instrument
   \setname1{Piano}                   % whose name is Piano
   \setstaffs1{2}                     % with two staffs
   \generalmeter{\meterfrac44}        % 4/4 meter chosen
   \startextract                      % starting real score
     \Notes\ibu0f0\qb0{cge}\tbu0\qb0g|\hl j\en
     \Notes\ibu0f0\qb0{cge}\tbu0\qb0g|\ql l\sk\ql n\en
     \bar
     \Notes\ibu0f0\qb0{dgf}|\qlp i\en
     \notes\tbu0\qb0g|\ibbl1j3\qb1j\tbl1\qb1k\en
     \Notes\ibu0f0\qb0{cge}\tbu0\qb0g|\hl j\en
     \bar
   \zendextract                       % terminate excerpt
\end{music}

\begin{music}
\hsize=100mm
\generalmeter{\meterfrac24}%
\parindent 0pt
\generalsignature{-3}
\nostartrule
\startpiece\bigaccid
\NOtes\qu{ce}\en\bar
\NOtes\qu{gh}\en\bar
\NOtes\qu{=b}\en
\Notes\ds\cu g\en\bar
\NOtes\qu{^f=f}\en\bar
\NOtes\qu{=e}\itied0e\qu{_e}\en\bar
\Notes\ttie0\Qqbu ed{_d}c\en\bar
\Notes\ibu0b{-2}\qb0{=b}\en
\notes\nbbu0\qb0{=a}\tqh0N\en
\Notes\Dqbu cf\en\bar
\NOtes\uptext{\it tr}\qu e\uptext{\it tr}\qu d\en\bar
\NOtes\qu c\qp\en\Endpiece
\end{music}


\small

% insert the nomenclature here

% mycsrf Deutsch Nomenclation Tokens Include Module 
%
% (c) Karsten Reincke, Frankfurt a.M. 2012, ff.
%
% This file is licensed under the Creative Commons Attribution 3.0 Germany
% License (http://creativecommons.org/licenses/by/3.0/de/): 
% For details see teh file LICENSE in the top directory

% specific abbreviations
\abbr[utb]{UTB}{Uni-Taschenbuch}
\abbr[stw]{stw}{suhrkamp taschenbuch wissenschaft}% mycsrf  Deutsch Nomenclation Tokens Include Module 

% general abbreviations
\abbr[vgl]{vgl.}{vergleiche}
\abbr[aaO]{a.a.O.}{am angegebenen Ort}
\abbr[ds]{ds.}{kollektiv für ders., dies., \ldots}
\abbr[ebda]{ebda.}{ebenda}
% \abbr[id]{id.}{idem = latin for 'the same', be it a man, woman or a group\ldots}
% \abbr[ibid]{ibid.}{ibidem = latin for 'at the same place'}
\abbr[ifross]{ifross}{Institut für Rechtsfragen der Freien und Open Source
Software}
% \abbr[lc]{l.c.}{loco citato = latin for 'in the place cited'}
\abbr[wp]{wp.}{webpage = Webdokument ohne innere Seitennummerierung}
% mycsrf English Nomenclation Tokens Include Module 
%
% (c) Karsten Reincke, Frankfurt a.M. 2012, ff.
%
% This file is licensed under the Creative Commons Attribution 3.0 Germany
% License (http://creativecommons.org/licenses/by/3.0/de/): 
% For details see teh file LICENSE in the top directory
%

\abbr[afda]{AfdA}{Anzeiger für deutsches Altertum}
%\abbr[zfda]{ZfdA}{Zeitschrift für deutsches Altertum und deutsche Literatur [ISSN: 00442518]}
%\abbr[zfaw]{}{Zeitschrift für Allgemeine Wissenschaftstheorie / Journal for General Philosophy of Science [ISSN: 0044-2216]}

\printnomenclature

% insert the bibliographical data here
\bibliography{bib/literature}

\end{document}
