% mycsrf cloak file
%
% (c) Karsten Reincke, Frankfurt a.M. 2010, 2011, ff.
%
% This file is licensed under the Creative Commons Attribution 3.0 Germany
% License (http://creativecommons.org/licenses/by/3.0/de/): 
% For details see teh file LICENSE in the top directory
%
% select the document class
% S.26: [ 10pt|11pt|12pt onecolumn|twocolumn oneside|twoside notitlepage|titlepage final|draft
%         leqno fleqn openbib a4paper|a5paper|b5paper|letterpaper|legalpaper|executivepaper openrigth ]
% S.25: { article|report|book|letter ... }
%
% oder koma-skript S.10 + 16
\documentclass[
  DIV=calc,
  BCOR=5mm,
  11pt,
  headings=small,
  oneside,
  abstract=true,
  toc=bib,
  english,ngerman]{scrartcl}
  
%%% (1) general configurations %%%
\usepackage[utf8]{inputenc}

%%% (2) language specific configurations %%%
\usepackage[]{a4,babel}
\selectlanguage{ngerman}

% package for improving the grey value and the line feed handling
\usepackage{microtype}

%language specific quoting signs
\usepackage{csquotes}

% jurabib configuration
\usepackage[see]{jurabib}
\bibliographystyle{jurabib}
% mycsrf German jurabib configuration include module file 
%
% (c) Karsten Reincke, Frankfurt a.M. 2012, ff.
%
% This file is licensed under the Creative Commons Attribution 3.0 Germany
% License (http://creativecommons.org/licenses/by/3.0/de/): 
% For details see teh file LICENSE in the top directory

% the first time cite with all data, later with shorttitle
\jurabibsetup{citefull=first}

%%% (1) author / editor list configuration
%\jurabibsetup{authorformat=and} % uses 'und' instead of 'u.'
% therefore define your own abbreviated conjunction: 
% an 'and before last author explicetly written conjunction

% for authors in citations
\renewcommand*{\jbbtasep}{\ u.\ } % bta = between two authors sep
\renewcommand*{\jbbfsasep}{,\ } % bfsa = between first and second author sep
\renewcommand*{\jbbstasep}{\ u.\ }% bsta = between second and third author sep
% for editors in citations
\renewcommand*{\jbbtesep}{\ u.\ } % bta = between two authors sep
\renewcommand*{\jbbfsesep}{,\ } % bfsa = between first and second author sep
\renewcommand*{\jbbstesep}{\ u.\ }% bsta = between second and third author sep

% for authors in literature list
\renewcommand*{\bibbtasep}{\ u.\ } % bta = between two authors sep
\renewcommand*{\bibbfsasep}{,\ } % bfsa = between first and second author sep
\renewcommand*{\bibbstasep}{\ u.\ }% bsta = between second and third author sep
% for editors  in literature list
\renewcommand*{\bibbtesep}{\ u.\ } % bte = between two editors sep
\renewcommand*{\bibbfsesep}{,\ } % bfse = between first and second editor sep
\renewcommand*{\bibbstesep}{\ u.\ }% bste = between second and third editor sep

% use: name, forname, forname lastname u. forname lastname
\jurabibsetup{authorformat=firstnotreversed}
\jurabibsetup{authorformat=italic}

%%% (2) title configuration
% in every case print the title, let it be seperated from the 
% author by a colon and use the slanted font
\jurabibsetup{titleformat={all,colonsep}}
%\renewcommand*{\jbtitlefont}{\textit}

%%% (3) seperators in bib data
% separate bibliographical hints and page hints by a comma
\jurabibsetup{commabeforerest}

%%% (4) specific configuration of bibdata in quotes / footnote
% use a.a.O if possible
\jurabibsetup{ibidem=strict}
% replace ugly a.a.O. by ders., a.a.O. resp. ders., ebda.
% but if there are more than one author or girl writers?
\AddTo\bibsgerman{
  \renewcommand*{\ibidemname}{Ds.,\ a.a.O.}
  \renewcommand*{\ibidemmidname}{ds.,\ a.a.O.}
}
\renewcommand*{\samepageibidemname}{Ds.,\ ebda.}
\renewcommand*{\samepageibidemmidname}{ds.,\ ebda.}

%%% (5) specific configuration of bibdata in bibliography
% ever an in: before journal and collection/book-titles 

\renewcommand*{\bibjtsep}{in:\ }
\renewcommand*{\bibbtsep}{in:\ }

% ever a colon after author names 
\renewcommand*{\bibansep}{:\ }
% ever a semi colon after the title 
\renewcommand*{\bibatsep}{;\ }
% ever a comma before date/year
\renewcommand*{\bibbdsep}{,\ }

% let jurabib insert the S. and p. information
% no S. necessary in bib-files and in cites/footcites
\jurabibsetup{pages=format}

% use a compressed literature-list using a small line indent
\jurabibsetup{bibformat=compress}
\setlength{\jbbibhang}{1em}

% which follows the design of the cites and offers comments
\jurabibsetup{biblikecite}

% print annotations into bibliography
\jurabibsetup{annote}
\renewcommand*{\jbannoteformat}[1]{{ \itshape #1 }}

%refine the prefix of url download
\AddTo\bibsgerman{\renewcommand*{\urldatecomment}{Referenzdownload: }}

% we want to have the year of articles in brackets
\renewcommand*{\bibaldelim}{(}
\renewcommand*{\bibardelim}{)}

%Umformatierung des Reihentitels und der Reihennummer
\DeclareRobustCommand{\numberandseries}[2]{%
\unskip\unskip%,
\space\bibsnfont{(=~#2}%
\ifthenelse{\equal{#1}{}}{)}{, [Bd./Nr.]~#1)}%
}%

%Umformatierung Referenzverweises
\usepackage{xpatch}
\AfterFile{dejbbib.ldf}{%
  \xapptocmd{\bibsgerman}{%
     \def\inname{\ifjboxford in:\else\ifjbchicago in:\else in:\fi\fi}%
    \def\incollinname{\ifjboxford in:\else\ifjbchicago in:\else in:\fi\fi}%
  }{}{}%
}

% the field printed before ISBN, ISSN or URL is the bibfield note
% Hence: If you insert into the field note the type of the literature
% [ Print | [FreeWeb | BibWeb] / [ PDF | HTML ] ] then you now
% get entries like:
% Print: ISBN ....
% BibWeb / PDF => http...
% That's nice for dealing with electronic sources correctly
\DeclareRobustCommand{\jbissn}[1]{\unskip:\space ISSN #1}%
\DeclareRobustCommand{\jbisbn}[1]{\unskip:\space ISBN #1}%

\DeclareRobustCommand{\biburlprefix}{$\Rightarrow$ }
\DeclareRobustCommand{\biburlsuffix}{}



% language specific hyphenation
%mycsrfk Hyphenation Include Module text
%
% (c) Karsten Reincke, Frankfurt a.M. 2012, ff.
%
% This file is licensed under the Creative Commons Attribution 3.0 Germany
% License (http://creativecommons.org/licenses/by/3.0/de/): 
% For details see teh file LICENSE in the top directory
%


\hyphenation{ Mehr-stimmig-keit Musik-wissen-schaft-ler}



%%% (3) layout page configuration %%%

% select the visible parts of a page
% S.31: { plain|empty|headings|myheadings }
%\pagestyle{myheadings}
\pagestyle{headings}

% select the wished style of page-numbering
% S.32: { arabic,roman,Roman,alph,Alph }
\pagenumbering{arabic}
\setcounter{page}{1}

% select the wished distances using the general setlength order:
% S.34 { baselineskip| parskip | parindent }
% - general no indent for paragraphs
\setlength{\parindent}{0pt}
\setlength{\parskip}{1.2ex plus 0.2ex minus 0.2ex}


%%% (4) general package activation %%%
%\usepackage{utopia}
%\usepackage{courier}
%\usepackage{avant}
\usepackage[dvips]{epsfig}

% graphic
\usepackage{graphicx,color}
\usepackage{array}
\usepackage{shadow}
\usepackage{fancybox}

\usepackage{tikz}
\usetikzlibrary{arrows}
\usetikzlibrary{shapes,snakes}
\usetikzlibrary{positioning}
\usetikzlibrary{decorations.text}
\usetikzlibrary{trees}
\usetikzlibrary{matrix}

\usepackage{amsmath}
\usepackage{amsfonts}
\usepackage{amssymb}
\usepackage{wasysym}
\usepackage{chngcntr}


%- start(footnote-configuration)

\deffootnote[1.5em]{1.5em}{1.5em}{\textsuperscript{\thefootnotemark)\ }}

% if document class = book: count footnotes from start to end
% \counterwithout{footnote}{chapter}
%- end(footnote-configuration)

% package for macking tables with broken lines
\usepackage{multirow}

%for using label as nameref
\usepackage{nameref}

%integrate nomenclature
% mycsrf  Deutsch Nomenclation Declaration Include Module 
%
% (c) Karsten Reincke, Frankfurt a.M. 2012, ff.
%
% This file is licensed under the Creative Commons Attribution 3.0 Germany
% License (http://creativecommons.org/licenses/by/3.0/de/): 
% For details see teh file LICENSE in the top directory

\usepackage[intoc]{nomencl}
\let\abbr\nomenclature
% Deutsche Überschrift
%\renewcommand{\nomname}{Abbreviations}
\renewcommand{\nomname}{Abkürzungen}

\setlength{\nomlabelwidth}{.20\hsize}
\renewcommand{\nomlabel}[1]{#1 \dotfill}
% reduce the line distance
\setlength{\nomitemsep}{-\parsep}
\makenomenclature


% depth of contents
\setcounter{secnumdepth}{5}
\setcounter{tocdepth}{5}

% Hyperlinks
\usepackage{hyperref}
\hypersetup{bookmarks=true,breaklinks=true,colorlinks=true,citecolor=blue,draft=false}


\usepackage{musixtex}
% Unfortunately musixtex still uses outdated commands for
% establishing its own \bar command. Hence for enabling
% the use of musixtex we must 'redefine' these outdated commands:
\makeatletter
\DeclareOldFontCommand{\rm}{\normalfont\rmfamily}{\mathrm}
\DeclareOldFontCommand{\sf}{\normalfont\sffamily}{\mathsf}
\DeclareOldFontCommand{\tt}{\normalfont\ttfamily}{\mathtt}
\DeclareOldFontCommand{\bf}{\normalfont\bfseries}{\mathbf}
\DeclareOldFontCommand{\it}{\normalfont\itshape}{\mathit}
\DeclareOldFontCommand{\sl}{\normalfont\slshape}{\@nomath\sl}
\DeclareOldFontCommand{\sc}{\normalfont\scshape}{\@nomath\sc}
\makeatother 

\usepackage{harmony}

\begin{document}

%% use all entries of the bliography
\nocite{*}

%%-- start(titlepage)
\titlehead{Musikwissenschaft}
\subject{Release \input{rel.inc}}
\title{mycsrf mit MusixTex \& Harmony}
\subtitle{Die Musikanalyse in wissenschaftlichen Texten}
\author{Karsten Reincke% mycsrf License Include Module
%
% (c) Karsten Reincke, Frankfurt a.M. 2012, ff.
%
% This file is licensed under the Creative Commons Attribution 3.0 Germany
% License (http://creativecommons.org/licenses/by/3.0/de/): 
% For details see teh file LICENSE in the top directory
%

\footnote{\textbf{This file is distributed under the terms of license XYZ}
Here, you can insert your conditions for using your text. Good examples
for such licenses are offered under \texttt{https://creativecommons.org/}. 
Traditionally it also possible to say : \emph{All rights reserved}.
In accordance to the license \texttt{CC BY 3.0 DE}, under which mycrsf
is released, you must finally point to mycsrf:
\newline 
{ \tiny \itshape [Format derived from \texttt{mind your Scholar Research
Framework} \copyright K. Reincke CC BY 3.0 DE http://fodina.de/mycsrf)] }}

}

%thanks entry cannot be combined with license footnote
%\thanks{den Autoren von KOMA-Script und denen von Jurabib}

\maketitle
%%-- end(titlepage)

\footnotesize
\tableofcontents

\normalsize

\section*{Einleitung} Das \emph{MusiXTeX}-Tutorial  erwähnt mehrfach, dass man
in einen \emph{LaTeX}-Text Notenbeispiele eher nicht 'händisch' einfügen wolle:
Anfänger sollten - wie es heißt - nicht mit diesem Tutorial beginnen,
sondern lieber gleich \emph{PMX} lernen\footcite[vgl.][iii]{VogSimRyc2018a}.
Zwar könne man \emph{MusiXTeX}-Befehle durchaus auch manuell in eine
\emph{LaTeX}-Datei einfügen. Gleichwohl würden es die meisten weniger
anstrengend finden, die dabei anstehenden Entscheidungen durch einen
Präprozessor wie \emph{PMX} treffen zu lassen\footcite[vgl.][1]{VogSimRyc2018a}.
Wolle man jedoch Fließtext und Musik in einem Dokument
vereinen, dann sei die direkte Verwendung von \emph{MusiXTeX} durchaus eine
Alternative\footcite[vgl.][1]{VogSimRyc2018a}.

Wir teilen diese Einschätzung. Mehr noch: wir glauben sogar, dass für kleinere
Beispiele - wie etwa Kadenzen, Motive oder Analysen - die direkte Einbettung von
\emph{MusiXTeX} der produktivere Weg ist\footnote{Ganz ähnlich siehen es die
Autoren des genannten Handbuchs, wenn sie im 24. Kapitel genauer erläutern, wie
man \emph{MusiXTeX} basiert Notenbeispiele in einen Text einbetten möchte.
\cite[vgl.][113ff]{VogSimRyc2018a}.}. Deshalb hier einige auf diesen Zweck
ausgerichtete Beispiele\footnote{Deren Nutzung setzt außer der Installation von
\emph{LaTeX} und \emph{MusiXTeX} nur die Zeile
\texttt{\textbackslash{usepackage\{musixtex\}}} in der Präambel voraus.
Das für \emph{MusiXTeX} konstitutive \enquote{Three Pass System} ist bereits in
das \emph{mycsrf}-Makefile  eingearbeitet worden.}:

\section{Einzeilige Kadenz}

\begin{music}%
  \largemusicsize%
  % using defaults: \instrumentnumber{1}% + \setstaffs{1}{1}  
  % + \setclef{1}{\treble}+ no bar type + \generalsignature{0}%
  \nobarnumbers%
  \startextract%
  \setdoublebar%
  \NOTEs\lcharnote{10}{(1) }\uptext{T}\zchar{-10}{I}\zw{ce}\wh{g}\en%
  \NOTEs\uptext{S}\zchar{-10}{IV}\zw{fh}\wh{j}\en%
  \NOTEs\uptext{D}\zchar{-10}{V}\zw{gi}\wh{k}\en%
  \bar%
  \NOTEs\lcharnote{10}{(2) }\uptext{T}\zchar{-10}{I}\zw{ac}\wh{e}\en% 
  \NOTEs\uptext{S}\zchar{-10}{IV}\zw{df}\wh{h}\en%
  \NOTEs\uptext{D}\zchar{-10}{{I}}\sh{g}\zw{eg}\wh{i}\en%
  \setdoublebar%
  \endextract%
\end{music}%
\vspace{0.5cm}
Dieses 'Extrakt'  bildet das Beispiel 187 aus der Harmonielehre von Grabner
nach\footcite[vgl.][107]{Grabner1974a}. Man sieht, dass man mit MusixTex und
LaTeX dem 'Original' recht nahe kommt. Es wird durch folgenden
\emph{MusiXTeX}-Code generiert:
\begin{verbatim}
\begin{music}%
  \largemusicsize%
  % using defaults: \instrumentnumber{1}% + \setstaffs{1}{1}  
  % + \setclef{1}{\treble} + no bar type + \generalsignature{0}%
  \nobarnumbers%
  \startextract%
  \setdoublebar%
  \NOTEs\lcharnote{10}{(1) }\uptext{T}\zchar{-10}{I}\zw{ce}\wh{g}\en%
  \NOTEs\uptext{S}\zchar{-10}{IV}\zw{fh}\wh{j}\en%
  \NOTEs\uptext{D}\zchar{-10}{V}\zw{gi}\wh{k}\en%
  \bar%
  \NOTEs\lcharnote{10}{(2) }\uptext{T}\zchar{-10}{I}\zw{ac}\wh{e}\en% 
  \NOTEs\uptext{S}\zchar{-10}{IV}\zw{df}\wh{h}\en%
  \NOTEs\uptext{D}\zchar{-10}{{I}}\sh{g}\zw{eg}\wh{i}\en%
  \setdoublebar%
  \endextract%
\end{music}%
\end{verbatim}

Zumindest prinzipell gibt es auch die Möglichkeit, Notentext in den Fließtext
einzubetten, indem man den Absatzumbruch per 
\texttt{{\textbackslash}let{\textbackslash}extractline{\textbackslash}relax}
unterbindet:
\begin{music}%
  \nostartrule \smallmusicsize
  \let\extractline\relax
  \staffbotmarg0pt 
  \startextract
  \NOTEs\zw{e}\wh{g}\en
  \NOTEs\zw{h}\wh{j}\en
  \NOTEs\zw{i}\wh{k}\en
  \setdoublebar
  \endextract
\end{music} . Gleichwohl wird das Ergebnis meist unschön sein, weil eine
Notensatzzeile selbst mit sehr reduzierten Informationen immer noch deutlich
höher ist, als eine normale Textzeile, sodass der Zeilenfluss unterbrochen und der
gleichmäßige Grauwert zerstört wird.


\section{Zweizeilige Kadenz in einem System}

\begin{music}
  \normalmusicsize
  \parindent4em
  \instrumentnumber{1}
  \setstaffs{1}{2}
  \setclef{2}{\treble}
  \setclef{1}{\bass}
  \setname{1}{Piano}
  \generalsignature{2}% D-DUR
  \generalmeter{\meterfrac42}
  \startextract
  \NOTes\zmidstaff{T}\zh{K}\hl{a}|\zh{f}\hu{k}\en%
  \NOTes\zmidstaff{S}\zh{G}\hl{N}|\zh{i}\hu{k}\en%
  \NOTes\zmidstaff{D\textsuperscript{7}}\zh{J}\hl{N}|\zh{h}\hu{l}\en%
  \NOTes\zmidstaff{T}\zh{K}\hl{M}|\zh{h}\hu{k}\en%
  \bar
  \NOTes\zmidstaff{T}\zh{K}\hl{a}|\zh{k}\hu{m}\en%  
  \NOTes\zmidstaff{S}\zh{N}\hl{b}|\zh{g}\hu{k}\en%  
  \NOTes\zmidstaff{D\textsuperscript{4 -- 3}}\zhl{a}\hl{c}|
    \zh{e}\isluru{0}{k}\qu{k}\tslur{0}{j}\qu{j}\en%  
  \NOTes\zmidstaff{T}\zh{K}\hl{a}|\zh{f}\hu{k}\en%  
  \setdoublebar
  \endextract
\end{music}

\emph{MusiXTex} verwendet zwei Methoden, zu instrumentieren: Die eine ordnet
jedem Instrument eine Stimme zu, die andere fasst mehrere Stimmen resp. Systeme
- wie bei einem Klaviertext - zu einem Instrument zusammen. Hier eine solche
Kadenz, die durch folgenden \emph{MusiXTeX}-Code generiert wird:
\begin{verbatim}
\begin{music}
  \normalmusicsize
  \instrumentnumber{1}
  \setstaffs{1}{2}
  \setclef{2}{\treble}
  \setclef{1}{\bass}
  \generalsignature{2}% D-DUR
  \generalmeter{\meterfrac42}
  \startextract % statt: startpiece
  \NOTes\zmidstaff{T}\zh{K}\hl{a}|\zh{f}\hu{k}\en%
  \NOTes\zmidstaff{S}\zh{G}\hl{N}|\zh{i}\hu{k}\en%
  \NOTes\zmidstaff{D\textsuperscript{7}}\zh{J}\hl{N}|\zh{h}\hu{l}\en%
  \NOTes\zmidstaff{T}\zh{K}\hl{M}|\zh{h}\hu{k}\en%
  \bar
  \NOTes\zmidstaff{T}\zh{K}\hl{a}|\zh{k}\hu{m}\en%  
  \NOTes\zmidstaff{S}\zh{N}\hl{b}|\zh{g}\hu{k}\en%  
  \NOTes\zmidstaff{D\textsuperscript{4 -- 3}}\zhl{a}\hl{c}|\zh{e}\qu{kj}\en%  
  \NOTes\zmidstaff{T}\zh{K}\hl{a}|\zh{f}\hu{k}\en%  
  \setdoublebar
  \endextract % statt: endpiece
\end{music}
\end{verbatim}

\section{Zweizeilige Kadenz in zwei Systemen}
Bei der anderen Methode wird jedem Instrument ein eigenes Notenliniensystem
zugeordnet. Entsprechend müssen Noten anders verteilt werden. Neben dem
zentrierten \texttt{\textbackslash{startextract}} steht bei \emph{MusiXTeX} auch
eine linksorientierte Darstellung \texttt{\textbackslash{startpiece}} zur
Verfügung, mit der ganze (mehrzeilige) Stücke in einen Text eingefügt werden
können, und zwar so, dass das erste Gesamtsystem für die Instrumentenbezeichnung
auch entsprechend eingerückt wird. Das folgende Beispiel zeigt die entsprechende
Integration in eine LaTeX-Datei\footnote{Da unser Beispiel nur drei Takte enthält,
die der eingefügten Analysesymbole wegen al\-lerdings knapp über den Rand des
Seitenspiegels hinausreichen würden, haben wir zu Demonstrationszwecken die
erlaubte Breite mit dem Kommando \texttt{\textbackslash{hsize=100mm}} verringert
und so einen Umbruch auf zwei Zeilen erzwungen}:

\begin{music}
  \hsize=100mm
  \smallmusicsize
  \parindent4em
  \instrumentnumber{2}
  \setstaffs{2}{1}
  \setstaffs{1}{1}
  \setclef{2}{\treble}
  \setclef{1}{\bass}
  \songtop{2}
  \songbottom{1}
  \setname1{Diskant}
  \setname2{Bass}
  \generalsignature{-2}% ES-DUR
  \generalmeter{\meterfrac58}
  \startpiece
  % Takt 1/1: 2 8tel abgekürzt + 4tel Akkord
  \NOtes \zmidstaff{T} \Dqbl I b & \zq{ik}\qu{m} \en
  % Takt 1/3: 2 8tel explizit  mit Vorzeichen + 4tel Akkord
  \NOtes \zmidstaff{\HH.D.3-3$\flat$.8.7..} \ibl{0}{a}{0}\qb{0}{a}\tbl{0}\fl{a}\qb{0}{a} & \zq{j}\rq{l}\qu{m} \en
  % Takt 1/5: punktierte 16tel explizit mit Vorzeichen + 8el Akkord
  \NOtes \zmidstaff{\HH.S.3-3$\flat$....} \ibbl{2}{N}{0}\qbp{2}{N}\roff{\tbbbl{2}\fl{N}\tqb{2}{N}} & \zq{il}\cu{p}\en
  \bar
  % Takt 2/1: 2 8tel abgekürzt + 4tel Akkord
  \NOtes \zmidstaff{\HH.D.8-7.7...} \Dqbl M L & \zq{jl}\qu{o} \en
   % Takt 2/3: punktierte 8tel  + 4tel Akkord
  \NOtes \zmidstaff{\HH.Dp.8-8$\flat$....} \ibl{1}{K}{0}\qbp{1}{K}\roff{\tbbl{1}\fl{K}\tqb{1}{K}} & \zq{hk}\qu{m}\en
  % Takt 2/5: punktierte 16tel explizit mit Vorzeichen + 8el Akkord 
  \NOtes \zmidstaff{\HH.D.5-3.7...} \ibbl{2}{J}{0}\qbp{2}{J}\roff{\tbbl{2}\tqb{2}{H}} & \zq{j}\rq{l}\cu{m} \en%
  \bar
  \notes \zmidstaff{T} \ca I & \ds \en
%   % Takt 3/2: 
   \notes \ds & \zq{fk}\cu{m}\en
   % Takt 3/3: 
   \notes \zmidstaff{\HH.D..7...} \ca M & \zq{eh}\cu{j}\en
   % Takt 3/4-5: 
   \notes \zmidstaff{T} \qa b & \zq{df}\qu{i}\en
  \Endpiece 
\end{music}

Dieses Beispiel enthält Analysesymbole, die nicht von \emph{MusiXTeX} selbst
bereitgestellt werden. Unser folgendes Kapitel zeigt, wie man
diese 'Fachsprache' typographish trotzdem integrieren kann.

\section{Harmonische Analyse}

\emph{MusiXTeX} selbst trägt wenig zur funktionstheoretischen Analyse bei. Man
kann zwar\footnote{wie in Beispiel 1 und 2 gezeigt} beliebigen Text unter über
oder zwischen die Systeme einfügen. Ohne weitere Maßnahme stehen dafür aber nur
die Zeichen zur Verfügung, die LaTeX sonst auch bietet.

Die harmonische Analyse hat allderings - und zwar insbesondere dort, wo sie die
Notation der Funktionstheorie mit der Generalbassnotation samt Vorhalten,
Durchgängen und Alterationen vereinigt - eine eigene, theoriespezifische
Symbolschrift entwickelt.

Das Paket \emph{harmony}\footcite[vgl.][1ff]{WegWeg2007a} - einzubinden über
\texttt{\textbackslash{usepackage\{harmony\}}} - stellt die entsprechenden Zeichen
unter LaTeX bereit. So können spezifische Funktionssymbole genutzt (\Dohne ,
\DD , \DS , \Ohne[2pt]{S}  \ldots), komplexe Akkorde detalliert beschrieben
(\HH.D.3.7... , \Fermi{\HH.D.3.9\VM-8.7.5.} ) und Rhythmen innerhalb einer
Zeile spezifiziert werden: \Halb ~ \Acht ~ \Vier ~ \Acht

Die in solchen Analysen auch zu verwenden Alterations- und Auflösungszeichen
stellt allerdings auch \emph{harmony} nicht selbst zur Verfügung. Stattdessen
werden sie von LaTeX direkt beigesteuert und können im Mathematikmodus als
\texttt{\$\textbackslash{sharp}\$ = $\sharp$}, \texttt{\$\textbackslash{flat}\$
= $\flat$} und \texttt{\$\textbackslash{natural}\$ = $\natural$} direkt abgerufen
werden. Sie sind mit \emph{harmony} und \emph{MusiXTeX} kombinierbar.

\section{Fazit}

Tatsächlich sind Notenbeispiele per \emph{MusiXTeX} nur mit Aufwand zu erzeugen.
Der Wunsch nach einem graphischen Editor ist bei der Arbeit allgegenwärtig.
Gerade bei der Verknüpfung von Text, Musik, und feinerer harmonischer Analyse
punkten aber die reinen Auszeichnungssprachen \emph{LaTeX}, \emph{MusiXTeX}
\emph{harmony}.

\small

% insert the nomenclature here

% mycsrf Deutsch Nomenclation Tokens Include Module 
%
% (c) Karsten Reincke, Frankfurt a.M. 2012, ff.
%
% This file is licensed under the Creative Commons Attribution 3.0 Germany
% License (http://creativecommons.org/licenses/by/3.0/de/): 
% For details see teh file LICENSE in the top directory

% specific abbreviations
\abbr[utb]{UTB}{Uni-Taschenbuch}
\abbr[stw]{stw}{suhrkamp taschenbuch wissenschaft}% mycsrf  Deutsch Nomenclation Tokens Include Module 

% general abbreviations
\abbr[vgl]{vgl.}{vergleiche}
\abbr[aaO]{a.a.O.}{am angegebenen Ort}
\abbr[ds]{ds.}{kollektiv für ders., dies., \ldots}
\abbr[ebda]{ebda.}{ebenda}
% \abbr[id]{id.}{idem = latin for 'the same', be it a man, woman or a group\ldots}
% \abbr[ibid]{ibid.}{ibidem = latin for 'at the same place'}
\abbr[ifross]{ifross}{Institut für Rechtsfragen der Freien und Open Source
Software}
% \abbr[lc]{l.c.}{loco citato = latin for 'in the place cited'}
\abbr[wp]{wp.}{webpage = Webdokument ohne innere Seitennummerierung}
%% mycsrf English Nomenclation Tokens Include Module 
%
% (c) Karsten Reincke, Frankfurt a.M. 2012, ff.
%
% This file is licensed under the Creative Commons Attribution 3.0 Germany
% License (http://creativecommons.org/licenses/by/3.0/de/): 
% For details see teh file LICENSE in the top directory
%

\abbr[afda]{AfdA}{Anzeiger für deutsches Altertum}
%\abbr[zfda]{ZfdA}{Zeitschrift für deutsches Altertum und deutsche Literatur [ISSN: 00442518]}
%\abbr[zfaw]{}{Zeitschrift für Allgemeine Wissenschaftstheorie / Journal for General Philosophy of Science [ISSN: 0044-2216]}

\printnomenclature

% insert the bibliographical data here
\bibliography{bib/literature}

\end{document}
