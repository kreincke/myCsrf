% mycsrf 'for beeing included' snippet template
%
% (c) Karsten Reincke, Frankfurt a.M. 2012, ff.
%
% This text is licensed under the Creative Commons Attribution 3.0 Germany
% License (http://creativecommons.org/licenses/by/3.0/de/): Feel free to share
% (to copy, distribute and transmit) or to remix (to adapt) it, if you respect
% how you must attribute the work in the manner specified by the author(s):
% \newline
% In an internet based reuse please link the reused parts to mycsrf.fodina.de
% and mention the original author Karsten Reincke in a suitable manner. In a
% paper-like reuse please insert a short hint to mycsrf.fodina.de and to the
% original author, Karsten Reincke, into your preface. For normal quotations
% please use the scientific standard to cite
%


%% use all entries of the bibliography

\subsection{Lilypond}

Das \enquote{Notensatzsystem} \emph{LilyPond} tritt unter dem Slogan
\enquote{Notensatz für jedermann} an, \enquote{[\ldots] Notendruck in (bester)
Qualität} zu ermöglichen, also \enquote{[\ldots] die Ästhetik handgestochenen
traditionellen Notensatzes mit computergesetzten Noten zu
erreichen}\footcite[vgl.][\nopage wp]{LilyPond2018x}. Was das systemisch
bedeutet, hat das LilyPond-Entwicklerteam in einem besonderen Artikel
dargestellt\footcite[vgl.][5ff]{LilyPond2018e}. Um die entsprechenden Techniken
erfolgreich anzuwenden, wird man sich ein einfach strukturiertes
Lerntutorial\footcite[vgl.][20ff]{LilyPond2018b} aneignen und ein umfangreiches
Referenzhandbuch\footcite[vgl.][1ff]{LilyPond2018b} bereitlegen müssen.

Wie die anderen bisher diskutierten Systeme auch, erwartet \emph{LilyPond}, dass
man Code schreibt, keine Noten: Der Texteditor ist das bevorzugte Mittel, um
Musik im \emph{LilyPond}-Dialekt zu notieren.

Trotzdem gibt es Unterschiede, die über die bloße Syntax hinausgehen: 

Die wichtigste Eigenart dürfte sein, dass Lilypond konsequent zwischen Musik und
Druck unterscheidet: Wer in D-Dur ein fis einfügen möchte, kann sich hier nicht
auf die zu Beginn gesetzte Tonartssignatur 'verlassen', er muss trotzdem statt
'f' 'fis' tippen, an jeder Stelle, wo er 'fis' meint. Das hat den gewaltigen
Vorteil, dass Lilypond bei alterierten Passagen die nötigen Vorzeichen
automatisch setzen kann: in g-moll erhält das fis ein Kreuz, in D-Dur nicht.

Und der auffälligste dürfte sein, dass LilyPond seine Nutzer konsequent dazu
anhält, in Elementen zu denken, die in einer 1:n-Beziehung verschachtelt sind:
Das Notenheft besteht aus einem oder mehreren Stücken, das Stück besteht aus
einem oder mehreren Notensystemen, ein Notensystem besteht aus einer oder
mehrerer Stimmen, die Stimme kann solistisch oder akkordisch sein. Die
Informatik würde sagen, dass Datenmodell ist als Baum angelegt. Die syntaktische
Markanten diese ineinandergreifenden Teile machen es sehr einfach, LilyPond-Code
zu verstehen.

Gelegentlich wird gesagt, dass LilyPond sich historisch aus TeX entwickelt habe
(zit), Wikipedia meint sogar, dass seine Syntax der von \TeX\ ähnelt. Beides
dürfte nur sehr bedingt stimmen. Lilypond selbst führt sein Entstehen

Der entscheidende Punkt hierbei ist aber, dass LilyPond (heute) systemisch
gesehen nichts (mehr) \LaTeX\ oder \TeX\  zu tun hat: es nutzt seine eigene
Eingabesprache und seine eigene Maschine zum erzeugen des Notencodes. Als
Ausgabeformate bietet lilypond PDF und PS \ldots an. Von der Softwarearchektur
habe wir hier also die gleiche Struktur, wie bei ABC oder MusiX\TeX: ein
Probgramm -- früher \texttt{abcm2ps} oder \texttt{musixtex}, hier
\texttt{lilypond} -- nimmt eine Datei mit passendem Inhalt und passender
Extension -- früher \texttt{.abc} oder \texttt{.tex}, hier
\texttt{.ly} -- und erzeugt daraus eine Graphik im PS, PDF oder PNG Format.

\subsubsection{Technische Voraussetzungen}

Damit wissen wir, dass wir auch hier einiges vorzubereiten haben, bevor wir
\emph{LilyPond} erfolgreich verwenden können.

$\RHD$ Zunächst muss -- ganz unabhängig von \LaTeX\ -- \emph{LilyPond}
installiert werden\footnote{Unter Ubuntu: \texttt{sudo apt-get lilypond
lilypond-data}}. Dieses Paket stellt auch LilyPond-Book bereit, ein Tool, mit
dem -- analog zum ABC-Verfahren -- im  ersten Durchgang des \LaTeX basierten
PDF-Erzeugungsprozess aus der \LaTeX-Datei die \emph{LilyPond}-Daten extrahiert
und in ein Postscriptbild umgewandelt werden, das dann bei der nächsten Runde --
automatisiert -- anstelle des \emph{LilyPond}-Codes in die \LaTeX-Datei
eingesetzt wird.
  
$\RHD$ Sofern es die \TeX-Distribution oder das Lilypond-Paket nicht eh schon
mitliefert, muss noch das lilypond-\LaTeX-Paket\footcite[vgl.][\nopage
wp]{CtanAbc2018a} installiert werden\footnote{Bei Ubuntu 18.04 reicht
Installation der Pakete \emph{texlive-music} und Lilypond aus.}.
  
$\RHD$ Danach muss -- wie bei \LaTeX\ üblich -- dieses
lilypondbook-\LaTeX-Paket  mit dem entsprechenden Kommando in der Präambel
eingebunden (\texttt{\textbackslash{usepackage}\{lilypond\}}).

%\footcite[vgl.][]{LilyPond2018a}.

% this is only inserted to eject fault messages in texlipse
%\bibliography{../bib/literature}
