% mycsrf 'for beeing included' snippet template
%
% (c) Karsten Reincke, Frankfurt a.M. 2012, ff.
%
% This text is licensed under the Creative Commons Attribution 3.0 Germany
% License (http://creativecommons.org/licenses/by/3.0/de/): Feel free to share
% (to copy, distribute and transmit) or to remix (to adapt) it, if you respect
% how you must attribute the work in the manner specified by the author(s):
% \newline
% In an internet based reuse please link the reused parts to mycsrf.fodina.de
% and mention the original author Karsten Reincke in a suitable manner. In a
% paper-like reuse please insert a short hint to mycsrf.fodina.de and to the
% original author, Karsten Reincke, into your preface. For normal quotations
% please use the scientific standard to cite
%


%% use all entries of the bibliography

\subsection{Lilypond}

\emph{LilyPond} möchte guten \enquote{Notensatz für jedermann} anbieten: Als
elektronisches \enquote{Notensatzsystem} wolle es \enquote{[\ldots] Notendruck
in (bester) Qualität} ermöglichen, also \enquote{[\ldots] die Ästhetik
handgestochenen traditionellen Notensatzes mit computergesetzten Noten [\ldots]
erreichen}\footcite[vgl.][\nopage wp]{LilyPond2018a}. In einem besonderen
Artikel haben die LilyPond-Entwickler dargestellt, was das systemisch
bedeutet\footcite[vgl.][5ff]{LilyPond2018d} und welchen Konsequenzen sich daraus
für ein Notensatzprogramm ergeben\footcite[vgl.][8ff]{LilyPond2018d}. Der daraus
erwachsende Anspruch ist hoch:

\begin{quote}\begin{em}
  \enquote{LilyPond wurde geschaffen, um die Probleme zu lösen, die wir in
  existierenden Programmen gefunden haben und um schöne Noten zu schaffen, die
  die besten handgestochenen Partituren imitieren.}\footcite[vgl.][2]{LilyPond2018d}
\end{em}\end{quote}

Wer die entsprechenden Techniken erfolgreich anwenden will, kann auf ein einfach
strukturiertes Lerntutorial\footcite[vgl.][20ff]{LilyPond2018b} und ein kürzeres
Nutzungshandbuch\footcite[vgl.][1ff]{LilyPond2018e} zurückgreifen. Letztlich
wird er sich auch das umfangreiche
Referenzhandbuch\footcite[vgl.][1ff]{LilyPond2018c} bereitlegen wollen.

Wie die bisher diskutierten Systeme erwartet auch \emph{LilyPond}, dass man Code
schreibt, keine Noten: Hier wie da ist der Texteditor das bevorzugte Werkzeug,
um Musik im entsprechenden 'Dialekt' zu notieren. Trotzdem gibt es Unterschiede,
die über die bloße Syntax hinausgehen:

Die wichtigste Eigenart dürfte sein, dass Lilypond konsequent zwischen Musik und
Druck unterscheidet: Wer in D-Dur ein \emph{fis} einfügen möchte, kann sich hier
nicht auf die zu Beginn spezifizierte Tonart 'berufen', er muss trotzdem
\texttt{fis} tippen, nicht \texttt{f}, und zwar an jeder Stelle, wo er
\emph{fis} meint. Diese Abkehr von der Musikertradition hat einen gewichtigen
Vorteil: Lilypond kann bei alterierten Passagen die nötigen Vorzeichen
automatisch setzen. In \emph{g-moll} erhält das \emph{fis} ein Kreuz, in D-Dur
nicht\footcite[vgl.][21]{LilyPond2018b}.

Die augenfälligste Besonderheit dürfte jedoch sein, dass \emph{LilyPond} seine
Elemente konsequent in einer 1:n-Beziehung anordnet: Das Notenheft besteht aus
einem oder mehreren Stücken, das Stück besteht aus einem oder mehreren
Notensystemen, ein Notensystem besteht aus einer oder mehrerer Stimmen, die
Stimme kann solistisch oder akkordisch sein. Das Datenmodell ist mithin als Baum
ausgelegt. Und syntaktisch haben die Ebenen je eigene Markanten. Das macht das
Lesen und Verstehen von \emph{LilyPond} auf Dauer einfacher, es entsteht ein
klarerer Code\footcite[vgl.][40ff]{LilyPond2018b}.

Systemisch gesehen hat LilyPond (heute) nichts (mehr) \LaTeX\ , MusiX\TeX\ oder
\TeX\ zu tun: es nutzt seine eigene Eingabesprache und seine eigene Maschine zum
Erzeugen des Notenbildes: Als \enquote{Standardausgabeformat} -- heißt es --
seien \emph{PDF}\footnote{Portable Document Format} und
\emph{PS}\footnote{Postscript} gesetzt; außerdem könnten
\emph{SVG\footnote{Scalable Vector Graphics}-}, \emph{EPS\footnote{Encapsulated
PostScript}-} und \emph{PNG\footnote{Portable Network Graphics}-}Dateien erzeugt
werden\footcite[vgl.][481]{LilyPond2018c}.

\subsubsection{Technische Voraussetzungen}

\emph{LilyPond} sagt selbst, dass man Notenbeispiele in Form von Graphiken auch
manuell in den \LaTeX-Text einfügen könne, einfach in dem man -- zuerst und
unabhängig von \LaTeX -- die Graphiken erzeuge und sie danach mit \LaTeX-Mitteln
einbinde\footcite[vgl.][20]{LilyPond2018e}. Bei vielen Notenbeispielen kann das
allerdings aufwendig werden, insbesondere wenn man manuell die Länge der
Notenzeilen und die Graphikbreite auf die gewünschte Zeilenlänge des Dokumentes
ausrichten muss. 

Deshalb bietet \emph{LilyPond} -- neben \texttt{lilypond} als Tool zur Erzeugung
der Graphiken\footnote{und aller anderen Outputformate wie \emph{midi} u.Ä.m.}
-- auch das Tool \texttt{lilypond-book} an: es \enquote{automatisiert} die
manuelle Integration, in dem es die \enquote{[\ldots] Musik-Schnipsel aus Ihrem
Dokument (extrahiert), [\ldots] lilypond (aufruft) und [\ldots] die
resultierenden Bilder in Ihr Dokument (einfügt)}, wobei es \enquote{[\ldots] die
Länge der Zeilen und die Schriftgröße dabei [automatisch] (dem) Dokument
(anpasst}\footcite[vgl.][20]{LilyPond2018e}.


Wer diesen Weg beschreitet, muss
allerdings beachten, dass \texttt{lilypond-book} 


Damit wissen wir, dass wir auch hier einiges vorzubereiten haben, bevor wir
\emph{LilyPond} erfolgreich verwenden können:

$\RHD$ Zunächst muss -- ganz unabhängig von \LaTeX\ -- \emph{LilyPond}
installiert werden\footnote{Unter Ubuntu: \texttt{sudo apt-get lilypond
lilypond-data}}. Dieses Paket stellt dann auch LilyPond-Book bereit.
  
$\RHD$ In diesem Fall muss keine LilyPond-Paket in die \LaTeX-Präambel
eingebunden werden: Da \texttt{lilypond-book} immer zuerst und damit unabhängig
von \texttt{latex} bzw.
\texttt{pdflatex} aufgerufen wird

$\RHD$ Schließlich kann der \emph{LilyPond}-Quelltext der Notenbeispiele jeweils
in einer (virtuellen) Umgebung
\verb|\begin{lilypond}...\end{lilypond}|-\LaTeX-Umgebung eingegeben werden.
Virtuell ist diese Umgebung insofern, als \LaTeX ja nichts von \emph{LilyPond} weiß.
% \footcite[vgl.][]{LilyPond2018a}.



% this is only inserted to eject fault messages in texlipse
%\bibliography{../bib/literature}
