% mycsrf 'for beeing included' snippet template
%
% (c) Karsten Reincke, Frankfurt a.M. 2012, ff.
%
% This text is licensed under the Creative Commons Attribution 3.0 Germany
% License (http://creativecommons.org/licenses/by/3.0/de/): Feel free to share
% (to copy, distribute and transmit) or to remix (to adapt) it, if you respect
% how you must attribute the work in the manner specified by the author(s):
% \newline
% In an internet based reuse please link the reused parts to mycsrf.fodina.de
% and mention the original author Karsten Reincke in a suitable manner. In a
% paper-like reuse please insert a short hint to mycsrf.fodina.de and to the
% original author, Karsten Reincke, into your preface. For normal quotations
% please use the scientific standard to cite
%


%% use all entries of the bibliography


\subsection{Das harmony-Paket: ein echtes Highlight}

Das Paket \textit{harmony}\footcite[vgl.][\nopage wp]{CtanHarmony2018a} bietet für
den Gebrauch innerhalb einer Textzeile ein besonders ausgefeiltes System von
Musikzeichen an. Als \LaTeX-Paket muss es natürlich ebenfalls zuerst mittels
eines Befehls in die Präambel eingebunden werden (\small 
\texttt{\textbackslash{usepackage\{harmony\}}}), bevor es Zeichen für die beiden
Bereiche \textit{Rhythmik} und \textit{Harmonieanalyse} bereitstellt\footcite[Für
einen vollen Überblick über den Zeichenvorrat und die Kombinationsmöglichkeiten
vgl.][4ff]{WegWeg2007a}:

\subsubsection{\small Rhythmik}

Zunächst enthält es gut funktionierende Encodierungen für Taktarten \{ 
\Takt{3}{4} \ (= \texttt{\small \textbackslash{Takt}\{3\}\{4\}}),
\Takt{4}{4} \ (= \texttt{\small \textbackslash{Takt}\{4\}\{4\}}),
\ldots,
\Takt{c}{0} \ (= \texttt{\small \textbackslash{Takt}\{c\}\{0\}}),
\Takt{c}{1} \ (= \texttt{\small \textbackslash{Takt}\{c\}\{1\}})
\}.
Dann offeriert es nicht nur einfache Notenlängen \{
\Ganz \ (= \texttt{\small \textbackslash{Ganz}}),
\Halb \ (= \texttt{\small \textbackslash{Halb}}),
\Vier \ (= \texttt{\small \textbackslash{Vier}}),
\Acht \ (= \texttt{\small \textbackslash{Acht}}),
\Sech \ (= \texttt{\small \textbackslash{Sech}}),
\Zwdr \ (= \texttt{\small \textbackslash{Zwdr}}),
\}  -- \ die sogar punktiert werden können 
\{
\Halb\Pu \ (= \texttt{\small \textbackslash{Halb}\textbackslash{Pu}}),
\Vier\Pu \ (= \texttt{\small \textbackslash{Vier}\textbackslash{Pu}}),
\ldots
\}
-- ,
sondern es stellt die Symbole auch mit waagerechten Balken zu Verfügung
\{
\AchtBL \ (= \texttt{\small \textbackslash{AchtBL}}),
\SechBL \ (= \texttt{\small \textbackslash{SechBL}}),
\Vier\AchtBL \ (= \texttt{\small \textbackslash{Vier} \textbackslash{AchtBL} }),
\Vier\SechBL \ (= \texttt{\small \textbackslash{Vier} \textbackslash{SechBL} }),
\ldots
\}, 
sodass die einzelnen Elemente zu ganzen Rhythmusketten kombiniert werden können: 
\Takt{c}{0} \Vier \ \Vier\AchtBL \ \Vier\Pu \ \Acht \ $|$ 
\AchtBR\Pu \SechBl \ \AchtBR\kern-0.15em\SechBR\Vier \ \SechBr\Vier\SechBl \ $|$
\ -- eine wirklich ausgefuchste Lösung.

\subsubsection{\small Harmonik}

Einen ähnlich geschickten Ansatz bietet das Paket \textit{harmony} da, wo es
Symbole erzeugt, die -- der Funktions- und Stufentheorie entsprechend --
harmonische Zusammenhänge repräsentieren.

Den Kern bildet die allgemeine Tupelkonstruktion \texttt{\small
\textbackslash{HH.X.u.v.w.z.}}, mit der einfache und komplexe Aspekte
dargestellt werden können\footcite[vgl. dazu][2ff]{WegWeg2007a}: Was zwischen
den ersten beiden Punkten erscheint (X), erscheint als Hauptzeichen; dann folgt
das, was darunter erscheinen soll (u) und schließlich das, was rechts oben
daneben angezeigt werden soll, und zwar von oben nach unten (v w z):

\begin{center}
\HH.X.u.v.w.z.
\end{center}

Der Witz ist nun, dass man nahezu beliebige Latexkonstrukte zwischen den Punkten
einsetzen kann. Und wenn man nichts zwischen den Punkten einträgt, bleibt der
Platz leer. So können die einfachen Symbole der Funktionstheorie \{ \HH.T..... 
\ (= \texttt{\small \textbackslash{HH.T.....}}), \HH.Tp.....  \ (=
\texttt{\small \textbackslash{HH.Tp.....}}), \HH.S.....  \ (= \texttt{\small
\textbackslash{HH.S.....}}), \HH.D.....  \ (= \texttt{\small
\textbackslash{HH.D.....}}) \} ebenso erzeugt werden, wie komplexere \{
\HH.D.3.9.7..  \ (= \texttt{\small \textbackslash{HH.D.3.9.7..}}),
\HH.T..9$\flat\rightarrow$8.7..  (= \verb|\HH.T..9$\flat\rightarrow$8.7..|).
Die speziellen Zeichen der Funktionstheorie, die nicht so einfach aus dem Fundus
normaler Fonts gebildet werden können, stellt  \textit{harmony} gesondert zur
Verfügung:
\{ \Dohne  \ (= \texttt{\small \textbackslash{Dohne}}), \DD \ (= \texttt{\small
\textbackslash{DD}}), \DS  \ (= \texttt{\small \textbackslash{DS}}) \}.
Selbstverständlich können diese ebenfalls in das allgemeine Tupelkonstrukt
eingebettet werden\footcite[Vgl. dazu][6]{WegWeg2007a}:
\begin{center}
 \texttt{\textbackslash{HH}.\textbackslash{DD}.5\textbackslash{VM}.7...} 
 $\rightarrow$ \HH.\DD.5\VM.7...
\end{center}

Zudem erlaubt es die Flexibilität des Grundkonstruktes (in gewissen Grenzen),
Symbole für die Stufentheorie und den Generalbass zu erzeugen:

\begin{center}
\HH.I..\texttt{(5)}.\texttt{(3)}.. \ 
\HH.III..\texttt{ 6 }.\texttt{(3)}.. \ 
\HH.V..\texttt{ 6}.\texttt{ 4}.. \ 
\HH.I..\texttt{(5)}.\texttt{(3)}.. \ 
\HH.I..\texttt{(5)}.\texttt{ 3$\flat$ }.. \ 
\HH.III..\texttt{ 6 }.\texttt{ 3$\flat$}.. \ 
\HH.III..\texttt{ 6 }.\texttt{ 5 }.\texttt{(3)}. \ 
\HH.III..\texttt{ 6$\flat$}.\texttt{ 3$\flat$}.. \ 
\end{center}

Man sieht an diesem Beispiel auch, dass sich die \textit{harmony}-Konstrukte (hier
\texttt{\textbackslash{texttt}} und \texttt{\$\textbackslash{flat}\$} ) gut mit
anderen \LaTeX-Elementen kombinieren lassen:
\begin{verbatim}
\begin{center}
\HH.I..\texttt{(5)}.\texttt{(3)}.. \ 
\HH.III..\texttt{ 6 }.\texttt{(3)}.. \ 
\HH.V..\texttt{ 6}.\texttt{ 4}.. \ 
\HH.I..\texttt{(5)}.\texttt{(3)}.. \ 
\HH.I..\texttt{(5)}.\texttt{ 3$\flat$ }.. \ 
\HH.III..\texttt{ 6 }.\texttt{ 3$\flat$}.. \ 
\HH.III..\texttt{ 6 }.\texttt{ 5 }.\texttt{(3)}. \ 
\HH.III..\texttt{ 6$\flat$}.\texttt{ 3$\flat$}.. \ 
\end{center}
\end{verbatim}

Später werden wir zeigen, dass sich die \textit{harmony}-Elemente ihrerseits auch
gut in MusiX\TeX-Syntagmen einbetten lassen. Insofern haben die Programmierer
von \textit{harmony} der Community ein mächtiges Werzeug zur Verfügung
gestellt\footnote{Trotzdem wollen auch wir wenigstens darauf hinweisen, dass
\textit{harmony}-Konstrukten auf die Einbettung in einen Fließtext mit 12 Pt.
ausgelegt sind. Bei kleineren Größen von 11PT abwärts werden die Zeilenabständen
-- wie in diesem Beispieltext erkennbar -- gedehnt, es entsteht ein leicht
unruhigeres Druckbild. (\cite[Vgl. dazu][2]{WegWeg2007a}.) Allerdings ist dieser
Hinweis nicht mehr als ein Jammern auf sehr hohem Niveau.}.


% this is only inserted to eject fault messages in texlipse
%\bibliography{../bib/literature}
