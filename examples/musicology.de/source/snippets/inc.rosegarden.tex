% mycsrf 'for beeing included' snippet template
%
% (c) Karsten Reincke, Frankfurt a.M. 2012, ff.
%
% This text is licensed under the Creative Commons Attribution 3.0 Germany
% License (http://creativecommons.org/licenses/by/3.0/de/): Feel free to share
% (to copy, distribute and transmit) or to remix (to adapt) it, if you respect
% how you must attribute the work in the manner specified by the author(s):
% \newline
% In an internet based reuse please link the reused parts to mycsrf.fodina.de
% and mention the original author Karsten Reincke in a suitable manner. In a
% paper-like reuse please insert a short hint to mycsrf.fodina.de and to the
% original author, Karsten Reincke, into your preface. For normal quotations
% please use the scientific standard to cite
%


%% use all entries of the bibliography

\subsubsection{Rosegarden ($\bigstar\bigstar\bigstar\bigstar$)}

\label{Rosegarden}\acc{Rosengarden} versteht sich als Kompositionsumgebung, die
um einen 'MIDI Sequencer' herum aufgebaut worden sei und dabei auch als
Notensatz- und digitales Audiosystem fungiere.\footcite[vgl.][\nopage
wp]{Rosegarden2019a} Als \enquote{MIDI and audio sequencer} und \enquote{musical
notation editor} wolle es das Tool der Wahl für jene sein, die es vorziehen,
mittels Noten zu arbeiten:\footcite[vgl.][\nopage wp]{Rosegarden2019c}

\begin{quote}\textit{Rosegarden allows you to record, arrange, and compose
music, in the shape of traditional score or MIDI data, or of audio files either
imported or recorded from a microphone, guitar or whatever audio source you care
to specify.}\footcite[vgl.][\nopage wp]{Rosegarden2019c} \end{quote}

Man könne mit \acc{Rosegarden} Musik schreiben, editieren oder komponieren,
diese synthetisieren, mit Effekten anreichern oder abmischen, um sie schließlich
auf CD zu brennen. Und nicht zuletzt biete \acc{Rosegarden} eben den
\enquote{[\ldots] well-rounded notation editing support for high quality printed
output via LilyPond.} \footcite[vgl.][\nopage wp]{Rosegarden2019c}

Bedenkt man nun noch, dass \enquote{Kern eines Sequenzers [\ldots] die
Speicherung und Über\-mitt\-lung einer Partitur an einen Tonerzeuger (sei)},
dass diese \enquote{ [\ldots] in einem maschinenlesbaren Format (vorliege) und
[\ldots] Tonhöhe, Tondauer und ggf. weitere Aspekte der wiederzugebenden Noten
einer oder mehrerer Stimmen in ihrer zeitlichen Reihenfolge an ein Gerät
(weitergäbe), das entsprechende Töne (erzeuge)}\footcite[vgl.][\nopage
wp]{WpedSequencer2018a}, dann hat man eine recht genaue Vorstellung von dem, was
\acc{Rosegarden} leisten will: Es erlaubt den Import von \acc{MIDI}-Aufnahmen,
bietet die Möglichkeit, diese zu korrigieren, zu modifizieren, zu arrangieren
und wieder abzuspielen\footcite[vgl.][\nopage wp]{Rosegarden2019c}; und die
Modifikationen können auch visuell über einen Noteneditor
erfolgen.\footcite[vgl.][\nopage wp]{Rosegarden2019d}

Als Open-Source-Software wird der Quelltext von \acc{Rosegarden} öffentlicht
gehostet und weiterentwickelt.\footcite[vgl.][\nopage wp]{Rosegarden2019e} Die
Dokumentation wird als Wiki gepflegt\footcite[vgl.][\nopage wp]{Rosegarden2019c}
und auch in thematischen Teilenbereichen bereitgestellt\footcite[vgl.][\nopage
wp]{Rosegarden2019d}. Daneben gibt es noch spezielle
Tutorials\footcite[vgl.][\nopage wp]{Rosegarden2019b}, die ausgehend von einem
bestimmten Aspekt in die Nutzung von \acc{Rosegarden}
einführen.\footcite[vgl.][\nopage wp]{McIntyre2008a}

Vom Format her erlaubt es \acc{Rosegarden}, außer dem eigenen Dateityp auch
\acc{MIDI}- und \acc{MusicXML}-Dateien zu öffnen bzw. zu importieren. Bei Export
steht dann u.a. noch das \acc{LilyPond}-Format zur Auswahl. Startet man einen
\acc{MIDI}-Server -- etwa \acc{timidity}\footnote{über eine Shell per
\texttt{timidity -iA}} --, bevor man \acc{Rosegarden} aufruft, können die
geladenen Noten außerdem erfolgreich abgespielt werden.
% this is only inserted to eject fault messages in texlipse
%\bibliography{../bib/literature}
