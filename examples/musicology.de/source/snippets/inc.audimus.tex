% mycsrf 'for beeing included' snippet template
%
% (c) Karsten Reincke, Frankfurt a.M. 2012, ff.
%
% This text is licensed under the Creative Commons Attribution 3.0 Germany
% License (http://creativecommons.org/licenses/by/3.0/de/): Feel free to share
% (to copy, distribute and transmit) or to remix (to adapt) it, if you respect
% how you must attribute the work in the manner specified by the author(s):
% \newline
% In an internet based reuse please link the reused parts to mycsrf.fodina.de
% and mention the original author Karsten Reincke in a suitable manner. In a
% paper-like reuse please insert a short hint to mycsrf.fodina.de and to the
% original author, Karsten Reincke, into your preface. For normal quotations
% please use the scientific standard to cite
%


%% use all entries of the bibliography

\subsubsection{Audimus ($\bigstar$)}

\label{Audimus}\acc{Audimus} ist Teil eines Projektes \enquote{[\ldots] to
develop educational software for music teachers and their
students}.\footnote{\cite[vgl.][\nopage wp]{Audimus2008a}. Die SourceForge-Seite
nennt als Lizenz die GPL-2.0, womit das Programm unter einer anerkannten
Open-Source-Lizenz distribuiert wird ($\rightarrow$
\href{https://opensource.org/licenses/alphabetical}
{https://opensource.org/licenses/alphabetical}).} Die Notationssoftware
firmiert unter dem Namen \acc{audimus-notes}. Es handelt sich um ein
Javaprogramm, das -- nach Installation einer geeignete Java-Runtime-Umgebung --
einfach mit dem Kommando \texttt{java -jar Audimus.jar} gestartet werden kann,
ein entsprechendes Shellskript liegt bei. Die letzte ausgereifte Version stammt
vom September 2007, ein letztes Folgerelease aus dem August
2008.\footcite[vgl.][\nopage wp]{Audimus2008b}

Das Alter macht sich bemerkbar. Die Software kann gestartet werden, das
graphische Frontend funktioniert aber nicht (mehr): es scheinen Fonts und
Zeichenelemente zu fehlen. Damit ist dieses Notensatzprogramm, das MusicXML
lesen und schreiben soll, für Musikwissenschaftler heute nicht mehr verwendbar.
Sein Output hätte eh nur über Konverter mit \LaTeX\ verbunden werden können. Aus
Gründen des Respekts möge es jedoch noch einen Stern erhalten.



% this is only inserted to eject fault messages in texlipse
%\bibliography{../bib/literature}
