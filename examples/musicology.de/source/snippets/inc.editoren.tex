% mycsrf 'for beeing included' snippet template
%
% (c) Karsten Reincke, Frankfurt a.M. 2012, ff.
%
% This text is licensed under the Creative Commons Attribution 3.0 Germany
% License (http://creativecommons.org/licenses/by/3.0/de/): Feel free to share
% (to copy, distribute and transmit) or to remix (to adapt) it, if you respect
% how you must attribute the work in the manner specified by the author(s):
% \newline
% In an internet based reuse please link the reused parts to mycsrf.fodina.de
% and mention the original author Karsten Reincke in a suitable manner. In a
% paper-like reuse please insert a short hint to mycsrf.fodina.de and to the
% original author, Karsten Reincke, into your preface. For normal quotations
% please use the scientific standard to cite
%


%% use all entries of the bibliography

\subsection{Editoren I}

Die bereits erwähnte umfangreiche Sichtung von Notensatzprogrammen klassifiziert
die Kandidaten nach den Merkmalen \enquote{kostenpflichtig} versus \enquote{Open
Source} und \enquote{WYSIWYG-Benutzeroberfläche} versus
\enquote{Markup-Notensatz} versus \enquote{Sequenzer mit
Notensatzfunktion}.\footcite[vgl.][\nopage wp]{WpedNotensatz2019a} Wir
bezeichnen sie als graphisches resp. textbasierte 'Frontends' für die
vorgestellten Backendsysteme. Das ist berechtigt, weil die gelisteten
Notensatzprogramme in der Regel Eingaben annehmen und das Ergebnis an die bisher
erwähnten Techniken der Backendsysteme 'durchreichen': Das graphische oder
textbadierte Editieren einer ABC-Notendatei ist das eine, die Umwandlung der
Datei in eine PDF- oder PS-Datei mit 'Noten' das andere. In dieser Liste von
Notensatzprogrammen finden sich dann die freien Frontends \textit{Aria
Maestosa}, \textit{Brahms}, \textit{Canorus}, \textit{Denemo},
\textit{Laborejo}, \textit{Mup}, \textit{MuseScore}, \textit{NoteEdit},
\textit{NtEd}, \textit{Rosegarden}.\footcite[vgl.][\nopage
wp]{WpedNotensatz2019a} Außerdem trifft man im Netz noch auf die Tools
\textit{Frescobaldi}, \textit{muX2d} und \textit{Elysium/Eclipse}. Für die
\acc{ABC-Notationsmethode} stehen die Opensource-Frontends \acc{EasyABC} und
\acc{ABCJ} zur Verfügung.\footcite[vgl.][\nopage wp]{Abc2018b} Die wohl
umfangreichste Sammlung von Musiksoftware liefert die Site
\acc{MusicXML}\footcite[vgl.][\nopage wp]{MusicXML2018b}, einfach weil sie
schlicht auflistet, welche Software MusicXML-Dateien liest, schreibt oder liest
und schreibt. Diese Liste enthält mithin auch andere Programme als solche für
den Notensatz. Und sie listet proprietäre und freie Applikationen. Von den
freien Notensatzprogrammen dieser Liste evaluieren wir noch \textit{Audimus
Notes}, \textit{Free Clef}, \textit{MusEdit}, \textit{Jniz} und
\textit{Ptolemaic}\footnote{Die Evaluation von Programmen, die dezidiert für
Android oder iOS oder Windows gedacht sind oder die spezielle Zwecke verfolgen
-- also etwa \textit{Audovia}, \acc{Candezii}, \acc{Chaconne}, \acc{Crescendo},
\textit{Jfugue}, \acc{flabc}, \acc{Finale Notepad}, \acc{Mc MusicEditor},
\acc{Notation Pad}, \acc{Ossia Viewer}, \acc{Score Creator} -- überlassen wir
mit einer gewissen Willkür späteren Darstellungen.}.


\textit{Aria Maestosa}, \textit{Brahms} und \textit{Rosegarden}
sind in erster Linie \textsf{Sequenzer}, \textit{Frescobaldi} und
\textit{Elysium/Eclipse} bieten 'nur' eine textuelle Eingabeschnittstelle,
machen das Ergebnis der Eingabe aber direkt als Bild sichtbar. Die anderen
dürfen als visuelle Notensatzprogramme angesehen werden.

Systematisch kann man die Lage so darstellen:

 
\begin{center}\scriptsize
\begin{tabulary}{15cm}{|R|C||C|C|C|C|C|C||C|C||C|C|C|C|C|C|C|}
\hline
Frontend & &
  \multicolumn{6}{c||}{Import} & 
  \multicolumn{2}{c||}{Change} & 
  \multicolumn{7}{c|}{Export} \\
\hline
Programm & 
  \rotatebox{90}{Seite} & 
  \rotatebox{90}{ABC} & 
  \rotatebox{90}{LilyPond} & 
  \rotatebox{90}{Midi} & 
  \rotatebox{90}{MusicXML} & 
  \rotatebox{90}{MusiX\TeX} & 
  \rotatebox{90}{PMX} &
  \rotatebox{90}{graphisch} & \rotatebox{90}{textuell} &
  \rotatebox{90}{ABC} & 
  \rotatebox{90}{LilyPond} & 
  \rotatebox{90}{Midi} & 
  \rotatebox{90}{MusicXML} & 
  \rotatebox{90}{MusiX\TeX\ / PMX} & 
  \rotatebox{90}{PDF / PS / PNG / JPEG / SVG\ } &  
  \rotatebox{90}{SOUND}
\\
\hline
\hline
ABCJ & \pageref{ABCJ} &
  \checkmark & & \checkmark & & & & & \checkmark &  
  \checkmark & & \checkmark & & & & \\
\hline
Aria Mae. & \pageref{AriaMaestosa} &
  & & \checkmark & & & & \checkmark & & 
  & & \checkmark & & & & \\
\hline
Audimus N. & \pageref{Audimus} &
  & & \checkmark & \checkmark & & & \checkmark & & 
  & & \checkmark & \checkmark & & &  \\
\hline
Brahms & \pageref{Brahms} &
   & & & & & & ? & ? &  & & & & & & \\
\hline
Canorus & \pageref{Canorus} &
  &  & \checkmark & \checkmark & & & 
 \checkmark & & 
  & \checkmark & \checkmark & \checkmark & & \checkmark &  \\
\hline
Denemo & &
  1 & 2 & 3 & 4 & 5 & 6 &
  7 & 8 & 
  9 & A & B & C & D  & F & H \\
\hline
EasyABC & \pageref{EasyABC} &
   \checkmark  &  & \checkmark & \checkmark &  &  & & \checkmark  & 
  \checkmark  &  & \checkmark  & \checkmark  &   & \checkmark &   \\
\hline
Elysium & \pageref{Elysium} &
  & \checkmark & & & & & & \checkmark & 
   & \checkmark & & & & \checkmark & \checkmark  \\
\hline
Free Clef & \pageref{FreeClef} &
    &  & & \checkmark & & & ? & ? & 
   &  &  & \checkmark & & & \\
\hline
Frescobaldi & \pageref{Frescobaldi} &
  \checkmark & \checkmark & \checkmark & \checkmark &  &  & & \checkmark & 
  & \checkmark & \checkmark & & & \checkmark & \checkmark \\
\hline
Jniz & \pageref{Jniz} &
   &  &  &  &  &  &
  \checkmark &  & 
  & \checkmark & \checkmark & \checkmark & & \checkmark & \\
\hline
Laborejo & \pageref{Laborejo} & 
  & & & & &
  \checkmark & \checkmark & 
   & \checkmark & \checkmark &  &  &  &  \checkmark & H \\
\hline
MusEdit & \pageref{MusEdit} &
   &  &  & \checkmark &   &   & ? &  ? & 
    &   &   & \checkmark  &   &   &   \\
\hline
MuseScore & &
  1 & 2 & 3 & 4 & 5 & 6 &
  7 & 8 & 
  9 & A & B & C & D  & F & H \\
\hline
muX2D & \pageref{MuX2d} &
  & & & \checkmark & & & ? & ? & 
  & & & \checkmark & & & \\
\hline
NoteEdit & \pageref{NoteEdit} & & & & & & & ? & ? & 
& & & & & & \\
\hline 
NtEd & \pageref{NtEd} &  & & \checkmark & \checkmark &  &  &
 \checkmark &  &
   & \checkmark & \checkmark &  &  & \checkmark &  \\
\hline
Ptolemaic & \pageref{Ptolemaic} &
  & & & \checkmark & & & \checkmark & & & & & & & & \\
\hline
\end{tabulary}
\end{center}
 
Eine solche Fülle von Möglichkeiten fordert geradezu dazu auf, praktisch zu
überprüfen, welche den Anforderungen von Musikwissenschaftlern (am besten)
gerecht werden. Deshalb werden wir testen, wie diese Tools mit der Referenzkadenz II
umgehen können, in wie weit sie die Symbole der Harmonieanalyse aufzunehmen
vermögen und ob sie diesen Inhalt in einem Format abspeichern
können\footnote{Konkret werden wir unsere Referenzkadenz-II jeweils mit den
\acc{LilyPond}-kompatiblen Frontends erfassen, das Result von dort aus als
\acc{LilyPond}-Datei exportieren und diese wiederum vom Eclipse-Plugin
\acc{Elysium} einlesen lassen. \acc{Elysium} leitet aus dieser ursprünglich
exportierten Datei automatisch eine PDF-Datei ab, die genauso aussehen sollte,
wie das Original. Stimmt also diese Datei mit den Erwartungen überein,
betrachten wir den ursprünglichen Export als verifziert.\label{ExportVerifikation}
Die anderen Exportformate werden wir entsprechend überprüfen. Einzelheiten dazu
vermerken wir bei den entsprechenden Fällen.}.


% this is only inserted to eject fault messages in texlipse
%\bibliography{../bib/literature}
