% mycsrf 'for beeing included' snippet template
%
% (c) Karsten Reincke, Frankfurt a.M. 2012, ff.
%
% This text is licensed under the Creative Commons Attribution 3.0 Germany
% License (http://creativecommons.org/licenses/by/3.0/de/): Feel free to share
% (to copy, distribute and transmit) or to remix (to adapt) it, if you respect
% how you must attribute the work in the manner specified by the author(s):
% \newline
% In an internet based reuse please link the reused parts to mycsrf.fodina.de
% and mention the original author Karsten Reincke in a suitable manner. In a
% paper-like reuse please insert a short hint to mycsrf.fodina.de and to the
% original author, Karsten Reincke, into your preface. For normal quotations
% please use the scientific standard to cite
%


%% use all entries of the bibliography

\subsubsection{Aria Meastosa ($\bigstar$$\bigstar$)}

\label{AriaMaestosa}\acc{Aria Maestosa} ist Open-Source-Software und bezeichnet
sich selbst als  \enquote{midi sequencer/editor}, der \acc{MIDI}-Datei erzeugen,
editieren und abspielen könne\footcite[vgl.][\nopage wp]{AriaMaestosa2017a}. Er
bringt eine Anleitung zur Nutzung\footcite[vgl.][\nopage wp]{AriaMaestosa2017b}
und eine zur Installation\footcite[vgl.][\nopage wp]{AriaMaestosa2017c} mit. Die
letzte Version stammt aus dem Jahr 2017\footnote{$\rightarrow$
\href{https://sourceforge.net/projects/ariamaestosa/files/}
{https://sourceforge.net/projects/ariamaestosa/files/}. \acc{Aria Maestosa} wird
mittels des Kompatibilitätslayers \acc{wxwidgets} kompiliert. Für Ubuntu gibt es
nur veraltete Pakete. Der Versuch, die Version von 2017 aus den Quellen zu
installieren -- was ansich sehr gut beschrieben ist -- scheitert auf einem
mo\-der\-ne\-ren GNU/Linux daran, dass das Kompilat eine veraltete Bibliothek
erwartet, die zu aktivieren nicht mehr ohne radikalen Systemumbau möglich ist.}.

Das Besondere an \acc{Aria Maestosa} ist sicherlich, dass es in erster Linie
nicht das Schreiben von Noten über einen \enquote{Score Editor} graphisch
simuliert, sondern das Spielen von Instrumenten: ein \enquote{Piano Editor}
simuliert die Eingabe der Töne über die Klaviatur, ein \enquote{Guitar
Editor} über die Bünde einer Guitarre\footcite[vgl.][\nopage
wp]{AriaMaestosa2017b} etc. Das Programm liest und schreibt im Wesentlichen
\acc{MIDI}-Dateien, als Exportformat soll \acc{aiff}\footnote{Audio Interchange
File Format} zu Verfügung stehen\footcite[vgl.][\nopage wp]{Guepewi2017a}. Neben
dem Editieren und Abspielen bietet \acc{Aria Maestosa} auch eine
Druckfunktion an, womit es in den Bereich der Notensatzprogrammme hineingreift,
ohne vom Anspruch her wirklich eines zu sein.

Für den Musikwissenschaftler dürfte dieses Programm seiner spezifischen
Ausrichtung und seiner begrenzten Im- und Exportfunktionen wegen als generelles
Tool kaum in Frage kommen. Das kann man nicht dem Programm vorhalten: es
tut das, was es tun will -- und das offensichtlich gut. In unserem Kontext ist
es uns deshalb noch zwei Sterne wert. In einem anderen könnten es mehr sein.

% this is only inserted to eject fault messages in texlipse
% \bibliography{../bib/literature}
