% mycsrf 'for beeing included' snippet template
%
% (c) Karsten Reincke, Frankfurt a.M. 2012, ff.
%
% This text is licensed under the Creative Commons Attribution 3.0 Germany
% License (http://creativecommons.org/licenses/by/3.0/de/): Feel free to share
% (to copy, distribute and transmit) or to remix (to adapt) it, if you respect
% how you must attribute the work in the manner specified by the author(s):
% \newline
% In an internet based reuse please link the reused parts to mycsrf.fodina.de
% and mention the original author Karsten Reincke in a suitable manner. In a
% paper-like reuse please insert a short hint to mycsrf.fodina.de and to the
% original author, Karsten Reincke, into your preface. For normal quotations
% please use the scientific standard to cite
%

\begin{abstract}
\noindent \itshape
Der folgende Text stellt vor, wie man \LaTeX-Texte mit Open-Source-Mitteln um
musikalische Beispiele anreichert. Dazu sichtet er Notensatzprogramme, Editoren,
Konverter und Tools, die Notentexte erzeugen, verändern und in den \LaTeX-Text
einbetten. Und er skizziert, wie man ganze 'Produktionsketten' aus
Frontendsystemen, Konvertern und Backendsystemen 'zusammenstöpselt'.
Zuletzt entsteht so eine 'Landkarte' verschiedener Wege.
\end{abstract}




% this is only inserted to eject fault messages in texlipse
%\bibliography{../bib/literature}
