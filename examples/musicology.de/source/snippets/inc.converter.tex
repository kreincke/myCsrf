% mycsrf 'for beeing included' snippet template
%
% (c) Karsten Reincke, Frankfurt a.M. 2012, ff.
%
% This text is licensed under the Creative Commons Attribution 3.0 Germany
% License (http://creativecommons.org/licenses/by/3.0/de/): Feel free to share
% (to copy, distribute and transmit) or to remix (to adapt) it, if you respect
% how you must attribute the work in the manner specified by the author(s):
% \newline
% In an internet based reuse please link the reused parts to mycsrf.fodina.de
% and mention the original author Karsten Reincke in a suitable manner. In a
% paper-like reuse please insert a short hint to mycsrf.fodina.de and to the
% original author, Karsten Reincke, into your preface. For normal quotations
% please use the scientific standard to cite
%


%% use all entries of the bibliography

\subsection{Konverter}

Wer nach Konvertierungssoftware sucht, findet drei wichtige Anlaufstellen:

Für die \textit{ABC}-Notation gibt es eine anregende und umfängliche
\enquote{Softwareseite}\footcite[vgl.][\nopage wp]{Abc2018b}, die u.A.
auflistet, mit welcher Software welche Formate aus \textit{ABC}-Code erzeugt und
aus welch anderen Formaten \textit{ABC}-Code generiert werden kann. Für
\textit{MusiX\TeX\ } und \textit{PMX} bietet das Icking-Music-Archive eine
ähnliche, wenn auch nicht ganz so extensive Webseite.\footcite[vgl.][\nopage
wp]{Tennent2018b} Und für \textit{Lilypond} listet dessen Nutzungshandbuch eine Reihe von
Konvertierungsoptionen auf\footcite[vgl.][42ff]{LilyPond2018e}.

Diese Anlaufstellen verweisen auch auf \textit{MusicXML}\footcite[vgl.][\nopage
wp]{WpedMusicXML2018a}, das sich selbst als \enquote{das Standardformat für den
Austausch von digitalen Musiknoten}\footcite[vgl.][\nopage wp]{MusicXML2018a}
bezeichnet\footnote{Trotzdem gibt es Alternativen, auch wenn sie in unserem
Kontext (noch) keine Rolle spielen. So zielt etwa \textit{MusixJSON} -- als
Format publiziert in einer github-Markdown-Datei -- 
auf einen vereinfachten Transport übers Netz.
($\rightarrow$
\href{https://github.com/soundio/music-json}
{\texttt{https://github.com/soundio/music-json}}
RDL: 2019-01-17). Und es existieren bereits Konverter, wie etwa
\textit{musicJSON2abc} ($\rightarrow$
\href{https://github.com/freakimkaefig/musicjson2abc}
{\texttt{https://github.com/freakimkaefig/musicjson2abc}}
RDL: 2019-01-17). Die Grund\-idee zur Einführung dieses Formates dürfte folgende
gewesen sein: XML ist syntaktisch sehr extensiv. In Zeiten kooperierenden
Computer bedarf es aber auch eines schlanken Formates. Als eine solche
Enkodierung hat sich längst die JSON-Notation etabliert. Mit ihr lassen sich --
im Rückgriff auf das http-Protokoll -- Daten im standardisierten Verfahren
durchs Web übertragen. Um Musik in Notenform zu übermitteln, bedürfte es also
nur eines 'JSON-Dialektes', damit man die bereits existierenden
Programmiertechniken wiederverwenden könnte.}. In diesem Sinne listet die
\textit{MusicXML}-Homepage denn auch selbst eine Fülle von Software auf, die
dieses Format zu verarbeiten vermag\footcite[vgl.][\nopage wp]{MusicXML2018b}.

Führt man diese Quellen zusammen, entsteht in etwa folgendes Bild von
Konvertierungsvarianten:

\begin{center}
\renewcommand{\arraystretch}{1.5}
\begin{tabulary}{14cm}{|C|R||C|C|C|C|}
\hline
  & nach: & ABC & MusiX\TeX & PMX & LilyPond \\
\hline
\hline
\multirow{5}{*}{\rotatebox{90}{von:}} 
  & ABC & $\times$ & abc2mtex & $\varnothing$ & abc2ly \\
\cline{2-6}
  & MusiX\TeX & $\varnothing$ & $\times$ & ($\varnothing$) &  $\varnothing$ \\
\cline{2-6}
  & PMX & $\varnothing$  & (pmxab) & $\times$ & \sout{pmx2ly} \\
\cline{2-6}
  & LilyPond & \dashuline{ly2abc} & $\varnothing$ & $\varnothing$ & $\times$  \\
\cline{2-6}
  & MusicXML &   \dashuline{xml2abc},   \dashuline{mxml2abc} & $\varnothing$ & xml2pmx & xml2ly musicxml2ly \\
\hline 
\hline
\end{tabulary}
\renewcommand{\arraystretch}{1}
\end{center}

Diese Tabelle ist wie folgt zu lesen:

\begin{itemize}
  \item Dass es keine Konverter gibt, die ihren Input auf dasselbe Inputformat
  abbilden -- also etwa von \textit{ABC} nach \textit{ABC} oder von
  \textit{MusiX\TeX} nach \textit{MusiX\-\TeX} --, liegt in der Natur der Sache.
  [$\rightarrow \times$] \item Dass es keinen Konverter von \textit{MusiX\TeX}
  nach \textit{PMX} zu geben scheint, überrascht nicht wirklich. Schließlich ist
  \textit{PMX} von vornherein als Präprozessor für \textit{MusiX\TeX} gedacht.
  Und dass umgekehrt als Konverter von \textit{PMX} zu \textit{MusiX\TeX} das
  bereits diskutierte PMX-Standardtool \textit{pmxab} erwähnt wird, liegt in der
  Natur des Konzeptes. [$\rightarrow$ (eingeklammert)]
  \item Der Konverter \textit{pmy2ly} wird heute nicht mehr mit Lilypond
  ausgeliefert; es gibt Stimmen im Netz, die sagen, dass er nicht mehr weiter
  gepflegt werde und deshalb aus dem Bestand genommen worden sei\footnote{$\rightarrow$
  \href{https://lists.gnu.org/archive/html/lilypond-user/2008-08/msg00056.html}{
  \texttt{https://lists.gnu.org/archive/html/lilypond-user/2008-08/msg00056.html}}}
  RDL. 2019-01-17). [$\rightarrow$ \sout{durchstrichen}]
  \item Für einige Kombinationen haben wir (noch) keine Konverter gefunden,
  obwohl es prinzipiell sinnvoll wäre, wenn es solche gäbe. [$\rightarrow$
  $\varnothing$] 
  \item Die Konverter von \acc{LilyPond} nach \acc{ABC} resp. von \acc{MusicXML}
  nach \acc{ABC} scheinen auf den ersten Blick nachrangig zu sein: Denn wenn wir
  beispielsweise schon eine \textit{LilyPond}-Datei hätten, brächte eine
  Umwandlung ins \textit{ABC}-Format per \texttt{ly2abc} keinen Gewinn mehr,
  weil der Musikwissenschaftler im \textit{LilyPond}-Format mehr auszudrücken
  vermag, als im \textit{ABC}-Format. Indirekt spielen diese Konverter aber sehr
  wohl ein Rolle. Denn mit ihnen ließe sich eine \textit{LilyPond}- resp.
  \textit{MusicXML}-Datei ins \acc{ABC}-Format und von dort aus mittels des
  Konverters \acc{abc2mtex} ins Musix\TeX-Format transferieren.
  [$\rightarrow$ \dashuline{unterstrichelt}]
\end{itemize}

So ergeben sich -- ganz prinzipiell -- direkte und indirekte Wege,
Dateien in das Format eines unserer Backendsystem \acc{ABC}, \acc{LilyPond}, 
\acc{PMX} oder \acc{MusiX\TeX} zu überführen:

\begin{center}
\renewcommand{\arraystretch}{1.5}
\begin{tabulary}{14cm}{|C|C C C C C C C|C|}
\hline
 VON  & \ra & VIA &  & & & & \ra & NACH \\
\hline
\hline
 .ly  & \ra & ly2abc &  & & & & \ra & .abc \\
 .xml & \ra & xml2abc mxml2abc & & &  &  & \ra & .abc \\
\hline
 .abc & \ra & abc2ly & & &  &  & \ra & .ly \\
 .xml & \ra & xml2ly musicxml2ly & & &  &  & \ra & .ly \\
\hline
 .xml & \ra & xml2pmx & & &  &  & \ra & .pmx \\
\hline
 .abc & \ra & abc2mtex & & &  &  & \ra & .tex \\
 .ly  & \ra & ly2abc & \ra &  .abc & \ra & abc2mtex & \ra & .tex \\
 .pmx & \ra & pmxab & & &  &  & \ra & .tex \\
 .xml & \ra & { xml2abc, mxml2abc} & \ra & .abc & \ra & abc2mtex & \ra & .tex \\
\hline
\end{tabulary}
\renewcommand{\arraystretch}{1}
\end{center}

Und daraus folgen -- wiederum ganz prinzipiell -- einfache Seiteneffekte für die
Analyse freier graphischer Notensatzsysteme\footnote{Zur Erinnerung: Bei der
Auswahl der Programme folgen wir der Definition für freie Software. (\cite[Vgl.
dazu][\nopage wp]{FSF2018a})} für Musikwissenschaftler:

\begin{itemize}
  \item Gäbe es ein sehr gutes Frontend mit \textit{ABC}-Export, dann ließe sich dieses,
  sofern \texttt{abc2mtex} gut funktioniert, indirekt auch als Frontend für
  \textit{MusiX\TeX} nutzen.
  \item Gäbe es ein sehr gutes Frontend mit \textit{ABC}-Export, dann ließe sich dieses,
  sofern \texttt{abc2ly} gut funktioniert, indirekt auch als Frontend für
  \textit{LiliPond} nutzen.
  \item  Gäbe es ein sehr gutes Frontend mit \textit{MusicXML}-Export, dann ließe sich dieses,
  sofern \texttt{xml2pmx} gut funktioniert, indirekt auch als Frontend für
  \textit{MusiX\TeX} nutzen.
  \item Gäbe es ein sehr gutes Frontend mit \textit{MusicXML}-Export, dann ließe
  sich dieses, sofern \texttt{xml2ly} oder \texttt{musicxml2ly} gut
  funktionieren, indirekt auch als Frontend für \textit{MusiX\TeX} nutzen.
\end{itemize}

Ob diese Möglichkeiten auch praktisch existieren und ob sie Grenzen unterliegen,
werden wir später verifizieren.

 

% this is only inserted to eject fault messages in texlipse
%\bibliography{../bib/literature}
