% mycsrf 'for beeing included' snippet template
%
% (c) Karsten Reincke, Frankfurt a.M. 2012, ff.
%
% This text is licensed under the Creative Commons Attribution 3.0 Germany
% License (http://creativecommons.org/licenses/by/3.0/de/): Feel free to share
% (to copy, distribute and transmit) or to remix (to adapt) it, if you respect
% how you must attribute the work in the manner specified by the author(s):
% \newline
% In an internet based reuse please link the reused parts to mycsrf.fodina.de
% and mention the original author Karsten Reincke in a suitable manner. In a
% paper-like reuse please insert a short hint to mycsrf.fodina.de and to the
% original author, Karsten Reincke, into your preface. For normal quotations
% please use the scientific standard to cite
%


%% use all entries of the bibliography

\subsection{TeXmuse: die veraltete Interimslösung}

Wer nach Notationssystemen für Musik als Teil von \LaTeX-Texten sucht, wird
gelegentlich auf \textit{{\TeX}muse} stoßen. Wengistens die CTAN-Übersicht erwähnt
den Ansatz noch\footnote{$\rightarrow$ https://ctan.org/topic/music RDL:
2018-12-27}. Allerdings bekundet die Paketbeschreibung der letzten und einzigen
Veröffentlichung, dass es sich nur um ein \enquote{in­terim re­lease} handele
und dass das Programm \enquote{strictly lim­ited} ausgeliefert 
werde\footcite[vgl.][\nopage wp]{CtanTexmuse2005a}. Und auch die zugehörige Anleitung 
betont, das Programm sei noch \enquote{incomplete}: \enquote{[\ldots]
as it stands it can be called a ‘first stage'}\footcite[vgl.][1]{Garcia2005a}.

Obwohl das Paket graphisch ansprechende Beispiele enthält\footnote{$\rightarrow$
https://ctan.org/tex-archive/macros/texmuse/Samples/pdf/ RDL: 2018-12-27},
stehen Aufwand zur Integration und Nutzung auf der einen Seite und
Nachhaltigkeit auf der anderen nicht in einem angemessenen Verhältnis: Außer dem
erwähnten initialen Upload von 2005 hat es (bisher) kein weiteres Update oder
Upgrade gegeben. Das Paket in diesem Zustand zu verwenden, hieße (z.Zt.) auf ein
falsches Pferd zu setzen.


% this is only inserted to eject fault messages in texlipse
%\bibliography{../bib/literature}
