% mycsrf 'for beeing included' snippet template
%
% (c) Karsten Reincke, Frankfurt a.M. 2012, ff.
%
% This text is licensed under the Creative Commons Attribution 3.0 Germany
% License (http://creativecommons.org/licenses/by/3.0/de/): Feel free to share
% (to copy, distribute and transmit) or to remix (to adapt) it, if you respect
% how you must attribute the work in the manner specified by the author(s):
% \newline
% In an internet based reuse please link the reused parts to mycsrf.fodina.de
% and mention the original author Karsten Reincke in a suitable manner. In a
% paper-like reuse please insert a short hint to mycsrf.fodina.de and to the
% original author, Karsten Reincke, into your preface. For normal quotations
% please use the scientific standard to cite
%


%% use all entries of the bibliography

\subsection{MusiX\TeX: komplex, kompliziert und doch unabdingbar}

Das MusiX\TeX-Tutorial erwähnt mehrfach, dass man Notentexte mit dieser
Beschreibungssprache eher nicht 'händisch' erarbeiten wolle: Anfänger sollten -
wie es heißt - nicht mit mit MusiX\TeX\ beginnen, sondern lieber gleich
\textit{PMX} lernen\footcite[vgl.][iii]{VogSimRyc2018a}. Zwar könne man
MusiX\TeX-Befehle durchaus auch manuell in eine \LaTeX-Datei
einfügen. Gleichwohl würden es die meisten weniger anstrengend finden, die dabei
anstehenden Entscheidungen durch einen Präprozessor wie \textit{PMX} treffen zu
lassen\footcite[vgl.][1]{VogSimRyc2018a}. Wolle man jedoch Fließtext und Musik
in einem Dokument vereinen, dann sei die direkte Verwendung von MusiX\TeX\ 
durchaus eine Alternative\footcite[vgl.][1]{VogSimRyc2018a}.

Wir teilen diese Einschätzung. Mehr noch: wir meinen, dass es Rahmen einer
musikwissenschaftlichen Arbeit deutlich produktiver ist, seine MusiX\TeX\ 
encodierten Kadenzen, Motive oder Analysen etc. 'manuell' in den
\LaTeX-Fließtext einzufügen, anstatt Umwege über PMX zu gehen. Was PMX
betrifft, so werden wir den Schwierigkeiten später noch genauer nachgehen.

Für die direkte Verknüpfung von \LaTeX\ und MusiX\TeX\ gäbe es hingegen - wie es
heißt - zwei grundsätzlichen Methoden: entweder erzeuge man mit MusiX\TeX\ 
Bilder seiner Notenseiten\footnote{z.B. im Format \textit{EPS} = Encapsulated
Postscript} -- und zwar unabhängig von \LaTeX\ -- und binde diese Bilder
dann mit \LaTeX-Standardmitteln -- unabhängig von MusiX\TeX\ -- in
seinen Code ein. Oder man schreibe seinen MusiX\TeX-Code direkt als
Bestandteil seines \LaTeX-Quellcodes\footcite[vgl.][114]{VogSimRyc2018a}.
Beides habe Vor- und Nachteile, ganz zuletzt aber gelte dann doch:
\begin{quote}\textit{\enquote{On balance, the direct method (of embedding
musical excerpts in \LaTeX documents) is probably to be preferred}\footcite[vgl.
dazu][114]{VogSimRyc2018a}.}
\end{quote}


\subsubsection{Technische Vorbereitung}

Auch die Nutzung von MusiX\TeX\ bedarf einiger systemischen
Vorarbeiten\footnote{Zum generellen Verhältnis von \TeX, \LaTeX, MusiX\TeX\
und verwandter Software \cite[vgl. auch][\nopage wp]{Tennent2018a}}:

$\RHD$ Sofern es die \textit{(La)\TeX}-Distribution nicht eh schon mitliefert, muss das
MusiX\TeX-Paket\footcite[vgl.][\nopage wp]{CtanMusixTex2018a} installiert
werden\footnote{Bei Ubuntu 18.04 ist es im Paket \textit{texlive-music} enthalten.}.

$\RHD$ Wie bei \LaTeX\ üblich, gilt es danach, das MusiX\TeX-Paket mit dem
Kommando \texttt{\textbackslash{usepackage}\{musixtex\}} in die Präambel
einzubinden.

$\RHD$ Schließlich müssen auch hier die \LaTeX-Durchgänge zur Erzeugung des PDFs
modifiziert werden. MusiX\TeX\ verwendet etwas, was als \enquote{three pass
system}\footcite[Vgl.][5]{VogSimRyc2018a} bezeichnet wird: Beim ersten
\LaTeX-Durchgang schreibt das MusiX\TeX-Paket Informationen über die benötigten
Noten und ihre Kennzeichnungen in eine Datei, die wie das \TeX-Dokument heißt,
für das der \LaTeX-Durchgang gestartet worden ist -- nur dass diese
Auslagerungsdatei anstelle der Extension \textit{.tex} die Exzension
\textit{.mx1} erhält. Danach muss für diese \textit{.mx1}-Auslagerungsdatei das
Tool \textit{musicflx} aufgerufen werden. \textit{musicflx} optimiert die
Darstellung und schreibt seine Berechnungen in eine 'namensgleiche' Datei mit
der Extension \textit{mx2}. Und schließlich liest das MusiX\TeX-Paket beim
nächsten \LaTeX-Durchgang die Optimierungsinformationen wieder ein und
integriert sie in das endgültige Dokument\footnote{\cite[Vgl.
dazu][5]{VogSimRyc2018a}. Bei näherem Hinsehen 'verknäuelt'sich die Beziehung
von \LaTeX\ und MusiX\TeX\ etwas und muss 'aufgeröselt' werden. Wenn es darum
geht, welches Tool welche Datei erfolgreich auswertet, muss man sorgsam
unterscheiden:
\begin{itemize}
\item MusiX\TeX: (1) \texttt{etex mutx-file.tex} $\rightarrow$
mtx-file.mx1 (2) \texttt{musixflx mutx-file.mx1} $\rightarrow$ mutx-file.mx2 (3)
\texttt{etex mutx-file.tex} $\rightarrow$ mutx-file.dvi
\item \LaTeX: (1) \texttt{latex latx-file.tex} $\rightarrow$
latx-file.mx1 (2) \texttt{musixflx latx-file.mx1} $\rightarrow$ latx-file.mx2
(3) \texttt{latex latx-file.tex} $\rightarrow$ latx-file.dvi
\end{itemize}
Wichtig ist dabei, dass die Dateien \textit{mutx} und \textit{latx} einen
verschiedenen, auf MusiX\TeX\ bzw. \LaTeX\ ausgerichteten Inhalt haben, obwohl
es beide immer noch \TeX-Dateien sind. Neben den Tools \textit{etex} und
\textit{latex}, die -- wie beschrieben -- dvi-Dateien erzeugen, gibt es noch die
Tools \textit{pdflatex} und \textit{musixtex}. Beide erzeugen direkt pdf-Dateien:
\textit{pdflatex} aus einer reinen \LaTeX-Datei, \textit{musixtex} aus einer reinen
MusiX\TeX-Datei. In unserem Fall erzeugen wir jedoch \LaTeX-Dateien mit
MusiX\TeX-Anteilen. Deshalb müssen wir -- sozusagen händisch -- den Aufruf von
\textit{musixflx} selbst organisieren.}. Der Algorithmus ist also einfach: Findet
das MusiX\TeX-Paket als Teil des \LaTeX-Durchgangs (noch) keine
\textit{.mx2}-Auslagerungsdatei, schreibt es die \textit{.mx1}-Auslagerungsdatei,
ansonsten verarbeitet es die gefunden \textit{.mx2}-Auslagerungsdatei. Dies kann
automatisiert werden. Die entsprechende Passage aus einem Makefile könnte so
aussehen:\footnote{Wiederum gilt:
Wissenschaftliche \LaTeX-Dokumente mit Fußnoten und bibliographischen Angaben
werden eh schon über mehrere \LaTeX-Durchgänge hinweg erzeugt: Jeder einzelnen
Durchgang lagert später benötigte Angaben in Hilfsdateien aus und liest die, die
er selbst verwenden will, aus Dateien ein, die vorhergehende erzeugt haben. Wie
man beides schlüssig automatisiert, kann im Makefile dieses Tutorials eingesehen
werden ($\rightarrow$
\lnka{http://github.com/kreincke/mycsrf/tree/master/examples/musicology.de}) } 


\begin{small}
\begin{verbatim}
.tex.pdf:
# (A) the first latex pass also creates the mx1 file
  @ pdflatex $<
# (B) create the help files for the literature        
  @ bibtex `basename $< .tex`
  @ makeindex `basename $< .tex`.nlo -s cfg/nomencl.ist -o `basename $< .tex`.nls
# (C) create the mx2 file
  @ musixflx $<
# (D) create the final document 
  @ pdflatex $< 
  @ pdflatex $< 
\end{verbatim}
\end{small}

Nach Abschluss dieser Vorarbeiten kann man dann beliebig viele
MusiX\TeX-Sektionen in seinem \LaTeX-Quellcode erzeugen und mit 
MusiX\TeX-Notationen füllen, wobei jede Sektion mit
\texttt{\textbackslash{begin\{musictex\}}} eröffnet und mit
\texttt{\textbackslash{end\{musictex\}}} beendet wird:

\subsubsection{Kadenz I: einzeilig}

Wir beginnen auch hier wieder mit der Nachbildung des Beispieles aus der
Harmonielehre von Grabner\footcite[vgl.][107]{Grabner1974a}:

\begin{center}
\begin{music}%
  \largemusicsize%
  % using defaults: \instrumentnumber{1}% + \setstaffs{1}{1}  
  % + \setclef{1}{\treble}+ no bar type + \generalsignature{0}%
  \nobarnumbers%
  \startextract%
  \setdoublebar%
  \NOTEs\lcharnote{10}{(1) }\uptext{T}\zchar{-10}{I}\zw{ce}\wh{g}\en%
  \NOTEs\uptext{S}\zchar{-10}{IV}\zw{fh}\wh{j}\en%
  \NOTEs\uptext{D}\zchar{-10}{V}\zw{gi}\wh{k}\en%
  \bar%
  \NOTEs\lcharnote{10}{(2) }\uptext{T}\zchar{-10}{I}\zw{ac}\wh{e}\en% 
  \NOTEs\uptext{S}\zchar{-10}{IV}\zw{df}\wh{h}\en%
  \NOTEs\uptext{D}\zchar{-10}{{I}}\sh{g}\zw{eg}\wh{i}\en%
  \setdoublebar%
  \endextract%
\end{music}%
\cad{I}{MusiXTeX}
\end{center}

Man sieht, mit MusiX\TeX\ und \LaTeX\ kommt man dem 'Original' wirklich nahe,
näher, als mit der Rekonstruktion anhand der \textit{ABC}-Methodik. Die 
einzeilige Kadenz in der MusiX\TeX-Version wird durch folgenden Code generiert:
\begin{verbatim}
\begin{music}%
  \largemusicsize%
  % using defaults: \instrumentnumber{1}% + \setstaffs{1}{1}  
  % + \setclef{1}{\treble} + no bar type + \generalsignature{0}%
  \nobarnumbers%
  \startextract%
  \setdoublebar%
  \NOTEs\lcharnote{10}{(1) }\uptext{T}\zchar{-10}{I}\zw{ce}\wh{g}\en%
  \NOTEs\uptext{S}\zchar{-10}{IV}\zw{fh}\wh{j}\en%
  \NOTEs\uptext{D}\zchar{-10}{V}\zw{gi}\wh{k}\en%
  \bar%
  \NOTEs\lcharnote{10}{(2) }\uptext{T}\zchar{-10}{I}\zw{ac}\wh{e}\en% 
  \NOTEs\uptext{S}\zchar{-10}{IV}\zw{df}\wh{h}\en%
  \NOTEs\uptext{D}\zchar{-10}{{I}}\sh{g}\zw{eg}\wh{i}\en%
  \setdoublebar%
  \endextract%
\end{music}%
\end{verbatim}

Zumindest prinzipell ermöglicht es MusiX\TeX\ auch, Notentext in den Fließtext
einzubetten. Dazu unterbindet man den Absatzumbruch, indem man das Kommando 
\texttt{{\textbackslash}let{\textbackslash}extractline{\textbackslash}relax}
in den Quellcode einfügt:
\begin{music}%
  \nostartrule \smallmusicsize
  \let\extractline\relax
  \staffbotmarg0pt 
  \startextract
  \NOTEs\zw{e}\wh{g}\en
  \NOTEs\zw{h}\wh{j}\en
  \NOTEs\zw{i}\wh{k}\en
  \setdoublebar
  \endextract
\end{music} . Gleichwohl kann das kaum schöne Ergebnis erzeugen, weil eine
Notensatzzeile selbst mit sehr reduzierten Informationen immer noch deutlich
höher ist, als eine normale Textzeile, sodass der Zeilenfluss geradezu
'stolpern' muss.

\subsubsection{Kadenz II: Zweizeilig für ein Instrument}

MusiX\TeX\ bietet zwei Methoden der Instrumentation an: Die eine ordnet jedem
Instrument eine Stimme zu, die andere fasst mehrere Stimmen resp. Systeme - wie
bei einem Klaviertext - zu einem Instrument zusammen. Hier deshalb zunächst eine
Kadenz für ein Instrument, das in mehreren Systemen notiert wird:

\begin{center}
\begin{music}
  \normalmusicsize
  \parindent4em
  \instrumentnumber{1}
  \setstaffs{1}{2}
  \setclef{2}{\treble}
  \setclef{1}{\bass}
  \setname{1}{Piano}
  \generalsignature{2}% D-DUR
  \generalmeter{\meterfrac42}
  \startextract
  \NOTes\zmidstaff{\HH.T.....}\zh{K}\hl{a}|\zh{f}\hu{k}\en%
  \NOTes\zmidstaff{(\HH.D..7...)}\zh{I}\hl{a}|\zh{f}\sh{k}\hu{k}\en%
  \NOTes\zmidstaff{\HH.Sp.7...7.}\zh{K}\hl{N}|\zh{i}\hu{l}\en%
  \NOTes\zmidstaff{\HH.\Dohne.3.$\flat$9=$\sharp$8.7.5.}\zh{J}\hl{N}|\zh{i}\sh{l}\hu{l}\en%
  \bar
  \NOTes\zmidstaff{\HH.Tp.3....}\zh{K}\hl{M}|\zh{i}\hu{m}\en%  
  \NOTes\zmidstaff{\HH.\DD.5...7.}\zh{I}\hl{K}|\zh{l}\sh{n}\hu{n}\en%  
  \NOTes\zmidstaff{\HH.D....4-3.}
    \zh{H}\isluru{0}{K}\qu{K}\tslur{0}{J}\qu{J}|
    \zhl{l}\hl{o}\en%  
  \NOTes\zmidstaff{\HH.T.....}\zh{D}\hl{K}|\zh{m}\hu{o}\en%  
  \setdoublebar
  \endextract
\end{music}
\cad{II}{MusiXTeX}
\end{center}

Diese Zeilen werden mit folgendem MusiX\TeX-Code generiert:
\begin{verbatim}
\begin{music}
  \normalmusicsize
  \parindent4em
  \instrumentnumber{1}
  \setstaffs{1}{2}
  \setclef{2}{\treble}
  \setclef{1}{\bass}
  \setname{1}{Piano}
  \generalsignature{2}% D-DUR
  \generalmeter{\meterfrac42}
  \startextract
  \NOTes\zmidstaff{\HH.T.....}\zh{K}\hl{a}|\zh{f}\hu{k}\en%
  \NOTes\zmidstaff{(\HH.D..7...)}\zh{I}\hl{a}|\zh{f}\sh{k}\hu{k}\en%
  \NOTes\zmidstaff{\HH.Sp.7...7.}\zh{K}\hl{N}|\zh{i}\hu{l}\en%
  \NOTes\zmidstaff{\HH.\Dohne.3..7.9.}\zh{J}\hl{N}|\zh{i}\hu{l}\en%
  \bar
  \NOTes\zmidstaff{\HH.Tp.3....}\zh{K}\hl{M}|\zh{i}\hu{m}\en%  
  \NOTes\zmidstaff{\HH.\DD.5...7.}\zh{I}\hl{K}|\zh{l}\sh{n}\hu{n}\en%  
  \NOTes\zmidstaff{\HH.D....4-3.}
    \zh{H}\isluru{0}{K}\qu{K}\tslur{0}{J}\qu{J}|
    \zhl{l}\hl{o}\en%  
  \NOTes\zmidstaff{\HH.T.....}\zh{D}\hl{K}|\zh{m}\hu{o}\en%  
  \setdoublebar
  \endextract
\end{music}
\end{verbatim}

Zudem zeigt dieses Beispiel sehr eindringlich, dass man in den MusiX\TeX-Code
direkt auch die MusiX\TeX-fremden Analysesymbole von \textit{harmony} einbetten
kann.

\subsubsection{Kadenz III: Zweizeilig für mehrere Instrumente}

Bei der zweiten Methode wird jedem Instrument ein eigenes Notenliniensystem
zugeordnet und die Stimmen für die Instrumente zu einer Partitur zusammengefügt.
Dem entsprechend müssen die Noten anders verteilt werden\footnote{Aus
Platzgründen belassen wir es hier symbolisch bei zwei 'Manualen'.}.

Das folgende Beispiel soll außerdem demonstrieren, dass MusiX\TeX\ neben
der zentrierten Darstellung über das Kommando
\texttt{\textbackslash{startextract}} auch eine linksorientierte Darstellung per
\texttt{\textbackslash{startpiece}} zur Verfügung stellt, um ganze Passagen in
den eigene Text einzufügen\footnote{Bei längeren Notentexten gelingt es
\textit{musicflx} gut, ein ausgewogenes Bild zu erzeugen. Bei kürzeren muss man
gelegentlich über den Parameter \texttt{\textbackslash{hsize=XYZmm}} oder über
den erzwungenen Zeilenumbruch \texttt{\textbackslash{alaligne}} in die
Verteilung eingreifen.}:

\begin{music}
  \smallmusicsize
  \parindent4em 
  \instrumentnumber{2}
  \setstaffs{2}{1}
  \setstaffs{1}{1}
  \setclef{2}{\treble}
  \setclef{1}{\bass} 
  \songtop{2} 
  \songbottom{1} 
  \setname1{Diskant}
  \setname2{Bass}
  \generalsignature{-2}
  \generalmeter{\meterfrac58}
  \startpiece
    % Takt 1/1: B: 2 8tel (Kurzschreibweise) + D: 4tel AKK
    \NOtes \zmidstaff{T} \Dqbl I b & \zq{ik}\qu{m}\en
    % Takt 1/3: B: 2 8tel explizit  mit Vorzeichen + D: 4etl AKK
    \NOtes \zmidstaff{\HH.D.3-3$\flat$.8.7..} 
      \ibl{0}{a}{0}\qb{0}{a}\tbl{0}\fl{a}\qb{0}{a} & 
      \zq{j}\rq{l}\qu{m} \en
    % Takt 1/5: B: punktierte 16tel explizit mit Vorzeichen + D: 8tel AKK
    \NOtes \zmidstaff{\HH.S.3-3$\flat$....} 
      \ibbl{2}{N}{0}\qbp{2}{N}\roff{\tbbbl{2}\fl{N}\tqb{2}{N}} & 
      \zq{il}\cu{p}\en
  \bar % T2: 2*(B:2*8. + D:4. AKK) +  B:2*16. + D:8. AKK
    \NOtes \zmidstaff{\HH.D.8-7.7...} 
      \Dqbl M L &  \zq{jl}\qu{o} \en
    \NOtes \zmidstaff{\HH.Dp.8-8$\flat$....} 
      \ibl{1}{K}{0}\qbp{1}{K}\roff{\tbbl{1}\fl{K}\tqb{1}{K}} & 
      \zq{hm}\qu{o}\en
    \NOtes \zmidstaff{\HH.D.5-3.7...}
      \ibbl{2}{J}{0}\qbp{2}{J}\roff{\tbbbl{2}\tqb{2}{H}} & 
      \zq{j}\rq{l}\cu{m} \en
  %\alaligne
  \bar %T3 B:8. + D:8. P + B:8. P + D:8.AKK + B:8. + D:8.AKK + B:4. + D:4.AKK 
    \notes \zmidstaff{T} \qa I & \ds \zq{fk}\cu{m}\en
    \notes \zmidstaff{\HH.D..7...} \ca M & \zq{eh}\cu{j}\en
    \notes \zmidstaff{T} \qa b & \zq{df}\qu{i}\en
  \bar %T4: 8. P + (B:2*8. + D:4. AKK) + B:8. + D:8.AKK + 8. P 
    \notes \ds & \ds \en
    \notes \zmidstaff{\HH.D..7...} \ca{M J} & \zq{eh}\qu{j}\en
    \notes \qa F & \zq{eh}\qu{j}\en  
  \bar %T5: 8. P + 2*(B:8. + D:8. AKK) + 8. P +  B:8. + D:8.AKK  
    \notes \ds & \ds \en
    \notes \zmidstaff{\HH.D..7...} \ca M & \zq{eh}\cu{j}\en
    \notes \zmidstaff{\HH.D.3.7...} \ca H & \zq{j}\rq{l}\cu{m}\en
    \notes \zmidstaff{T} \qa I & \zq{ik}\qu{m}\en  
  \bar % T6: 3*(B:8.  + D:8. AKK) + B:4. + D:4. AKK
    \notes \ds & \ds \en
    \notes \zmidstaff{\HH.S..5.6..}  \ca L & \zq{g}\rq{i}\cu{j}\en
    \notes \zmidstaff{\HH.D..7...} \ca M & \zq{eh}\cu{j}\en  
    \notes \zmidstaff{T} \qa b & \zq{df}\qu{i}\en    
  \Endpiece
\end{music}
\cad{III}{MusiXTeX}

Der entsprechende MusiX\TeX-Code sieht so aus:\begin{small}
\begin{verbatim}
\begin{music}
  \smallmusicsize
  \parindent4em 
  \instrumentnumber{2}
  \setstaffs{2}{1}
  \setstaffs{1}{1}
  \setclef{2}{\treble}
  \setclef{1}{\bass} 
  \songtop{2} 
  \songbottom{1} 
  \setname1{Diskant}
  \setname2{Bass}
  \generalsignature{-2}
  \generalmeter{\meterfrac58}
  \startpiece
    % Takt 1/1: B: 2 8tel (Kurzschreibweise) + D: 4tel AKK
    \NOtes \zmidstaff{T} \Dqbl I b & \zq{ik}\qu{m}\en
    % Takt 1/3: B: 2 8tel explizit  mit Vorzeichen + D: 4etl AKK
    \NOtes \zmidstaff{\HH.D.3-3$\flat$.8.7..} 
      \ibl{0}{a}{0}\qb{0}{a}\tbl{0}\fl{a}\qb{0}{a} & 
      \zq{j}\rq{l}\qu{m} \en
    % Takt 1/5: B: punktierte 16tel explizit mit Vorzeichen + D: 8tel AKK
    \NOtes \zmidstaff{\HH.S.3-3$\flat$....} 
      \ibbl{2}{N}{0}\qbp{2}{N}\roff{\tbbbl{2}\fl{N}\tqb{2}{N}} & 
      \zq{il}\cu{p}\en
  \bar % T2: 2*(B:2*8. + D:4. AKK) +  B:2*16. + D:8. AKK
    \NOtes \zmidstaff{\HH.D.8-7.7...} 
      \Dqbl M L &  \zq{jl}\qu{o} \en
    \NOtes \zmidstaff{\HH.Dp.8-8$\flat$....} 
      \ibl{1}{K}{0}\qbp{1}{K}\roff{\tbbl{1}\fl{K}\tqb{1}{K}} & 
      \zq{hm}\qu{o}\en
    \NOtes \zmidstaff{\HH.D.5-3.7...}
      \ibbl{2}{J}{0}\qbp{2}{J}\roff{\tbbbl{2}\tqb{2}{H}} & 
      \zq{j}\rq{l}\cu{m} \en
  %\alaligne
  \bar %T3 B:8.+D:8.P + B:8.P+D:8.AKK + B:8.+ D:8.AKK + B:4.+D:4.AKK 
    \notes \zmidstaff{T} \qa I & \ds \zq{fk}\cu{m}\en
    \notes \zmidstaff{\HH.D..7...} \ca M & \zq{eh}\cu{j}\en
    \notes \zmidstaff{T} \qa b & \zq{df}\qu{i}\en
  \bar %T4: 8. P + (B:2*8. + D:4. AKK) + B:8. + D:8.AKK + 8. P 
    \notes \ds & \ds \en
    \notes \zmidstaff{\HH.D..7...} \ca{M J} & \zq{eh}\qu{j}\en
    \notes \qa F & \zq{eh}\qu{j}\en  
  \bar %T5: 8. P + 2*(B:8. + D:8. AKK) + 8. P +  B:8. + D:8.AKK  
    \notes \ds & \ds \en
    \notes \zmidstaff{\HH.D..7...} \ca M & \zq{eh}\cu{j}\en
    \notes \zmidstaff{\HH.D.3.7...} \ca H & \zq{j}\rq{l}\cu{m}\en
    \notes \zmidstaff{T} \qa I & \zq{ik}\qu{m}\en  
  \bar % T6: 3*(B:8.  + D:8. AKK) + B:4. + D:4. AKK
    \notes \ds & \ds \en
    \notes \zmidstaff{\HH.S..6...}  \ca L & \zq{g}\rq{i}\cu{j}\en
    \notes \zmidstaff{\HH.D..7...} \ca M & \zq{eh}\cu{j}\en  
    \notes \zmidstaff{T} \qa b & \zq{df}\qu{i}\en    
  \Endpiece
\end{music}
\end{verbatim} 
\end{small}

\subsubsection{Einschätzung}

Die optische Qualität, die mit MusiX\TeX\ erzeugt werden kann, ist bestechend.
Graphische Feinheiten sind bis ins Letzte hinein darstellbar und beeinflussbar.
Externe Analyse- und Theoriesymbole können problemlos als \LaTeX-Konstrukte in
den Quellcode der Noten eingebettet werden. \LaTeX\ und die
musikwissenschaftlichen Methodik kann einfach mit MusiX\TeX\ verknüpft
werden: Sie stehen sich nicht gegenseitig im Weg. Und typographische Abstriche
aufgrund von Unzulänglichkeiten oder Eigenarten der gewählten Tools braucht man
bei dieser Konstellation nicht zu machen.

Diese Freiheit und Qualität wird mit einer aufwendigen und nicht eben intuitiven
Syntax der Auszeichnungssprache MusiX\TeX\ erkauft: Die Verknüpfung von
Achtelnoten unter einem Stamm mag als Beleg dafür gelten. Wer MusiX\TeX\ 
nutzt, wird sich nicht nur das Handbuch aufwendig aneignen müssen, er wird auch
später immer wieder darauf zurückkommen.

Der Gewinn für einen musikwissenschaftlichen Text ist allerdings groß:
typographisch ist alles erreichbar. Eine saubere, zeilenorientierte
Programmierung\footnote{wie wir sie bei unserem dritten Kadenzbeispiel aus
Platzgründen leider nicht haben vollständig einhalten können} erleichtert dabei
-- wie immer -- das Arbeiten. Und nach einiger Zeit wird man per Copy\&Paste
gewiss auch flüssiger vorankommen.

% this is only inserted to eject fault messages in texlipse
%\bibliography{../bib/literature}
