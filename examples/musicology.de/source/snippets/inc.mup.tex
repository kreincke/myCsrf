% mycsrf 'for beeing included' snippet template
%
% (c) Karsten Reincke, Frankfurt a.M. 2012, ff.
%
% This text is licensed under the Creative Commons Attribution 3.0 Germany
% License (http://creativecommons.org/licenses/by/3.0/de/): Feel free to share
% (to copy, distribute and transmit) or to remix (to adapt) it, if you respect
% how you must attribute the work in the manner specified by the author(s):
% \newline
% In an internet based reuse please link the reused parts to mycsrf.fodina.de
% and mention the original author Karsten Reincke in a suitable manner. In a
% paper-like reuse please insert a short hint to mycsrf.fodina.de and to the
% original author, Karsten Reincke, into your preface. For normal quotations
% please use the scientific standard to cite
%


%% use all entries of the bibliography

\section{Mup: das veraltete Auslaufmodell ($\bigstar$)}

Zu erwähnen bliebe schließlich noch das \enquote{music publication program},
auch \enquote{Mup} genannt.\footcite[vgl.][\nopage wp]{Arkka2017a} Es wird --
genau wie \enquote{ABC}, \enquote{LilyPond} oder
\enquote{Musix\TeX}\ -- der Klasse der \enquote{Markup-Notensatzprogramme}
zugerechnet.\footcite[vgl.][\nopage wp]{WpedNotensatz2019a} Das passt zur
Selbstdarstellung: seine Entwickler sagen, es nähme eine Textdatei als Input und
erzeuge daraus eine qualitativ hochwertige PostScript-Datei zum Drucken von
Notentexten.\footcite[vgl.][\nopage wp]{Arkka2017a} Obwohl anfänglich propritäre
Software, ist das Programm seit 2012\footcite[vgl.][\nopage wp]{Arkka2017a}
echte Opensource-Software geworden.\footnote{\cite[vgl.][\nopage
wp]{Arkka2017b}.
Bei der Lizenz handelt es sich um eine Instantiierung der BSD-3-Clause Lizenz
($\rightarrow$ \href{https://opensource.org/licenses/BSD-3-Clause}
{https://opensource.org/licenses/BSD-3-Clause})} Es liegt ebenso in Form
verschiedener Distributionspakete vor, wie im Quelltext.\footcite[vgl.][\nopage
wp]{Arkka2017c}

Unglücklicherweise liefern nicht alle Distributionen \acc{Mup} fertig integriert
mit.\footnote{jedenfalls nicht Ubuntu-18.04.} Zumindest einige der genannten
Methoden, es manuell nachzuinstallieren, funktionieren nicht. Und die
Kompilation aus den Quellen läuft -- jedenfalls unter Ubuntu 18.04 -- in Fehler,
die -- wenn überhaupt -- nur mit größeren Eingriffen in das System aus dem Weg
zu räumen wären. Man darf also von größeren Installationshürden ausgehen.

Ob sich diese Hürden zu nehmen lohnt, wagen wir -- im Kontext von
Musikwissenschaft und \LaTeX\ -- zu bezweifeln: Man müsste bei Mup eine nächste
Auszeichnungssprache lernen und bekäme am Ende doch nur Graphiken, die in
bekannter Manier\footnote{$\rightarrow$ S.\pageref{IncludeGraphics}} in den
\LaTeX-Text einzubinden wären. Andere Ausgabeformate werden in der Featureliste
ebenso wenig erwähnt, wie die Möglichkeit, Harmonieanalysesymbole in den
Notentext zu integrieren.\footcite[vgl.][\nopage wp]{Arkka2017d}

Mag die \acc{Mup}-Notationstechnik über die Zeit auch noch so große Verdienste
erworben haben, für den heutigen Musikwissenschaftler wird sie erst dann
(wieder) interessant, wenn sie auch eine solche Analysemethodik anböte. Darum
belassen wir es bei diesem bloßen Hinweis auf \acc{Mup} und reichen das
spannende Abenteuer einer detaillierten Evaluation an die programmierenden
Musiker oder musizierenden Programmierer weiter, die den sympathischen
Außenseiter aktivieren mögen.

% this is only inserted to eject fault messages in texlipse
%\bibliography{../bib/literature}
