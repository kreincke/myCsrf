% mycsrf 'for beeing included' snippet template
%
% (c) Karsten Reincke, Frankfurt a.M. 2012, ff.
%
% This text is licensed under the Creative Commons Attribution 3.0 Germany
% License (http://creativecommons.org/licenses/by/3.0/de/): Feel free to share
% (to copy, distribute and transmit) or to remix (to adapt) it, if you respect
% how you must attribute the work in the manner specified by the author(s):
% \newline
% In an internet based reuse please link the reused parts to mycsrf.fodina.de
% and mention the original author Karsten Reincke in a suitable manner. In a
% paper-like reuse please insert a short hint to mycsrf.fodina.de and to the
% original author, Karsten Reincke, into your preface. For normal quotations
% please use the scientific standard to cite
%


%% use all entries of the bibliography

\subsubsection{Das musicography-Paket: eine Lösung mit Hindernissen}

Das Zusatzpaket \emph{musicography}\footcite[vgl.][\nopage
wp]{CtanMusicography2018a} vereinigt und erweitert die bisher erwähnten
Möglichkeiten. Außerdem sagen manche Fürsprecher, dass es -- im Gegensatz zu
anderen Paketen -- auch mit \emph{pdflatex} druckfähige PDF-Dateien
erzeuge\footnote{\cite[Vgl. dazu etwa][1]{Cashner2018a}. Persönlich sind uns bei
der PDF-Generierung solche Irritationen mit anderen Paketen nicht begegnet.
\emph{mycsrf} nutzt auch \emph{pdflatex}. Und unter Ubuntu 18.04 werden dabei
alle Fonts eingebunden und können alles ausdrucken, was auch am Bildschirm
sichtbar ist: so auch bei dem Beispiel \emph{musicology.de}.}.
Selbstverständlich muss auch dieses \LaTeX-Paket in die Präambel des
\LaTeX-Dokumentes eingebunden werden (\texttt{\small
\textbackslash{usepackage\{musicography\}}}), um entsprechende Befehle nutzen zu
können.

Anschließend erlaubt das Paket, Vorzeichen \{
\musFlat \ (= \texttt{\small \textbackslash{musFlat}}),
\musSharp \ (= \texttt{\small \textbackslash{musSharp}}),
\musNatural \ (= \texttt{\small \textbackslash{musNatural}}),
\musDoubleFlat \ (= \texttt{\small \textbackslash{musDoubleFlat}}),
\musDoubleSharp \ (= \texttt{\small \textbackslash{musDoubleSharp}})
\}, Notensymbole \{
\musWhole \ (= \texttt{\small \textbackslash{musWhole}}),
\musHalf \ (= \texttt{\small \textbackslash{musHalf}}),
\musQuarter \ (= \texttt{\small \textbackslash{musQuarter}}),
\musEighth \ (= \texttt{\small \textbackslash{musEighth}}),
\musSixteenth \ (= \texttt{\small \textbackslash{musSixteenth}}),
\musHalfDotted \ (= \texttt{\small \textbackslash{musHalfDotted}}),
\musQuarterDotted \ (= \texttt{\small \textbackslash{musQuarterDotted}})
\ldots
\}
und Metren wie \meterCutC \ (= \texttt{\small \textbackslash{meterCutC}})
in den Fließtext einzuarbeiten.

Gleichwohl gibt es dabei einige 'Ungereimtheiten', die den Gebrauch erschweren:
\begin{itemize}
  \item Zum ersten muss man das Paket \emph{MusiX\TeX}, sofern man es zusammen
  mit \emph{musicology} verwenden will, nach \emph{musicology} in die Präambel
  einbinden. Andernfalls 'meckert' \emph{musicology}, dass der Befehl
  \texttt{meterC} schon definiert sei. Bindet man \emph{MusiX\TeX} tatsächlich
  nach \emph{musicology} ein, kann \emph{MusiX\TeX} den von \emph{musicology}
  eingeführten Befehl offensichtlich problemlos redefinieren. Leider passen die
  Symbole dann -- wie man hier sehen kann -- von der Größe her nicht mehr zu
  einander: \emph{musicology} $\rightarrow$ \meterCutC : \meterC $\leftarrow$
  \emph{MusiX\TeX}.
  \item Außerdem sind die Symbole für die Achtel- (\musEighth) und die
  Sechzehntelnoten (\musSixteenth) aus Musikersicht völlig unbrauchbar. Solche
  Zeichen gibt es in der Musik(wissenschaft) nicht.
\end{itemize}

Wenn man dieses Paket verwenden will, ist also eine gewisse Vorsicht angesagt.


% this is only inserted to eject fault messages in texlipse
%\bibliography{../bib/literature}
