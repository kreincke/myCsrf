% mycsrf 'for beeing included' snippet template
%
% (c) Karsten Reincke, Frankfurt a.M. 2012, ff.
%
% This text is licensed under the Creative Commons Attribution 3.0 Germany
% License (http://creativecommons.org/licenses/by/3.0/de/): Feel free to share
% (to copy, distribute and transmit) or to remix (to adapt) it, if you respect
% how you must attribute the work in the manner specified by the author(s):
% \newline
% In an internet based reuse please link the reused parts to mycsrf.fodina.de
% and mention the original author Karsten Reincke in a suitable manner. In a
% paper-like reuse please insert a short hint to mycsrf.fodina.de and to the
% original author, Karsten Reincke, into your preface. For normal quotations
% please use the scientific standard to cite
%


%% use all entries of the bibliography

\subsection{Sonderzeichen: der Standard}

Neben Zeichen für die Schriftsprache als solche bot \LaTeX\ - der
Mathematik sei Dank - immer schon auch graphische Symbole an, die wie
Schriftzeichen in einer Zeile platziert werden können\footcite[vgl.][543ff et
passim]{MitGoo2005a}. Sie werden im Mathematikmodus notiert, haben also die Form
\texttt{\small \$\textbackslash{ZEICHENNAME}\$}. Und in diesem Fundus von Sonderzeichen
gibt es auch die drei Symbole $\sharp$ (= \texttt{\small \$\textbackslash{sharp}\$}),
$\flat$ (= \texttt{\small \$\textbackslash{flat}\$}) und $\natural$ (=
\texttt{\small \$\textbackslash{natural}\$}).

Diese Zeichen können ohne Zusatzpaket in einem \textit{LaTeX}-Quelltext genutzt
werden. Unnötig zu erwähnen, dass sie -- für sich genommen -- nicht ausreichen,
Musik textuell zu erfassen. Gleichwohl wird man in der Kombination mit anderen
Ansätzen gelegentlich auf sie zurückkommen wollen\footnote{In diesem
Zusammenhang wäre auch zu erwähnen, dass Unicode und seine Encodierung UTF-8 von
sich aus einen Bereich mit 'Musikzeichen' definiert hat, nämlich die Zeichen
zwischen \texttt{U+1D100} und \texttt{U+1D1FF} (\cite[Vgl. dazu][\nopage
wp.]{Koellerwirth2015a}). Wenigstens unter (pdf)\LaTeX. bedarf Nutzung dieser
Zeichen -- wenn überhaupt möglich -- zusätzlicher Maßnahmen  \cite[Zum
Zusammenhang zwischen Unicode und UTF( vgl.][\nopage wp.]{Kuhn2019a} }.

% this is only inserted to eject fault messages in texlipse
%\bibliography{../bib/literature}
