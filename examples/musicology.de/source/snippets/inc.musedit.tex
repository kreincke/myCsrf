% mycsrf 'for beeing included' snippet template
%
% (c) Karsten Reincke, Frankfurt a.M. 2012, ff.
%
% This text is licensed under the Creative Commons Attribution 3.0 Germany
% License (http://creativecommons.org/licenses/by/3.0/de/): Feel free to share
% (to copy, distribute and transmit) or to remix (to adapt) it, if you respect
% how you must attribute the work in the manner specified by the author(s):
% \newline
% In an internet based reuse please link the reused parts to mycsrf.fodina.de
% and mention the original author Karsten Reincke in a suitable manner. In a
% paper-like reuse please insert a short hint to mycsrf.fodina.de and to the
% original author, Karsten Reincke, into your preface. For normal quotations
% please use the scientific standard to cite
%




\subsubsection{MusEdit (-)}

\label{MusEdit}Die neuere Sichtung von Notensatzprogrammen listet \acc{MusEdit}
schon nicht mehr\footcite[vgl.][\nopage wp]{WpedNotensatz2019a}, \acc{MusicXML}
erwähnt sie noch als \enquote{notation editor for Windows}, der schließlich
freie Software geworden sei\footcite[vgl.][\nopage wp]{MusicXML2018b}. Die
ursprüngliche Domain \texttt{http://www.musedit.com/} ist nicht mehr erreichbar.
Die Software wird über eine Ersatzhomepage 'vertrieben'. Dort  bekennt er Autor
unten, dass er die Software seit 2011 nicht mehr pflegen könne. Und oben
'kündigt' er an, dass die Software 'bald' Open-Source-Software
werde\footcite[vgl.][\nopage wp]{Rogers2011a} -- was aber offensichtlich
(bisher) nicht geschehen ist.

Unabhängig davon, wie gut dieses Programm früher einmal mit
\acc{MusicXML}-Dateien umgehen konnte, heute ist es (für Musikwissenschaftler)
keine Alternative mehr -- und schon gar nicht für diejenigen, die Wert auf freie
Software legen. Für den Ehrenstern hat uns das nicht gereicht.

% this is only inserted to eject fault messages in texlipse
%\bibliography{../bib/literature}
