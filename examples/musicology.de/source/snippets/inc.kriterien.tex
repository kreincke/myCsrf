% mycsrf 'for beeing included' snippet template
%
% (c) Karsten Reincke, Frankfurt a.M. 2012, ff.
%
% This text is licensed under the Creative Commons Attribution 3.0 Germany
% License (http://creativecommons.org/licenses/by/3.0/de/): Feel free to share
% (to copy, distribute and transmit) or to remix (to adapt) it, if you respect
% how you must attribute the work in the manner specified by the author(s):
% \newline
% In an internet based reuse please link the reused parts to mycsrf.fodina.de
% and mention the original author Karsten Reincke in a suitable manner. In a
% paper-like reuse please insert a short hint to mycsrf.fodina.de and to the
% original author, Karsten Reincke, into your preface. For normal quotations
% please use the scientific standard to cite
%


%% use all entries of the bibliography

\section{Kriterien}

Wollen wir den besten Weg finden, brauchen wir Maßstäbe. Deshalb werden wir den
Notensatzsystemen drei Referenzkadenzen vorlegen und erwarten, dass sie diese
optisch ansprechend erfassen und wiedergeben können, und zwar einschließlich der
zugehörigen Harmonieanalyse:

Als erste und einfachste Aufgabe beziehen wir uns auf ein Beispiel aus der
Harmonielehre von Grabner\footcite[vgl.][107]{Grabner1974a}:

\begin{center}
\begin{music}%
  \largemusicsize%
  % using defaults: \instrumentnumber{1}% + \setstaffs{1}{1}  
  % + \setclef{1}{\treble}+ no bar type + \generalsignature{0}%
  \nobarnumbers%
  \startextract%
  \setdoublebar%
  \NOTEs\lcharnote{10}{(1) }\uptext{T}\zchar{-10}{I}\zw{ce}\wh{g}\en%
  \NOTEs\uptext{S}\zchar{-10}{IV}\zw{fh}\wh{j}\en%
  \NOTEs\uptext{D}\zchar{-10}{V}\zw{gi}\wh{k}\en%
  \bar%
  \NOTEs\lcharnote{10}{(2) }\uptext{T}\zchar{-10}{I}\zw{ac}\wh{e}\en% 
  \NOTEs\uptext{S}\zchar{-10}{IV}\zw{df}\wh{h}\en%
  \NOTEs\uptext{D}\zchar{-10}{{I}}\sh{g}\zw{eg}\wh{i}\en%
  \setdoublebar%
  \endextract%
\end{music}%
\cad{I}{Referenz}
\end{center}

Die zweite Referenzkadenz soll die Erfassung harmonisch komplexer Zusammenhänge
abfordern:

\begin{center}
\begin{music}
  \normalmusicsize
  \parindent4em
  \instrumentnumber{1}
  \setstaffs{1}{2}
  \setclef{2}{\treble}
  \setclef{1}{\bass}
  \setname{1}{Piano}
  \generalsignature{2}% D-DUR
  \generalmeter{\meterfrac42}
  \startextract
  \NOTes\zmidstaff{\HH.T.....}\zh{K}\hl{a}|\zh{f}\hu{k}\en%
  \NOTes\zmidstaff{(\HH.D..7...)}\zh{I}\hl{a}|\zh{f}\sh{k}\hu{k}\en%
  \NOTes\zmidstaff{\HH.Sp.7...7.}\zh{K}\hl{N}|\zh{i}\hu{l}\en%
  \NOTes\zmidstaff{\HH.\Dohne.3.$\flat$9=$\sharp$8.7.5.}\zh{J}\hl{N}|\zh{i}\sh{l}\hu{l}\en%
  \bar
  \NOTes\zmidstaff{\HH.Tp.3....}\zh{K}\hl{M}|\zh{i}\hu{m}\en%  
  \NOTes\zmidstaff{\HH.\DD.5...7.}\zh{I}\hl{K}|\zh{l}\sh{n}\hu{n}\en%  
  \NOTes\zmidstaff{\HH.D....4-3.}
    \zh{H}\isluru{0}{K}\qu{K}\tslur{0}{J}\qu{J}|
    \zhl{l}\hl{o}\en%  
  \NOTes\zmidstaff{\HH.T.....}\zh{D}\hl{K}|\zh{m}\hu{o}\en%  
  \setdoublebar
  \endextract
\end{music}
\cad{II}{Referenz}
\end{center}

Und die dritte Referenzkadenz soll das in einen schwierigeren
rhythmish-metrischen Kontext einbetten:

\begin{music}
  \smallmusicsize
  \parindent4em 
  \instrumentnumber{2}
  \setstaffs{2}{1}
  \setstaffs{1}{1}
  \setclef{2}{\treble}
  \setclef{1}{\bass} 
  \songtop{2} 
  \songbottom{1} 
  \setname1{Diskant}
  \setname2{Bass}
  \generalsignature{-2}
  \generalmeter{\meterfrac58}
  \startpiece
    % Takt 1/1: B: 2 8tel (Kurzschreibweise) + D: 4tel AKK
    \NOtes \zmidstaff{T} \Dqbl I b & \zq{ik}\qu{m}\en
    % Takt 1/3: B: 2 8tel explizit  mit Vorzeichen + D: 4etl AKK
    \NOtes \zmidstaff{\HH.D.3-3$\flat$.8.7..} 
      \ibl{0}{a}{0}\qb{0}{a}\tbl{0}\fl{a}\qb{0}{a} & 
      \zq{j}\rq{l}\qu{m} \en
    % Takt 1/5: B: punktierte 16tel explizit mit Vorzeichen + D: 8tel AKK
    \NOtes \zmidstaff{\HH.S.3-3$\flat$....} 
      \ibbl{2}{N}{0}\qbp{2}{N}\roff{\tbbbl{2}\fl{N}\tqb{2}{N}} & 
      \zq{il}\cu{p}\en
  \bar % T2: 2*(B:2*8. + D:4. AKK) +  B:2*16. + D:8. AKK
    \NOtes \zmidstaff{\HH.D.8-7.7...} 
      \Dqbl M L &  \zq{jl}\qu{o} \en
    \NOtes \zmidstaff{\HH.Dp.8-8$\flat$....} 
      \ibl{1}{K}{0}\qbp{1}{K}\roff{\tbbl{1}\fl{K}\tqb{1}{K}} & 
      \zq{hm}\qu{o}\en
    \NOtes \zmidstaff{\HH.D.5-3.7...}
      \ibbl{2}{J}{0}\qbp{2}{J}\roff{\tbbbl{2}\tqb{2}{H}} & 
      \zq{j}\rq{l}\cu{m} \en
  %\alaligne
  \bar %T3 B:8. + D:8. P + B:8. P + D:8.AKK + B:8. + D:8.AKK + B:4. + D:4.AKK 
    \notes \zmidstaff{T} \qa I & \ds \zq{fk}\cu{m}\en
    \notes \zmidstaff{\HH.D..7...} \ca M & \zq{eh}\cu{j}\en
    \notes \zmidstaff{T} \qa b & \zq{df}\qu{i}\en
  \bar %T4: 8. P + (B:2*8. + D:4. AKK) + B:8. + D:8.AKK + 8. P 
    \notes \ds & \ds \en
    \notes \zmidstaff{\HH.D..7...} \ca{M J} & \zq{eh}\qu{j}\en
    \notes \qa F & \zq{eh}\qu{j}\en  
  \bar %T5: 8. P + 2*(B:8. + D:8. AKK) + 8. P +  B:8. + D:8.AKK  
    \notes \ds & \ds \en
    \notes \zmidstaff{\HH.D..7...} \ca M & \zq{eh}\cu{j}\en
    \notes \zmidstaff{\HH.D.3.7...} \ca H & \zq{j}\rq{l}\cu{m}\en
    \notes \zmidstaff{T} \qa I & \zq{ik}\qu{m}\en  
  \bar % T6: 3*(B:8.  + D:8. AKK) + B:4. + D:4. AKK
    \notes \ds & \ds \en
    \notes \zmidstaff{\HH.S..5.6..}  \ca L & \zq{g}\rq{i}\cu{j}\en
    \notes \zmidstaff{\HH.D..7...} \ca M & \zq{eh}\cu{j}\en  
    \notes \zmidstaff{T} \qa b & \zq{df}\qu{i}\en    
  \Endpiece
\end{music}
\cad{III}{Referenz}

Zuletzt werden wir ja, wie angekündigt, auch versuchen, gute Frontends für die
\LaTeX-kompatiblen Notensatzsysteme zu finden. Von ihnen werden verlangen, dass
sie die Referenzkadenz II erfassen und adäquat exportieren. Und von Konvertern
werden wir dasselbe verlangen. Wie wir verifizieren, dass eine Refrenzkadenz
adäquat exportiert bzw. konvertiert worden ist, werden wir später genauer
erläutern.\footnote{$\rightarrow$ S. \pageref{ExportVerifikation}}

% this is only inserted to eject fault messages in texlipse
%\bibliography{../bib/literature}
