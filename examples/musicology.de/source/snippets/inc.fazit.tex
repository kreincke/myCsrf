% mycsrf 'for beeing included' snippet template
%
% (c) Karsten Reincke, Frankfurt a.M. 2012, ff.
%
% This text is licensed under the Creative Commons Attribution 3.0 Germany
% License (http://creativecommons.org/licenses/by/3.0/de/): Feel free to share
% (to copy, distribute and transmit) or to remix (to adapt) it, if you respect
% how you must attribute the work in the manner specified by the author(s):
% \newline
% In an internet based reuse please link the reused parts to mycsrf.fodina.de
% and mention the original author Karsten Reincke in a suitable manner. In a
% paper-like reuse please insert a short hint to mycsrf.fodina.de and to the
% original author, Karsten Reincke, into your preface. For normal quotations
% please use the scientific standard to cite
%




\chapter{Fazit}

Damit haben wir das Ende unser Untersuchung erreicht. Fassen wir das Unterfangen
zusammen:

Aus Anlass einer anstehenden, größeren musikwissenschaftlichen Arbeit mussten
wir in \LaTeX-Texte Notenbeispiele einbetten können, und zwar insbesondere der
ausgefeilten Symbolkomplexe, wie sie in der funktionalen Harmonieanalyse üblich
sind. Eine einfache Anleitung dazu gab und gibt es dazu bisher nicht. Die Suche
im Netz verweist stattdessen auf eine Fülle von Tools und Techniken, deren
Nutzen und Nutzbarkeit und Kombinierbarkeit recht unklar bleibt. So haben wir
dieses Feld ausgeleuchtet, um den von der Qualität besten und von der
Handlichkeit einfachsten Weg zu finden:

Notenbeispiele können auf verschiedenen Wegen in \LaTeX-Texten eingebettet
werden. Wir haben drei Backendsystem gefunden, nämlich
\acc{\LaTeX\,+\,ABC}, \acc{\LaTeX\,+\,MusiX\TeX} und
\acc{\LaTeX\,+\,LilyPond}.

Nur \acc{\LaTeX\,+\,MusiX\TeX} bieten im Verbund mit dem \LaTeX-Tool
\acc{harmony} von sich aus die Option, hinreichend ausdrucksstarke
Harmonieanalysesymbole in die Notenbeispiele zu integrieren. Das Druckergebnis
ist exzellent.

Der Nachteil dieses Ansatzes liegt in der Komplexität und Unhandlichkeit der
Auszeichnungssprache \acc{MusiX\TeX}. Leider gibt es kein graphisches oder
semi-graphisches Frontend für dieses Backend. Die Idee, ein existierendes
Frontend über einen Konverter dafür nutzbar zu machen, scheitert entweder daran,
dass es keinen entsprechenden Konverter gibt (\acc{ly2musictex}) oder dass die
existierenden Konverter nicht adäquat arbeiten (\acc{abc2ly} resp. \acc{abc2mtex}).

Bei \acc{\LaTeX\,+\,LilyPond} kann man hinreichend ausdrucksstarke
Harmonieanalysesymbole über eine -- allerdings noch unfertige --
Zusatzbibliothek namens \acc{Harmonyli} in den \acc{LilyPond}-Notentext integrieren.
Dieses Verfahren hat drei Vorteile:

\begin{itemize}
  \item Zum ersten steht mit \acc{LilyPond} eine im Vergleich zu \acc{Musix\TeX}
  deutlich einfachere Auszeichnungssprache zur Verfügung.
  \item Zum zweiten können diese Harmonieanalysesymbole
  damit auch über den \acc{LilyPond}-spezifischen semi-graphischen Editor
  \acc{Frescobaldi} eingegeben werden.
  \item Und schließlich kann man für \acc{LilyPond} auch den graphischen Editor
  \acc{Muse\-Score} als Frontend verwenden, sofern man in Kauf nimmt, vor der
  eigentlichen Arbeit auf \acc{\LaTeX-LilyPond}-Ebene den Konverter
  \acc{musicxml2ly} auf den \acc{MuseScore}-XML-Export anzuwenden, das Ergebnis
  in \acc{Frescobaldi} zu laden und dort die Harmonieanalysesymbole in einem
  zweiten Schritt einzugeben.
\end{itemize}

Allerdings hat auch dieses Verfahren Nachteile:

\begin{itemize}
  \item Zum einen werden die Notenbeispiele so nicht nativ in den \LaTeX-Text
  eingebettet, sondern 'nur' als Graphik. Damit muss man die Skalierungsfragen
  besonders bedenken.
  \item Zum anderen muss die Bibliothek \acc{Harmonyli} noch optimiert werden,
  insbesondere im Hinblick auf das graphische Ergebnis.
\end{itemize}

Fragt sich also zuletzt, welche Methode wir selbst anwenden werden? 

Nun, wir werden versuchen, die Bibliothek \acc{Harmonyli} zusammen mit der
\acc{LilyPond} Community zu verbessern, bis das graphische Ergebnis der
unfraglichen Qualität von \acc{Music\TeX\ \& Harmony} gleichkommt. Kurzfristig
werden wir auf \acc{Musix\TeX \& Harmony} zugreifen,allerdings nur dann, wenn
es wirklich notwendig ist.

Insgesamt sind wir aber sehr froh, dass wir den Djungel der Möglichkeiten, der
sich nach der Internetrecherche abzeichnet, gelichtet und die beiden wirklich
gangbaren Wege gefunden haben.



% this is only inserted to eject fault messages in texlipse
%\bibliography{../bib/literature}
