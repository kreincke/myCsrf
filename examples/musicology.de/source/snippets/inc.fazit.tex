% mycsrf 'for beeing included' snippet template
%
% (c) Karsten Reincke, Frankfurt a.M. 2012, ff.
%
% This text is licensed under the Creative Commons Attribution 3.0 Germany
% License (http://creativecommons.org/licenses/by/3.0/de/): Feel free to share
% (to copy, distribute and transmit) or to remix (to adapt) it, if you respect
% how you must attribute the work in the manner specified by the author(s):
% \newline
% In an internet based reuse please link the reused parts to mycsrf.fodina.de
% and mention the original author Karsten Reincke in a suitable manner. In a
% paper-like reuse please insert a short hint to mycsrf.fodina.de and to the
% original author, Karsten Reincke, into your preface. For normal quotations
% please use the scientific standard to cite
%




\chapter{Fazit}

Damit haben wir das Ende unser Untersuchung erreicht. Fassen wir das Unterfangen
zusammen:

Aus Anlass einer anstehenden, größeren musikwissenschaftlichen Arbeit mussten
wir in \LaTeX-Texte Notenbeispiele einbetten können, und zwar samt der
ausgefeilten Symbolkomplexe, wie sie in der funktionalen Harmonieanalyse üblich
sind. Eine einfache Anleitung dazu gab bisher nicht. Die Suche im Netz verwies
stattdessen auf eine Fülle von Tools und Techniken, deren Nutzen und Nutzbarkeit
und Kombinierbarkeit unklar blieb. So haben wir dieses Feld ausgeleuchtet, um
den von der Qualität besten und von der Handlichkeit einfachsten Weg zu finden:

Notenbeispiele können auf verschiedenen Wegen in \LaTeX-Texten eingebettet
werden. Wir haben drei Backendsystem gefunden, nämlich
\acc{\LaTeX\,+\,ABC}, \acc{\LaTeX\,+\,MusiX\TeX} und
\acc{\LaTeX\,+\,LilyPond}.

Letztlich kann man aus den Tools in und um \acc{\LaTeX\,+\,ABC} kein akzeptables
Editiersystem für Musikwissenschaftler zusammenstellen. Wenn man unbedingt will,
steht der \acc{EasyABC}-Editor und sein Export nach \acc{musicxml} bereit, um
von dort aus über den Konverter \acc{musicxml2ly} in die \acc{LilyPond}-Welt
hinüberzuleiten. Dass der Nusikwissenschaftler dann zusätzlich immer auch
\acc{LilyPond} können muss, ergibt sich aus der Tatsache, dass die wirklich
komplexen Harmonyanalysesymbole erst über die \acc{LilyPond}-Zusatzbibliothek
\acc{harmonyli.ly} in das Notenbild eingebracht werden können. Ob und in wie
weit es Sinn macht, dafür zwei Repräsensationssprachen -- also \acc{ABC} und
\acc{LilyPond} -- zu lernen, möge jeder für sich selbst entscheiden.


Deutlich klarer sieht die Lage bei \acc{\LaTeX\,+\,MusiX\TeX} und
\acc{\LaTeX\,+\,LilyPond} aus:

\acc{\LaTeX\,+\,MusiX\TeX} -- im Verbund mit dem \LaTeX-Tool \acc{harmony} --
und \acc{\LaTeX\,+\,LilyPond} -- in Kombination mit der \acc{LilyPond}
Bibliothek \acc{harmonyli.ly} -- bieten von sich aus die Option, hinreichend
ausdrucksstarke Harmonieanalysesymbole in die Notenbeispiele zu integrieren.
Das Druckergebnis ist exzellent und ganz auf der Höhe, die ein \LaTeX-Nutzer
erwartet.

Der \textbf{Vorteil} des \acc{\LaTeX\,+\,\textbf{MusiX\TeX}}-Ansatzes liegt in
der bruchlosen Integration in das gewohnte \LaTeX-Handling und in der Tatsache,
dass \LaTeX-Syntagmen in MusiX\TeX-Bereichen verwendeten werden können.

Der \textbf{Nachteil} der \acc{\LaTeX\,+\,\textbf{MusiX\TeX}}-Methode liegt in
der Komplexität und Unhandlichkeit der Auszeichnungssprache \acc{MusiX\TeX}.
Leider gibt es kein graphisches oder semi-graphisches Frontend für dieses
Backend. Die Idee, ein existierendes Frontend über einen Konverter dafür nutzbar
zu machen, scheitert entweder daran, dass es keinen entsprechenden Konverter
gibt (\acc{ly2musictex}) oder dass die existierenden Konverter nicht adäquat
arbeiten (\acc{abc2ly} resp. \acc{abc2mtex}).

Der \textbf{Nachteil} des \acc{\LaTeX\,+\,\textbf{LilyPond}}-Ansatzes besteht
dagegen darin, dass die Notenbeispiele nicht nativ in den \LaTeX-Text
eingebettet werden, sondern 'nur' als vorab erzeugte Graphik. Damit muss man die
Skalierungsfragen ebenso gesondert bedenken, wie man seine Make-Prozedur auf die
Vorabnutzung von \acc{lilypond-book} umstellen muss. Und dann bleibt immer noch,
dass man \LaTeX-Konstrukte nicht in \acc{Lilypond}-Umgebungen verwenden kann.

Die \textbf{Vorteile} der \acc{\LaTeX\,+\,\textbf{LilyPond}}-Methode machen sie
jedoch zum Mittel der Wahl:

\begin{itemize}
  \item Zum ersten steht mit \acc{LilyPond} eine im Vergleich zu \acc{Musix\TeX}
  deutlich einfachere Auszeichnungssprache zur Verfügung.
  \item Zum zweiten bietet sich mit \acc{Frescobaldi} ein ausgezeichneter
  semi-graphischen Editor an, der auch die eingebundene Bibliothek.
   \acc{harmonyli.ly} und die daraus genutzten Funktionen korrekt auswertet.
  \item Neben diesem Standardeditor kann man auch auf das gute Eclipse-Plugin
  \acc{elysium} zurückgreifen, wenn man \acc{LilyPond} erstellen will, und zwar
  insbesondere dann, wenn man auch sein \LaTeX-Texte per
  \acc{{\TeX}lipse}-Eclipseplugin erzeugt.
  \item Außerdem darf man hoffen, dass mit \acc{Canorus} noch ein dritter
  Kandidat entsteht, der wenigstens zukünftig gute Dienste leisten kann.
  \item Und schließlich kann man für \acc{LilyPond} auch den graphischen Editor
  \acc{Muse\-Score} als Frontend verwenden, sofern man in Kauf nimmt, vor der
  eigentlichen Arbeit auf \acc{\LaTeX-LilyPond}-Ebene den Konverter
  \acc{musicxml2ly} auf den \acc{MuseScore}-XML-Export anzuwenden, das Ergebnis
  in \acc{Frescobaldi} zu laden und dort die Harmonieanalysesymbole in einem
  zweiten Schritt einzugeben.
\end{itemize}

Damit dürfte auch klar sein, welche Methode wir selbst anwenden werden, nämlich
\acc{\LaTeX\,+\,LilyPond\,+\,harmonyli.ly}. Das wäre nicht möglich gewesen, wenn
es sich hier nicht um Freie Open Source Software gehandelt hätte. Erst das gab
uns die Möglichkeit, die Dinge selbst zu ergänzen, die wir für ein
graphisch und inhaltlich adäquates Ergebnis noch benötigten.

Insgesamt sind wir froh, den Djungel der Möglichkeiten, der sich nach der reinen
Internetrecherche abgezeichnet hatte, gelichtet und die wirklich gangbaren Wege
gefunden zu haben. Jetzt wissen wir, woran wir sind. Wir wünschten allerdings,
wir hätten diesen Text schon eingangs von jemand anderem ausgehändigt bekommen,
anstatt ihn selbst geschrieben haben zu müssen. Das hätte uns sehr viel Zeit
gespart.

Möge also unser Text anderen den Aufwand für eine solche Tool-Evaluation sparen.
Und möge er anderen als Ausgang für Updates, Verfeinerungen und Zusatzarbeiten
dienen. Deshalb sei er auch als CC-BY-SA lizensierter Text veröffentlicht.

Frankfurt 2019-11-26
Karsten Reincke

\chapter{Nachrede: Das 'vergessene Kapitel'?}

Vielleicht mag sich der eine oder andere Leser nun wünschen, den idealen Weg
einmal im Detail vorgeführt zu bekommen. Diesen sei gesagt, dass sie das nicht
mehr nötig haben: Was zur Nutzung von \acc{LilyPond} unter und mit \LaTeX\ zu
sagen war, haben wir im Kapitel über das Backend \acc{LilyPond}
ausgeführt.\footnote{\ra\ \pageref{LilyPondBackend}} Und was zu einer guten
Nutzung der Bibliothek \acc{harmonyli.ly}\footcite[vgl.][\nopage
wp.]{ReinckeBlum2019a} noch zu sagen wäre, bietet bereits dessen
Tutorial.\footcite[vgl.][]{Reincke2019b} Insofern dürfen Sie jetzt direkt
loslegen.

% this is only inserted to eject fault messages in texlipse
%\bibliography{../bib/literature}
