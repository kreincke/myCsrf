% mycsrf 'for beeing included' snippet template
%
% (c) Karsten Reincke, Frankfurt a.M. 2012, ff.
%
% This text is licensed under the Creative Commons Attribution 3.0 Germany
% License (http://creativecommons.org/licenses/by/3.0/de/): Feel free to share
% (to copy, distribute and transmit) or to remix (to adapt) it, if you respect
% how you must attribute the work in the manner specified by the author(s):
% \newline
% In an internet based reuse please link the reused parts to mycsrf.fodina.de
% and mention the original author Karsten Reincke in a suitable manner. In a
% paper-like reuse please insert a short hint to mycsrf.fodina.de and to the
% original author, Karsten Reincke, into your preface. For normal quotations
% please use the scientific standard to cite
%

\subsection{Jniz (-)}

\label{Jniz}Ein ganz spezielles Programm ist \acc{Jniz}. Es bezeichnet sich
selbst als \enquote{support tool} für Komponisten, das es erlaube, mehrstimmige
Stücke gemäß der klassischen Regeln (semi-automatisch) zu
harmonisieren.\footcite[vgl.][\nopage wp]{Grandjean2019a} Das zum Download
angebotene Programm kann per \texttt{java -jar jnizpro.jar} gestartet werden.
Die letzte wirklich lauffähige Version stammt aus dem Jahr 2016, die aktuellste
von 2019.\footcite[vgl.][\nopage wp]{Jniz2019b} Dazu gibt es ein
Tutorial.\footcite[vgl.][\nopage wp]{Grandjean2019c}

Das Programm erlaubt keinen Import und lädt nur Dateien im eigenen Format.
Kadenzen können in einfacher Form gebildet, allerdings nur sehr bedingt manuell
gestaltet werden.\footnote{Es ist uns nicht gelungen, die Referenzkadenz II
einzugeben.} Der angebotene Export als \acc{MusicXML-}, \acc{LilyPond-},
\acc{MIDI-} oder \acc{PDF-}Datei ist darum nur begrenzt hilfreich.

Speziell ist dieses Programm nicht nur seines Ansatzes wegen, sondern auch ob
seiner Lizensierung. Hier verbietet der Autor, das Programm zu verkaufen. Und
mehr noch, er untersagt auch, seine Quellen weiterzugeben: \enquote{You do not
have the right to sell, distribute Jniz or use its sources under penalty of
law.}\footcite[vgl.][\nopage wp]{Grandjean2019b} Damit unterläuft es seine
'Selbstklassifikation' als freie Software. Dass es dennoch unter Sourceforge
gehostet wird, mutet merkwürdig an.\footnote{Immerhin bezeichnet sich
Sourceforge als \enquote{Open Source community resource dedicated to helping
open source projects} ($rightarrow$ \href{https://sourceforge.net/}
{https://sourceforge.net/}).} Noch bedenklicher aber ist die Tatsache, dass
\acc{Jniz} selbst unter der \acc{GPL} lizensierte \acc{LilyPond}-Bibliotheken
nutzt. Damit wird es zu einem abgeleiteten Werk und müsste -- des
\acc{Copyleft-Effektes} wegen -- ebenfalls unter der GPL veröffentlicht
werden.\footnote{Dies wird in den Reviews auf der Repositorysite diskutiert
(\cite[vgl.][\nopage wp]{Jniz2019a})}

So ist dieses Programm in unserem Kontext nicht nur funktional kaum nutzbar,
seine Nutzung ist auch lizenztechnisch bedenklich. Das ist zu wenig für den
einen 'Ehrenstern'.



% this is only inserted to eject fault messages in texlipse
%\bibliography{../bib/literature}
