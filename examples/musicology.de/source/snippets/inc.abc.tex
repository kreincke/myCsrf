% mycsrf 'for beeing included' snippet template
%
% (c) Karsten Reincke, Frankfurt a.M. 2012, ff.
%
% This text is licensed under the Creative Commons Attribution 3.0 Germany
% License (http://creativecommons.org/licenses/by/3.0/de/): Feel free to share
% (to copy, distribute and transmit) or to remix (to adapt) it, if you respect
% how you must attribute the work in the manner specified by the author(s):
% \newline
% In an internet based reuse please link the reused parts to mycsrf.fodina.de
% and mention the original author Karsten Reincke in a suitable manner. In a
% paper-like reuse please insert a short hint to mycsrf.fodina.de and to the
% original author, Karsten Reincke, into your preface. For normal quotations
% please use the scientific standard to cite
%


%% use all entries of the bibliography


\subsection{ABC: einfach und vielfach genutzt}

Das \textit{ABC}-Notationssystem ist eine ASCII basierte Methode zum Dokumentieren
von Musik. Initial entworfen wurde sie, um die Lieder zu
erfassen\footcite[vgl.][\nopage Subpage 'Intro']{Chambers2018a}. So bezeichnet
sich dieses Verfahren -- weil schon lange gepflegt und vielfach genutzt -- denn
auch als \textit{das} textbasierte Musiknotationssystem und als \textit{den}
Defacto-Standard für Volksmusik\footnote{\cite[vgl.][\nopage wp]{Abc2018a}. Im
Original: \enquote{the text-based music notation system and the de facto
standard for folk and traditional music}.}. Für dieses Verfahren gibt es eine
Fülle von Konvertern, die ganz verschiedene Nutzungsszenarien
bedienen\footcite[vgl.][\nopage wp]{Abc2018b}. Und eines der
Verwertungsszenarios ist eben die Einbettung von \textit{ABC-Noten} in
\LaTeX-Text, wie es von dem abc-\LaTeX-Paket ermöglicht
wird\footcite[vgl.][\nopage wp]{CtanAbc2018a}.

Die Art der Notation -- die natürlich für alle Verwertungsszenarios im
Wesentlichen gleich ist -- kann per Online-Tutorials gelernt werden, etwa dem
von Steve Mansfield\footcite[vgl.][\nopage wp]{Mansfield2016a} oder dem von
John Chambers\footcite[vgl.][\nopage wp]{Chambers2018a}; die Besonderheiten im
Rahmen der \LaTeX-Nutzung werden im Pakethandbuch
erläutert\footcite[vgl.][\nopage wp]{Gregorio2016a}.

Auch wenn die \textit{ABC-Notation} zunächst 'nur' Lieder erfassen sollte und
Mehrstimmigkeit nicht unbedingt das Anliegen der initialen Programmierer gewesen
ist, gibt es mit \textit{ABC-PLUS}\footcite[vgl.][\nopage wp]{Gonzato2018a}
inzwischen auch für mehrsystemige Partituren eine Lösung, die gut dokumentiert
ist\footcite[vgl.][XVff]{Gonzato2018b} und die Integration in
\LaTeX-Texte ermöglicht\footcite[vgl.][134]{Gonzato2018b}.

\subsubsection{Technische Vorbereitung}

Trotz aller Einfachheit bedarf es zur erfolgreichen Nutzung der ABC-Notation
in und mit einer \LaTeX-Datei einiger systemischen Vorbereitungen:

$\RHD$ Zunächst muss -- ganz unabhängig von \LaTeX\ -- das Tool \textit{abcm2ps}
installiert werden\footnote{Unter Ubuntu: \texttt{sudo apt-get abcm2ps}}. Es
wird genutzt, um die im  ersten Durchgang des PDF-Erzeugungsprozess aus der
\LaTeX-Datei extrahierten \textit{ABC}-Daten in ein Postscriptbild
umzuwandeln, das dann bei der nächsten Runde -- automatisiert -- anstelle des
\textit{ABC}-Codes in die \LaTeX-Datei eingesetzt wird.
  
$\RHD$ Sofern es die \TeX-Distribution nicht eh schon mitliefert, muss das
abc-\LaTeX-Paket\footcite[vgl.][\nopage wp]{CtanAbc2018a} installiert
werden\footnote{Bei Ubuntu 18.04 ist es im Paket \textit{texlive-music} enthalten.}.
  
$\RHD$ Danach muss -- wie bei \LaTeX\ üblich -- dieses
abc-\LaTeX-Paket  mit dem entsprechenden Kommando in der Präambel
eingebunden (\texttt{\textbackslash{usepackage}\{abc\}}).
  
$\RHD$ Schließlich müssen die \LaTeX-Durchgänge zur Erzeugung des PDFs
modifiziert werden: Um das Bild aus dem extrahierten Code generieren zu können,
ist es notwendig, dass das generierende Programm \textit{pdflatex} hilfsweise auch
externe Programme aufrufen darf\footnote{Dies ist normalerweise aus
Sicherheitsgründen untersagt, denn sonst könnten die Computer, die die Dokumente
erzeugen, aus 'Dokumenten' heraus manipuliert werden.}, also mit der Option
\texttt{--shell-escape} aufgerufen werden. Die entsprechende Passage aus einem
Makefile könnte so aussehen:\footnote{Wissenschaftliche \LaTeX-Dokumente
mit Fußnoten und bibliographischen Angaben werden eh schon über mehrere
\LaTeX-Durchgänge hinweg erzeugt: Jeder einzelnen Durchgang lagert später
benötigte Angaben in Hilfsdateien aus und liest die, die er selbst verwenden
will, aus Dateien ein, die vorhergehende erzeugt haben.
Verwendet man das \textit{abc-LaTeX}-Paket generiert der erste Durchgang also
nicht mehr nur die bibliographischen Hilfsdateien, sondern auch die
korrespondierenden \textit{ABC}-Dateien, für die er anschließend das externe Tool
\textit{abcm2ps} aufruft. Dies konvertiert die extrahierten Vorlagen dann
seinerseits in Postscript- und PDF-Dateien. Und der nächste
\LaTeX-Durchgang integriert dann diese erzeugten Dateien in das
Gesamtwerk. Wie man das schlüssig automatisiert, zeigt das Makefile dieses
Tutorials ($\rightarrow$
\lnka{http://github.com/kreincke/mycsrf/tree/master/examples/musicology.de}) } 

\begin{small}
\begin{verbatim}
.tex.pdf:
# (A) create a tmpdir for storing the abc help files
  mkdir -p abc
# (B) the first latex pass which extracts also the abc-files
  @ pdflatex $<
# (C) create the help files for the literature        
  @ bibtex `basename $< .tex`
  @ makeindex `basename $< .tex`.nlo -s cfg/nomencl.ist -o `basename $< .tex`.nls
# (D) generate the abc pictures by calling the external program abcm2ps
  @ pdflatex --shell-escape $<
# (E) mv the results into the tmpdir to enable the next pass to find them
  mv *.ps abc/
# (F) create the final document 
  @ pdflatex --shell-escape $< 
  @ pdflatex --shell-escape $< 
# (G) mv the recreated results also into the tmpdir for a better cleansing
  mv *.ps abc/
# (H) cleasing the environment
  rm -rf abc
\end{verbatim}
\end{small}

Nach Abschluss dieser Vorbereitungen kann man beliebig viele
\textit{ABC}-Sektionen in seinem \LaTeX-Quellcode erzeugen und mit 
\textit{ABC}-Notationen füllen, wobei jede Sektion mit
\texttt{\textbackslash{begin\{abc\}}} eröffnet und mit
\texttt{\textbackslash{end\{abc\}}} beendet wird:

\subsubsection{Kadenz I: einzeilig}

\begin{center}
\begin{abc}[name=abc/cadenca1]
X:1
M:
L:1/4
K: C
"T"[C4E4G4] "S"[F4A4c4] "D"[G4B4d4] || 
w: I IV V 
"T"[A,4C4E4] "S"[D4F4A4] "D"[E4^G4B4] ||
w: I IV V 
\end{abc}
\cad{I}{ABC}
\end{center}

Diese einzeilige Kadenz bildet das Beispiel 187 aus der Harmonielehre von Grabner
nach\footcite[vgl.][107]{Grabner1974a}. Sie wird mit folgendem Code erzeugt:

\begin{verbatim}
\begin{abc}[name=abc/cadenca1]
X:1
M:none
L:1/4
K: C
"T"[C4E4G4]  "S"[F4A4c4] "D"[G4B4d4] || 
w: I IV V 
"T"[A,4C4E4] "S"[D4F4A4] "D"[E4^G4B4] ||
w: I IV V 
\end{abc}
\end{verbatim}


\subsubsection{Kadenz II: mehrzeilig}

Als Beleg dafür, dass auch die \textit{ABC}-Methodik mittlerweile tatsächlich
mehrsystemige Konstrukte erzeugen kann, hier ein entsprechendes Beispiel:

\begin{center}
\begin{abc}[name=abc/cadenca2]
X: 1
L: 1/4 
K: D 
M: none
%%score { RH | LH }
V: RH clef=treble name="Piano" stem=up
V: LH clef=bass stem=down
[V:RH]    [F2d2]         [F2^d2]        [B2e2]           [B2^e2]   |
[V:LH] "T"[D,2A,2] "(D7)"[B,,2A,2]  "Sp7"[D,2G,2]    "D79"[C,2G,2] |
[V:RH]    [B2f2]         [e2^g2]        [e2(a1]a1)       [a2f2]   ||
[V:LH] "Tp"[D,2F,2] "DD7"[B,,2D,2] "D4-3"[A,,2(D,1]C,1) "T"[D,2D,,2] ||
\end{abc}
\cad{II}{ABC}
\end{center}

Es wird mit folgendem Code erzeugt:
\begin{verbatim}
\begin{abc}[name=abc/cadenca2]
X: 1
L: 1/4 
K: D 
M: none
%%score { RH | LH }
V: RH clef=treble name="Piano" stem=up
V: LH clef=bass stem=down
[V:RH]    [F2d2]         [F2^d2]        [B2e2]           [B2^e2]   |
[V:LH] "T"[D,2A,2] "(D7)"[B,,2A,2]  "Sp7"[D,2G,2]    "D79"[C,2G,2] |
[V:RH]    [B2f2]         [e2^g2]        [e2(a1]a1)       [a2f2]   ||
[V:LH] "Tp"[D,2F,2] "DD7"[B,,2D,2] "D4-3"[A,,2(D,1]C,1) "T"[D,2D,,2] ||
\end{abc}
\end{verbatim}

\subsubsection{Bewertung}

Offensichtlich kommt man mit der \textit{ABC}-Notationsmethode und \LaTeX\ dem
realen oder intendierten 'Original' recht nahe, und zwar ohne großen
Schreibaufwand: Die musikalische Notation ist (fast) intuitiv verständlich, Stufen-
und Funktionssymbole werden als normale Schriftzeichen in das Notenbild
integriert, und zwar über die Option, Liedtexte (Wörter) unter Noten und
'Griffsymbole' über Noten einzufügen. Gleiches gilt für die mehrzeilige Kadenz.

Man sieht diesen Beispielen aber auch an, dass das Ergebnis optisch nicht
optimal ist: Die Breite der Notensysteme wird automatisch auf die Seitenbreite
gesetzt und die Analysesymbole sitzen nicht mittig unter den
Akkorden\footnote{Es soll jedoch die Möglichkeit geben, Parameter an das
\textit{abcm2ps}-Tool aus dem Notencode heraus zu übergeben, mit dem solche
Aspekte zu steuern wären. Unglücklicherweise ist es uns nicht gelungen, das zu
aktivieren. Allerdings wird man diese 'Linksbündigkeit' auch bei den anderen
Ansätzen finden; es ist also -- wenn überhaupt -- gar keine
\textit{ABC}-spezifische Unschönheit.}. \label{AppraisalABC}Wichtiger ist jedoch,
dass man nicht in der Lage ist, die eingefügten Analysesymbole typographisch der
Stufen- oder Funktionstheorie entsprechend zu gestalten: Weder können hoch- und
tiefgestellte Kleinsymbole oder Sonderzeichen eingefügt werden, noch die
Alterationszeichen $\sharp$, $\flat$ oder $\natural$ aus den Sonderzeichen -
ganz zu schweigen von der Einbettung jener ausgefeilten Konstrukte, die das
\textit{harmony}-Paket zur Verfügung stellt\footnote{Das kann ja auch nicht sein,
weil die \textit{ABC-Notation} außerhalb von \LaTeX\ in ein Bild umgewandelt wird,
sodass eventuell noch eingebettete \textit{(La)\TeX}-Konstrukte gar nicht mehr
ausgewertet werden.}.

Außerdem muss man gelegentlich dort, wo man die Einfachheit der Notation
erhalten will, kleine 'Hacks' verwenden und Unsauberkeiten in Kauf nehmen -- wie
wir es etwa bei dem Vorhalt in der 2. Kadenz getan haben. Solche Stellen
typographisch sauber zu gestalten, würde verlangen, die Notation in mehrere
Stimmen aufzublähen\footnote{Tatsächlich bietet ABC Plus 'nur' die Option, die
Stimmen innerhalb eines Systems für das ganze Stück festzulegen, nicht aber
taktweise (\cite[vgl.][49f]{Gonzato2018b}). Wir hätten also -- um dem Preis einer
signifikanten Mehrarbeit -- die unschöne Halteklammer im Bass des zweiten Kadenz
vermeiden können, wenn wir den Bass durchweg zweistimmig angelegt hätten. Sie
einfach als \Halb\ zu notieren, ist aber keine Option, weil \textit{abc} den
folgenden Schlussakkord im Bass dann unter die Vorhaltsauflösung im Diskant
positioniert. Deshalb unser Hack der 'Vorhaltsverdoppelung mit gleichem Ton'
($\rightarrow$ S. \pageref{\cadlab{II}{ABC}}). Wir werden später erkennen, dass
\textit{PMX} unter demselben Positionierungsproblem leidet ($\rightarrow$ S.
\pageref{\cadlab{II}{PMX}}). Allerdings bietet es die taktweise Notation in
mehreren Stimmen, was die Mehrarbeit erträglich macht.
}. Und das kompliziert das Handling deutlich.

Schließlich bliebe zu erwähnen, dass das ABC-Verfahren sozusagen auf einer
kleinen, aber auch von anderen gern genutzten 'Mogelei' beruht: es wird der
Notentext ja nicht in (La)\TeX-Code verwandelt, sondern in ein Bild, das dann in
den (La)\TeX-Code eingebunden wird\footnote{Die Stufen des Prozesses spiegeln
sich in den Dateien, die anhand des Namensparameters gebildet werden: eine
\textit{.abs}-Datei, die korrespondierenden Postscriptdateien \textit{.ps} und
\textit{.eps} und die finale \textit{.pdf}-Datei.}. Und wie immer kann die
Einbindung von externen Graphiken ungünstigenfalls zu einigen Irritationen
hinsichtlich des Seitenumbruchs führen\footnote{Diese Idee ist natürlich nicht
ehrenrührig, im Gegenteil: wir werden noch sehen, dass andere Methoden sie auch
verwenden: LilyPondBook wäre ein Beispiel dafür, die manuelle Verbindung von PMX
und Musix\TeX\ die andere. Und man darf daraus sofort auch folgern, dass man
selbst ebenso vorgehen kann, um den Output ganz anderer Notensatzprogramme in
seine \LaTeX-Code einzubinden. Wir werden die Einbindung von Graphiken deshalb
auch ganz generell beschreiben.
($\rightarrow$ S. \pageref{IncludeGraphics})}.\label{AbcGraphics}

Trotzdem bietet die \textit{ABC}-Notationsmethode ein technisch ausgereiftes
Verfahren mit einer sehr steilen Lernkurve, das schnell ansprechende Ergebnisse
erzeugt. Wer sich für seine Verwendung entscheidet, nimmt gewisse optische und
funktionelle Einschränkungen in Kauf, die allerdings gerade bei der
theoretischen Musikwissenschaft nur bedingt akzeptabel sind.
Dafür eignet man sich eine Technik an, die Dank der vielfältigen Konverter in
vielen Szenarien verwendet werden kann\footnote{Pars pro toto
\cite[vgl.][\nopage wp]{Rosen2018a}.}.
% this is only inserted to eject fault messages in texlipse
% \bibliography{../bib/literature}
