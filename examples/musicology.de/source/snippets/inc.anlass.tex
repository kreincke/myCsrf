% mycsrf 'for beeing included' snippet template
%
% (c) Karsten Reincke, Frankfurt a.M. 2012, ff.
%
% This text is licensed under the Creative Commons Attribution 3.0 Germany
% License (http://creativecommons.org/licenses/by/3.0/de/): Feel free to share
% (to copy, distribute and transmit) or to remix (to adapt) it, if you respect
% how you must attribute the work in the manner specified by the author(s):
% \newline
% In an internet based reuse please link the reused parts to mycsrf.fodina.de
% and mention the original author Karsten Reincke in a suitable manner. In a
% paper-like reuse please insert a short hint to mycsrf.fodina.de and to the
% original author, Karsten Reincke, into your preface. For normal quotations
% please use the scientific standard to cite
%


%% use all entries of the bibliography

\subsection{Anlass}

\begin{quote}\textit{Was man braucht und nicht findet, muss man selbst herstellen.}
\end{quote}

Der Impuls, dieses Tutorial zu schreiben, kam mit dem Beginn eines größeren
musikwissenschaftlichen Projektes. Wie gewohnt, bestand meine erste Aufgabe
darin, meine Arbeit zu organisieren. Unnötige Tipparbeit ist mir zuwider. Lieber
verbringe ich doppelt so viel Zeit damit, die Anlegenheit zu automatisieren.
Glücklicherweise waren mir \acc{\LaTeX}, \acc{Bib\TeX}\ und \acc{JabRef} längst
schon vertraut. Meinen optimalen Zitierstil für das Schreiben
geisteswissenschaftlicher Arbeiten hatte ich bereits
konfiguriert\footcite[vgl.][\nopage wp]{Reincke2018a} und
dokumentiert\footcite[vgl][2ff]{Reincke2018b}. Damit hätte es im Kern losgehen
können.

Allerdings war für mich noch völlig unklar, wie man den Notentexte und -- vor
allem -- musikalische Analysen in \LaTeX-Texte einbindet. Nur das Gefühl sagte:
so schwierig würde das nicht werden. Am besten wäre natürlich ein
No\-ta\-tions\-system, das mich den Notentext hätte erfassen und -- sozusagen auf
Knopfdruck -- in meinen Text integrieren lassen. Danach galt es zu suchen.

Leider schwiegen sich meine sehr guten, einschlägigen \LaTeX-Bücher dazu
aus\footcite[vgl.][vi ff, insbesondere 905 u. 909: das umfangreiche Register
erwähnt weder Musik im allgemeinen noch LilyPond oder MusiX\TeX im
Besonderen]{Voss2012a}, selbst wenn sie auch Randbereiche
behandelten\footcite[vgl.][vii ff, insbesondere 1080 u.
1087: auch dieses umfangreiche Register erwähnt weder Musik im allgemeinen noch
LilyPond oder MusiX\TeX im Besonderen.]{MitGoo2005a}. Die entsprechende
Internetrecherche überrollte mich dagegen: so viele Notationssysteme und Tools,
aber kein systematischer Überblick.\footnote{Rühmlich die Ausnahme von
\cite[][\nopage wp]{Thoma2018a}. Allerdings ging sie nicht in die Tiefe, die ich
benötigte.}

Wollte ich meine große Arbeit nicht gefährden und Sackgassen vermeiden, musste
ich die Möglichkeiten zuerst grundsätzlich sichten. Andernfalls hätte ich nur
eine Variante willkürlich herausgegriffen und wäre Gefahr gelaufen, schließlich
-- trotz aller investierten Zeit und Arbeit -- doch 'aufs falsche Pferd gesetzt'
zu haben. Was ich brauchte war -- sozusagen -- eine technisch fundierte
'Landkarte' der Methoden und Tools, die mir den besten Weg weisen konnte. Und so
traf mich -- wieder einmal -- die Erkenntnis:

\begin{quote}\textit{Was man braucht und im Internet nicht findet, muss man
selbst erzeugen und -- sofern man sich als Teil der Open-Source-Community
versteht -- als Ausgleich für die schon erhaltene freie Software der
Allgemeinheit wieder zur Verfügung stellen.}
\end{quote}


% this is only inserted to eject fault messages in texlipse
%\bibliography{../bib/literature}
