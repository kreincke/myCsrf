% fodina humanities 'for being included' snippet template
%
% (c) Karsten Reincke, Frankfurt a.M. 2010, 2011, ff.
%
% This LaTeX-File is licensed under the Creative Commons Attribution-ShareAlike
% 3.0 Germany License (http://creativecommons.org/licenses/by-sa/3.0/de/): Feel
% free 'to share (to copy, distribute and transmit)' or 'to remix (to adapt)'
% it, if you '... distribute the resulting work under the same or similar
% license to this one' and if you respect how 'you must attribute the work in
% the manner specified by the author ...':
%
% In an internet based reuse please link the reused parts to www.fodina.de and
% mention the original author Karsten Reincke in a suitable manner. In a
% paper-like reuse please insert a short hint to www.fodina.de and to the
% original author, Karsten Reincke, into your preface. For normal quotations
% please use the scientific standard to cite.
%
\section{Form Follows Function: Wie soll es aussehen?}

Seit Ewigkeiten schwärme ich von der altphilologischen Nachweis- und
Zitiermethode, vom mächtigen Anmerkungsapparat in Fußnoten - auch wenn ich
meinen Hang dazu nicht immer ausleben darf und mich stattdessen auf Endnoten
anstelle von Fußnoten beschränken muss.

Für beide Arten hatte ich mir längst schon deren bruchlose Nachbildung im
Satzsystem \emph{LaTeX} gewünscht - wohl wissend, dass dieses eben nicht aus der
europäischen Geisteswissenschaft heraus entstanden ist, sondern aus der
computerisierten Mathematik und der anglo-amerikanischen Schreibtradition, wie
sie sich im \textit{Handbook for Writers of Research
Papers}\footcite[vgl.][]{ModLanAss2009a} niederschlug. Meine Leidenschaft für
die europäische Alternative ging so weit, dass ich immer wieder einmal -
selbstverständlich stets mehr oder minder erfolglos - eigenhändig Stil- und
Bibliotheksdateien editiert habe: Das arme
\emph{natib}\footcite[vgl.][]{Daly2000a} musste ebenso dran glauben, wie das
geschundene \emph{custom-bib}\footcite[vgl.][]{Daly2007a}. Dabei waren meine
Wünsche doch so einfach:

Ich wollte Zitate weder über inline-Referenzen belegt sehen, noch sie selbst so
belegen müssen, weder über esoterische Nummernblöcke[42], noch durch kryptische
BibTEX-Keys[Daly2000a], die sich - wie doch oft genug mit eigenen Augen erfahren
- immer nur als Stolpersteine im Lesefluss erwiesen. Ich liebe den Subtext im
Anmerkungsapparat, mit dem mich Autoren über die konzentrierte Argumentation
ihrer Haupttexte hinaus lustvoll auf Nebenwegen durch die mäandernde
Forschungsgeschichte führen. Diesem schwärmenden Hin und Her bin ich verfallen.
Das möchte ich genießen - möglichst ohne großes Blättern. Und natürlich möchte
auch ich es anderen anbieten können, ohne ihnen ein permanentes Seitengefrickel
zumuten zu müssen: das Mittel der Wahl sind demnach Fußnoten\footnote{Ohne
Frage, ein solcher Anmerkungsapparat kann angeberisch wirken: \glq{}Manno, was
der alles weiß, will der etwa, dass ich das alles lese?\grq{}. Will 'der'
natürlich nicht, im Gegenteil: wenn Sie ihm glauben, brauchen Sie seine Belege
nicht zu lesen. Aber wenn Sie zweifeln, wenn Sie ihn überprüfen wollen, dann
legt er Ihnen alles offen. Oder wenn Sie zu dem einen oder anderen nebenseitigen
Punkt doch gern mehr Informationen hätten, dann finden Sie dort die Details. Wie
dem auch sei: die vom Anschein her bescheidenere Variante wäre die mit Endnoten
anstelle von Fußnoten - auch wenn der Leser dafür den Preis des
Hin-und-Her-Blätterns zahlen muß: Ohne Blättern keine Belege oder Zusatzinfos.}.

Aber egal, ob in Form von Fuß- oder Endnoten\footnote{Um die Unterschiede direkt
erkennbar zu machen, habe ich diesen Text in zwei Versionen erzeugt, einmal
konfiguriert für einen Fußnotenapparat (fodinaClassicalScholar\ldots{}pdf),
einmal für einen Endnotenapparat (fodinaHumanities\ldots{}pdf). Diese
Unterschiede zu kennen, ist auch insofern wichtig, als man seinen Text
nachträgliche leider nur dann rein konfigurativ umstellen kann, wenn eine
Kleinigkeit vorab beachtet: nur $\backslash$footcite\{\ldots\} aus jurabib
verwenden, insbesondere keine (mehrfachen) $\backslash$cite\{\ldots\} innerhalb
von $\backslash$footnote\{\ldots\}, das verträgt bei einer Ersetzung von
\textit{footnote} durch \textit{endnote} $\backslash$endnote\{\ldots\} nicht.},
es ist der Anmerkungsapparat und seine expliziten bibliographischen Angaben, die
mir den Forschungskontext aufspannen. Schärfer noch: Ich möchte, dass eine
Quelle beim ersten Nachweis bibliographisch vollständig aufgeschlüsselt
wird\footnote{\cite[vgl.][195ff]{RueStaFra1980a} - dieses Werk bewahre ich aus
zwei Gründen auf: Zum einen bietet es auf den genannten Seiten eine griffige
Zusammenfassung der Regeln. Zum anderen zeigt der Rest der Seiten, wie man es
auf keinen Fall machen darf, wenn man spannende Wissenschaftslektüre schreiben
möchte: Pädagogisierte Texte sind nicht per se schon kunden- oder
leserorientiert. }. Ich möchte neben allen Angaben zu allen Autoren und allen
Titelfeinheiten auch die Auflage erfahren können, auf Übersetzungen und Reihen
hingewiesen werden und editorische Sonderfälle erkennen können\footcite[wie z.B.
bei][]{Covey2006a}. Und ja, ich stehe der Moderne mitnichten ablehnend
gegenüber: mittlerweile erleichtert die Angabe der ISBN das Wiederfinden des
richtigen Buches erheblich. Also wünsche ich mir auch deren Nennung\footnote{Als
Web-Mensch kann ich mich natürlich einem weiteren Zugeständnis von
\textit{jurabib} an die Moderne nicht verschließen: Wir müssen heute auch
Netz-Dokumente über URLs zitieren, wohl wissend, dass diese unter der Hand
geändert werden können. Mit den neuen BibTeX-Feldern 'url' und 'urldate' bietet
\textit{jurabib} die Möglichkeit, auf solche Dokumente Bezug zu nehmen, und zwar
nicht nur unter der Angabe der URL (Unified Resource Locator), sondern auch
unter Angabe des Abrufdatums\ldots was letztlich nicht mehr besagt, als dass der
jeweilige Autor um die Volatilität seiner Quellen weiß. Ich selbst gehe sogar
noch weiter: ich vermerke im BibTeX-Feld 'note', ob ich ein Werk effektiv in der
Hand gehabt habe (\textit{Print}), ob ich es im bibliothekseigenen Netz als PDF
etc. eingesehen habe (\textit{BibWeb/PDF}) oder ob ich es dem ganz
'unbeständigen' Internet entnommen habe (\textit{FreeWeb/PDF} oder
\textit{FreeWeb/HTML} oder sonstiges)}. Erst wenn im Laufe der Argumentation
erneut auf dieselbe Quelle zurückgegriffen wird\footcite[vgl.
dazu][194]{RueStaFra1980a} reicht mir ein verkürzter Beleg\footcite[wie z.B.
jetzt wieder bei][]{Covey2006a}, der mich zwanglos auf die mnemotechnisch
richtige Bahn bringt, ohne mir eine verklausulierte Geheimsprache aufzunötigen:
denn \emph{kurz} meint schließlich nicht \emph{esoterisch}\footnote{Man sieht,
dass es funktioniert: Den \emph{R\"uckriem/Stary/Franck} habe ich gerade
bibliographisch ebenso ausführlich ausgewiesen, wie den \emph{Covey}.
Anschließend habe ich erneut auf beide Quellen verwiesen, nun aber über die
Kurzform \emph{Autor: Titel, Jahr}. Dies alles ermöglicht das Heilmittel
\emph{Jurabib}( \cite[vgl. dazu][]{Berger2004a}). An einer Stelle blieb jedoch
ein Desiderat, das ich auf spezielle Weise umgehen musste: In den Kurzverweisen
wünsche ich mir neben Autor und Kurztitel immer auch das Jahr. Dies erleichtert
mir die Rezeption der Forschungs\emph{geschichte}. Jurabib bietet dieses
Schmankerl (noch) nicht an und liefert stattdessen nur den \emph{(short)author}
und den \emph{(short)title}. Allerdings kostet es ja nicht viel, an den eh
manuell zu erstellenden Kurztitel im Bibtex-Feld einfach auch noch das
Erscheinungsjahr anzuhängen. Schon hat man/ich, was man/ich will.}.

Immerhin: Wenn der Autor und ich uns im Haupttext auf eine verfeinerte Analyse
fremder Gedankengängen einlassen\footcite[vgl. etwa][32]{Allen2001a} und wenn
wir dabei mehrfach\footcite[vgl.][139]{Allen2001a} verschiedene
Passagen\footcite[vgl.][139]{Allen2001a} desselben Werkes referieren bzw.
rezipieren müssen, ja dann reichen die kleinen Hinweise, dass es sich wieder um
dieselbe Passage oder wieder um dasselbe Werk - wenn auch mit anderer Passage -
oder wieder um denselben Autor handelt: Nehmen wir an, wir zitierten zunächst aus
der Kritik der Urteilskraft\footcite[vgl.][9]{KantKdU1974} und anschließend aus
der Einleitung der Kritik der reinen Vernunft\footcite[vgl.][45]{KantKdV1974},
um sofort danach erst auf den 1. Absatz aus dem Kapitel \emph{Transzendentale
Ästhetik} verweisen zu müssen\footcite[vgl.][69]{KantKdV1974}, bevor auf den 3
Absatz derselben Seite eingehen können\footcite[vgl.][69]{KantKdV1974}, dann
müßte unser Anmerkungsapparat zunächst die beiden kompletten Belege auflisten,
gefolgt von einem \emph{ders., a.a.O. + neue Seite} wiederum gefolgt von einem
\emph{ders., ebda} erzeugen. Gingen wir nun zurück auf die \emph{Kritik der
Urteilskraft}\footcite[vgl.][9]{KantKdU1974}, dürfte unser Apparat sicher nicht
mehr mit dem Kürzel \emph{ders., a.a.O} arbeiten, sondern könnte allenfalls das
Kürzel \emph{ders.: Titel} anbieten oder sollte wenigstens - wie hier dann auch
geschehen - über den erneut genannten Autor + Kurztitel + Jahr einen neuen
expliziten Aufsatzpunkt verwenden\footnote{Wenn Sie der Demonstration gefolgt
sind, werden Sie sofort gesehen haben, dass in den wiederaufnehmenden
Anmerkungen eben nicht 'ders.' steht, sondern 'ds.' - was selbstverständlich
(vorerst noch) keine korrekte Abkürzung ist. Der Grund für diese 'Ersetzung' ist
einfach: Manchmal haben Werke mehrere Autoren. Dann dürfte im Text nicht
{\itshape ders.} erscheinen, sondern der Plural {\itshape dies.} Und oft genug
sind Bücher ja auch von Frauen geschrieben. In unseren heutigen Zeiten darf das
weibliche Geschlecht aber nicht mehr unter dem grammatischen 'bloß' mitgedacht
werden; Autorinnen können und werden zurecht auf einem {\itshape dies.}
bestehen, wenn schon ein {\itshape ders.} benutzt wird. Aus diesem Dilemma
könnte uns nur ein LaTeX befreien, dass aus der vorhergehenden Fußnote und dem
Original den Plural oder das Geschlecht der Autoren ableitet. Kurz:
Flektionsbewusstsein wäre nötig. LaTeX bietet das m.W. nicht. Also habe ich zu
einer anderen Lösung gegriffen, ich habe mir eine kollektive Abkürzung für die
einzelnen Abkürzungen definiert: {\itshape ds.} stehe für {\itshape ders.} und
{\itshape dies.}, so, wie {\itshape dies.} selbst schon für {\itshape dieselbe}
und {\itshape dieselben} steht.

Jurabib bietet von sich aus eine andere Lösung aus dem Dilemma: erst lässt die
Referenz auf den/die Autor(in/en) schlicht weg und verwendet anstelle von
{\itshape ders., a.a.O. + neue Seite} resp. {\itshape ders., ebda} schlicht
{a.a.O. + neue Seite} resp. {\itshape a.a.O.}. Das kann man machen. Ich
persönlich bevorzuge das durchgehaltene Muster 'Autor' : 'Buch', es erleichtert
mir das Lesen.}.

Auch die innere Struktur der bibliographischen Angaben wollte ich auf diesen Stil
ausgerichtet sehen. So wünschte ich mir bei kollektiv erarbeiteten Werken, dass
zwar der erste Autor mit dem Nachnamen zuerst genannt wird, dass alle folgenden
Autoren jedoch "`in natürlicher Weise"' erwähnt werden, also zuerst mit den
Vornamen und dann mit den Nachnamen\footnote{ wie \cite[hier bei:][]{Woods1991a}
oder \cite[hier bei:][]{RusNor2004a} oder ...} Außerdem wollte ich den jeweils
letzten Autor durch eine Konjunktion eingebunden sehen\footcite[... hier
bei:][]{SegEvaTay2009a}

Doch nicht nur Bücher, sondern auch Artikel\footcite[vgl. z.B.][]{Hays1985a}
sollten diesem Muster folgen, sei es solche aus Sammlungen\footcite[s.
etwa][]{Brachman1985a} oder solche aus Zeitschriften\footcite[s.
etwa][]{McCarthy1980a}: Immer wollte ich deren Form bruchlos als Erst- oder
Wiederholungszitat erkennen können, beim Sammlungsartikel\footcite[vgl.
erneut][]{Hays1985a} genauso, wie beim Zeitschriftenartikel\footcite[s.
nochmals][]{McCarthy1980a}. Und ich wollte dieses Muster auch dann so haben,
wenn es ökonomischer wäre, die bibliographischen Daten einer enthaltenden
Sammlung nicht als gesonderten Eintrag im Bibtex-File aufzunehmen, sondern sie -
sozusagen inline - in denen des enthaltenen Artikels einzubauen: Auch dann
sollte erst die Langversion erscheinen\footcite[vgl.][23]{RotCum2011a} und beim
erneuten Zitat die Kurzversion, sei nach der
'ds.ebda'-Mimik\footcite[vgl.][23]{RotCum2011a} oder - nach einem dazwischen
geschobenen anderen Zitat\footcite[vgl.][9]{KantKdU1974} wieder mit der
Kurztitelform\footcite[vgl.][23]{RotCum2011a}. Nur in der Bibliographie sollte
die Sammlung eben nicht als eigener Eintrag erscheinen, sondern mit dem Artikel
so verwoben sein, wie er auch im Text erschiene.

