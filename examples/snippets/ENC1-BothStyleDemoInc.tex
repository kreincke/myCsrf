% fodina humanities 'for being included' snippet template
%
% (c) Karsten Reincke, Frankfurt a.M. 2010, 2011, ff.
%
% This LaTeX-File is licensed under the Creative Commons Attribution-ShareAlike
% 3.0 Germany License (http://creativecommons.org/licenses/by-sa/3.0/de/): Feel
% free 'to share (to copy, distribute and transmit)' or 'to remix (to adapt)'
% it, if you '... distribute the resulting work under the same or similar
% license to this one' and if you respect how 'you must attribute the work in
% the manner specified by the author ...':
%
% In an internet based reuse please link the reused parts to www.fodina.de and
% mention the original author Karsten Reincke in a suitable manner. In a
% paper-like reuse please insert a short hint to www.fodina.de and to the
% original author, Karsten Reincke, into your preface. For normal quotations
% please use the scientific standard to cite.
%
% [ Derived from 'mykeds Classical Scholar Research Framework' 
%   mykeds-CSR-framework (c) K. Reincke CC BY 3.0  http://www.mykeds.net/ ]
%

\section{The Annotated Text}

A scientific paper\footnote{This English text is an extract of the longer
German version. The English text 'only' demonstrates the
classical sholar style (of humanities), The German text also explains why
this style is (a somewhat) better than the style recommended by the \textit{MLA
Handbook for Writers of Research Papers}. Both, the English and the German Version of \textit{Service For Readers and Scholars} are
generated on the base of the \textit{mykeds-CSR-framework}. It's an Open
(Source) Document and can freely be downloaded under
http://www.mykeds.net/en/docus/mykeds-csr.html} written in \textit{Classical Scholar Research} style not only wants to argue for a new position or insight
but to offer it's reader the possibility to adopt the research history by the
way. The history of humanities is the history of the secondary literature. Hence
the footnotes in the \textit{Classical Scholar Research} style present all
information about a work if it's quoted for the first time. If it is quoted
again, it's referred by the short title of the bib-file\footnote{I prefer the
pattern 'Author-Name: Short-Title, Year'. But I didn't find any solution to
convince jurabib to do this automatically. Therefore in each field 'shorttile'
of my bib-files I append at the real short title a comma followed by the year. If
anyone knows a better solution I would be glad to get a message from him.}. If
it is cited multiply - directly in a row of notes on the same page, then and
only then the shortcuts \textit{id.} and \textit{ibid.} or \textit{l.c.} should
be used. Let me demonstrate what this means:

\begin{itemize}
  \item A \textit{book}\footcite[cf.][123]{AllHen2008a} is quoted for the first time.
  \item A \textit{proceedings}\footcite[cf.][234]{Brachman1985a} is quoted for the first time.
  \item An \textit{inproceedings}\footcite[cf.][345]{Hays1985a}is quoted for the first time.
  \item An \textit{article of a journal}\footcite[cf.][456]{McCarthy1980a} is quoted for
  the first time.
  \item A \textit{book}\footcite[cf.][123]{AllHen2008a} is quoted for the second time.
  \item A \textit{proceedings}\footcite[cf.][234]{Brachman1985a} is quoted for the second
  time.
  \item An \textit{inproceedings}\footcite[cf.][345]{Hays1985a}is quoted for the
  second time.
  \item An \textit{article of a journal}\footcite[cf.][456]{McCarthy1980a} is quoted for the second time.
  \item A sophisticated book\footcite[cf.][567]{KantKdV1974} is quoted for the first time.
  \item Now - directly following - another page of this sophisticated
  book\footcite[cf.][678]{KantKdV1974} is quoted. 
  \item Now - again directly following - the same page of this sophisticated
  book\footcite[cf.][678]{KantKdV1974} is quoted again.
  \item Now another complex book of the same
  author\footcite[cf.][789]{KantKdU1974} is quoted for the first time.
  \item And now the first sophisticated book of the same
  author\footcite[cf.][789]{KantKdV1974} is quoted again.
  \item Sometimes it's better to note the bibliographic data of a book
  (collection, proceedings) [which covers / contains articles / parts of
  different authors] not as an autonomous bitex data set but as an inline part of
  the covered article. In this case the quoted article\footcite[cf.][23]{RotCum2011a} will be mentioned as integrated set of data at both places,
  on the page quoting the article and in the bibliography.
\end{itemize}




