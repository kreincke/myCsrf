% fodina Humanities LaTeX-/JuraBib-Reference-Learn-Document
%
% (c) Karsten Reincke, Frankfurt a.M. 2010, 2011, ff.
%
% This LaTeX-File is licensed under the Creative Commons Attribution-ShareAlike
% 3.0 Germany License (http://creativecommons.org/licenses/by-sa/3.0/de/): Feel
% free 'to share (to copy, distribute and transmit)' or 'to remix (to adapt)'
% it, if you '... distribute the resulting work under the same or similar
% license to this one' and if you respect how 'you must attribute the work in
% the manner specified by the author ...':
%
% In an internet based reuse please link the reused parts to www.fodina.de and
% mention the original author Karsten Reincke in a suitable manner. In a
% paper-like reuse please insert a short hint to www.fodina.de and to the
% original author, Karsten Reincke, into your preface. For normal quotations
% please use the scientific standard to cite.
%
% select the document class
% S.26: [ 10pt|11pt|12pt onecolumn|twocolumn oneside|twoside notitlepage|titlepage final|draft
%         leqno fleqn openbib a4paper|a5paper|b5paper|letterpaper|legalpaper|executivepaper openrigth ]
% S.25: { article|report|book|letter ... }
%
% oder koma-skript S.10 + 16
\documentclass[
  DIV=calc,
  BCOR=5mm,
  11pt,
  headings=small,
  oneside,
  abstract=true,
  toc=bib,
  english,ngerman]{scrartcl}

%%% (1) general configurations %%%
\usepackage[utf8]{inputenc}

%%% (2) language specific configurations %%%
\usepackage[]{a4,babel}
\selectlanguage{ngerman}

% package for improving the grey value and the line feed handling
\usepackage{microtype}

%language specific quoting signs
\usepackage{csquotes}

% jurabib configuration
\usepackage[see]{jurabib}
\bibliographystyle{jurabib}
% mycsrf German jurabib configuration include module file 
%
% (c) Karsten Reincke, Frankfurt a.M. 2012, ff.
%
% This file is licensed under the Creative Commons Attribution 3.0 Germany
% License (http://creativecommons.org/licenses/by/3.0/de/): 
% For details see teh file LICENSE in the top directory

% the first time cite with all data, later with shorttitle
\jurabibsetup{citefull=first}

%%% (1) author / editor list configuration
%\jurabibsetup{authorformat=and} % uses 'und' instead of 'u.'
% therefore define your own abbreviated conjunction: 
% an 'and before last author explicetly written conjunction

% for authors in citations
\renewcommand*{\jbbtasep}{\ u.\ } % bta = between two authors sep
\renewcommand*{\jbbfsasep}{,\ } % bfsa = between first and second author sep
\renewcommand*{\jbbstasep}{\ u.\ }% bsta = between second and third author sep
% for editors in citations
\renewcommand*{\jbbtesep}{\ u.\ } % bta = between two authors sep
\renewcommand*{\jbbfsesep}{,\ } % bfsa = between first and second author sep
\renewcommand*{\jbbstesep}{\ u.\ }% bsta = between second and third author sep

% for authors in literature list
\renewcommand*{\bibbtasep}{\ u.\ } % bta = between two authors sep
\renewcommand*{\bibbfsasep}{,\ } % bfsa = between first and second author sep
\renewcommand*{\bibbstasep}{\ u.\ }% bsta = between second and third author sep
% for editors  in literature list
\renewcommand*{\bibbtesep}{\ u.\ } % bte = between two editors sep
\renewcommand*{\bibbfsesep}{,\ } % bfse = between first and second editor sep
\renewcommand*{\bibbstesep}{\ u.\ }% bste = between second and third editor sep

% use: name, forname, forname lastname u. forname lastname
\jurabibsetup{authorformat=firstnotreversed}
\jurabibsetup{authorformat=italic}

%%% (2) title configuration
% in every case print the title, let it be seperated from the 
% author by a colon and use the slanted font
\jurabibsetup{titleformat={all,colonsep}}
%\renewcommand*{\jbtitlefont}{\textit}

%%% (3) seperators in bib data
% separate bibliographical hints and page hints by a comma
\jurabibsetup{commabeforerest}

%%% (4) specific configuration of bibdata in quotes / footnote
% use a.a.O if possible
\jurabibsetup{ibidem=strict}
% replace ugly a.a.O. by ders., a.a.O. resp. ders., ebda.
% but if there are more than one author or girl writers?
\AddTo\bibsgerman{
  \renewcommand*{\ibidemname}{Ds.,\ a.a.O.}
  \renewcommand*{\ibidemmidname}{ds.,\ a.a.O.}
}
\renewcommand*{\samepageibidemname}{Ds.,\ ebda.}
\renewcommand*{\samepageibidemmidname}{ds.,\ ebda.}

%%% (5) specific configuration of bibdata in bibliography
% ever an in: before journal and collection/book-titles 

\renewcommand*{\bibjtsep}{in:\ }
\renewcommand*{\bibbtsep}{in:\ }

% ever a colon after author names 
\renewcommand*{\bibansep}{:\ }
% ever a semi colon after the title 
\renewcommand*{\bibatsep}{;\ }
% ever a comma before date/year
\renewcommand*{\bibbdsep}{,\ }

% let jurabib insert the S. and p. information
% no S. necessary in bib-files and in cites/footcites
\jurabibsetup{pages=format}

% use a compressed literature-list using a small line indent
\jurabibsetup{bibformat=compress}
\setlength{\jbbibhang}{1em}

% which follows the design of the cites and offers comments
\jurabibsetup{biblikecite}

% print annotations into bibliography
\jurabibsetup{annote}
\renewcommand*{\jbannoteformat}[1]{{ \itshape #1 }}

%refine the prefix of url download
\AddTo\bibsgerman{\renewcommand*{\urldatecomment}{Referenzdownload: }}

% we want to have the year of articles in brackets
\renewcommand*{\bibaldelim}{(}
\renewcommand*{\bibardelim}{)}

%Umformatierung des Reihentitels und der Reihennummer
\DeclareRobustCommand{\numberandseries}[2]{%
\unskip\unskip%,
\space\bibsnfont{(=~#2}%
\ifthenelse{\equal{#1}{}}{)}{, [Bd./Nr.]~#1)}%
}%

%Umformatierung Referenzverweises
\usepackage{xpatch}
\AfterFile{dejbbib.ldf}{%
  \xapptocmd{\bibsgerman}{%
     \def\inname{\ifjboxford in:\else\ifjbchicago in:\else in:\fi\fi}%
    \def\incollinname{\ifjboxford in:\else\ifjbchicago in:\else in:\fi\fi}%
  }{}{}%
}

% the field printed before ISBN, ISSN or URL is the bibfield note
% Hence: If you insert into the field note the type of the literature
% [ Print | [FreeWeb | BibWeb] / [ PDF | HTML ] ] then you now
% get entries like:
% Print: ISBN ....
% BibWeb / PDF => http...
% That's nice for dealing with electronic sources correctly
\DeclareRobustCommand{\jbissn}[1]{\unskip:\space ISSN #1}%
\DeclareRobustCommand{\jbisbn}[1]{\unskip:\space ISBN #1}%

\DeclareRobustCommand{\biburlprefix}{$\Rightarrow$ }
\DeclareRobustCommand{\biburlsuffix}{}



% language specific hyphenation
%mycsrfk Hyphenation Include Module text
%
% (c) Karsten Reincke, Frankfurt a.M. 2012, ff.
%
% This file is licensed under the Creative Commons Attribution 3.0 Germany
% License (http://creativecommons.org/licenses/by/3.0/de/): 
% For details see teh file LICENSE in the top directory
%


\hyphenation{ Mehr-stimmig-keit Musik-wissen-schaft-ler}



%%% (3) layout page configuration %%%

% select the visible parts of a page
% S.31: { plain|empty|headings|myheadings }
%\pagestyle{myheadings}
\pagestyle{headings}

% select the wished style of page-numbering
% S.32: { arabic,roman,Roman,alph,Alph }
\pagenumbering{arabic}
\setcounter{page}{1}

% select the wished distances using the general setlength order:
% S.34 { baselineskip| parskip | parindent }
% - general no indent for paragraphs
\setlength{\parindent}{0pt}
\setlength{\parskip}{1.2ex plus 0.2ex minus 0.2ex}


%%% (4) general package activation %%%
%\usepackage{utopia}
%\usepackage{courier}
%\usepackage{avant}
\usepackage[dvips]{epsfig}

% graphic
\usepackage{graphicx,color}
\usepackage{array}
\usepackage{shadow}
\usepackage{fancybox}

\usepackage{tikz}
\usetikzlibrary{arrows}
\usetikzlibrary{shapes,snakes}
\usetikzlibrary{positioning}
\usetikzlibrary{decorations.text}
\usetikzlibrary{trees}
\usetikzlibrary{matrix}

\usepackage{amsmath}
\usepackage{amsfonts}
\usepackage{amssymb}
\usepackage{wasysym}
\usepackage{chngcntr}

%- start(footnote-configuration)

% formatting the footnote with koma script tools

\deffootnote[1.5em]{1.5em}{1.5em}{\textsuperscript{\thefootnotemark)\ }}


% package for macking tables with broken lines
\usepackage{multirow}

\usepackage{chngcntr}

%for using label as nameref
\usepackage{nameref}

%integrate nomenclature
% mycsrf  Deutsch Nomenclation Declaration Include Module 
%
% (c) Karsten Reincke, Frankfurt a.M. 2012, ff.
%
% This file is licensed under the Creative Commons Attribution 3.0 Germany
% License (http://creativecommons.org/licenses/by/3.0/de/): 
% For details see teh file LICENSE in the top directory

\usepackage[intoc]{nomencl}
\let\abbr\nomenclature
% Deutsche Überschrift
%\renewcommand{\nomname}{Abbreviations}
\renewcommand{\nomname}{Abkürzungen}

\setlength{\nomlabelwidth}{.20\hsize}
\renewcommand{\nomlabel}[1]{#1 \dotfill}
% reduce the line distance
\setlength{\nomitemsep}{-\parsep}
\makenomenclature


% depth of contents
\setcounter{secnumdepth}{5}
\setcounter{tocdepth}{5}

% Hyperlinks
\usepackage{hyperref}
\hypersetup{bookmarks=true,breaklinks=true,colorlinks=true,citecolor=blue,draft=false}

\begin{document}

%% use all entries of the bibliography
\nocite{*}

%%-- start(titlepage)
\titlehead{scholar-fonno.de: my Classical Scholar Research framework}
\subject{(Geistes-) Wissenschaftliche Texte mit \textit{jurabib}}
\title{Dienst am Leser, Dienst am Scholaren}
\subtitle{Über Anmerkungsapparate in Fußnoten - aber richtig: Realease \input{rel.inc}}
\author{Karsten Reincke% mycsrf License Include Module
%
% (c) Karsten Reincke, Frankfurt a.M. 2012, ff.
%
% This file is licensed under the Creative Commons Attribution 3.0 Germany
% License (http://creativecommons.org/licenses/by/3.0/de/): 
% For details see teh file LICENSE in the top directory
%

\footnote{\textbf{This file is distributed under the terms of license XYZ}
Here, you can insert your conditions for using your text. Good examples
for such licenses are offered under \texttt{https://creativecommons.org/}. 
Traditionally it also possible to say : \emph{All rights reserved}.
In accordance to the license \texttt{CC BY 3.0 DE}, under which mycrsf
is released, you must finally point to mycsrf:
\newline 
{ \tiny \itshape [Format derived from \texttt{mind your Scholar Research
Framework} \copyright K. Reincke CC BY 3.0 DE http://fodina.de/mycsrf)] }}

}

\maketitle
%%-- end(titlepage)

% fodina humanities 'for being included' snippet template
%
% (c) Karsten Reincke, Frankfurt a.M. 2010, 2011, ff.
%
% This LaTeX-File is licensed under the Creative Commons Attribution-ShareAlike
% 3.0 Germany License (http://creativecommons.org/licenses/by-sa/3.0/de/): Feel
% free 'to share (to copy, distribute and transmit)' or 'to remix (to adapt)'
% it, if you '... distribute the resulting work under the same or similar
% license to this one' and if you respect how 'you must attribute the work in
% the manner specified by the author ...':
%
% In an internet based reuse please link the reused parts to www.fodina.de and
% mention the original author Karsten Reincke in a suitable manner. In a
% paper-like reuse please insert a short hint to www.fodina.de and to the
% original author, Karsten Reincke, into your preface. For normal quotations
% please use the scientific standard to cite.
%
% [ Derived from 'mykeds Classical Scholar Research Framework' 
%   mykeds-CSR-framework (c) K. Reincke CC BY 3.0  http://www.mykeds.net/ ]
%


\begin{abstract}
\noindent \itshape
Der Umgang mit Quellennachweisen hat immer auch rezeptive Auswirkungen: wenn er
gut ist, erleichtert er das lernende Lesen. Das gilt besonders für den
(alt)philologischen Anmerkungsapparat. Dieser Artikel zeigt an sich und aus sich
heraus, wozu und wie so etwas per LaTeX-Paket \emph{Jurabib} erzeugt wird. Was
immer er über Zweck, Gestalt und Abfolge von Fuß- resp. Endnoten sagt, soll er
mithin an sich selbst demonstrieren\footnote{Um es noch schärfer zu sagen:
Quellenangaben dienen mir meist nur dazu, im Text angesprochene Formen
vorzuführen. Suchen Sie also nicht nach tieferem Sinn, wo keiner ist.}.
\end{abstract}






\footnotesize
\tableofcontents
\normalsize

% fodina humanities 'for being included' snippet template
%
% (c) Karsten Reincke, Frankfurt a.M. 2010, 2011, ff.
%
% This LaTeX-File is licensed under the Creative Commons Attribution-ShareAlike
% 3.0 Germany License (http://creativecommons.org/licenses/by-sa/3.0/de/): Feel
% free 'to share (to copy, distribute and transmit)' or 'to remix (to adapt)'
% it, if you '... distribute the resulting work under the same or similar
% license to this one' and if you respect how 'you must attribute the work in
% the manner specified by the author ...':
%
% In an internet based reuse please link the reused parts to www.fodina.de and
% mention the original author Karsten Reincke in a suitable manner. In a
% paper-like reuse please insert a short hint to www.fodina.de and to the
% original author, Karsten Reincke, into your preface. For normal quotations
% please use the scientific standard to cite.
%
\section{Form Follows Function: Wie soll es aussehen?}

Seit Ewigkeiten schwärme ich von der altphilologischen Nachweis- und
Zitiermethode, vom mächtigen Anmerkungsapparat in Fußnoten - auch wenn ich
meinen Hang dazu nicht immer ausleben darf und mich stattdessen auf Endnoten
anstelle von Fußnoten beschränken muss.

Für beide Arten hatte ich mir längst schon deren bruchlose Nachbildung im
Satzsystem \emph{LaTeX} gewünscht - wohl wissend, dass dieses eben nicht aus der
europäischen Geisteswissenschaft heraus entstanden ist, sondern aus der
computerisierten Mathematik und der anglo-amerikanischen Schreibtradition, wie
sie sich im \textit{Handbook for Writers of Research
Papers}\footcite[vgl.][]{ModLanAss2009a} niederschlug. Meine Leidenschaft für
die europäische Alternative ging so weit, dass ich immer wieder einmal -
selbstverständlich stets mehr oder minder erfolglos - eigenhändig Stil- und
Bibliotheksdateien editiert habe: Das arme
\emph{natib}\footcite[vgl.][]{Daly2000a} musste ebenso dran glauben, wie das
geschundene \emph{custom-bib}\footcite[vgl.][]{Daly2007a}. Dabei waren meine
Wünsche doch so einfach:

Ich wollte Zitate weder über inline-Referenzen belegt sehen, noch sie selbst so
belegen müssen, weder über esoterische Nummernblöcke[42], noch durch kryptische
BibTEX-Keys[Daly2000a], die sich - wie doch oft genug mit eigenen Augen erfahren
- immer nur als Stolpersteine im Lesefluss erwiesen. Ich liebe den Subtext im
Anmerkungsapparat, mit dem mich Autoren über die konzentrierte Argumentation
ihrer Haupttexte hinaus lustvoll auf Nebenwegen durch die mäandernde
Forschungsgeschichte führen. Diesem schwärmenden Hin und Her bin ich verfallen.
Das möchte ich genießen - möglichst ohne großes Blättern. Und natürlich möchte
auch ich es anderen anbieten können, ohne ihnen ein permanentes Seitengefrickel
zumuten zu müssen: das Mittel der Wahl sind demnach Fußnoten\footnote{Ohne
Frage, ein solcher Anmerkungsapparat kann angeberisch wirken: \glq{}Manno, was
der alles weiß, will der etwa, dass ich das alles lese?\grq{}. Will 'der'
natürlich nicht, im Gegenteil: wenn Sie ihm glauben, brauchen Sie seine Belege
nicht zu lesen. Aber wenn Sie zweifeln, wenn Sie ihn überprüfen wollen, dann
legt er Ihnen alles offen. Oder wenn Sie zu dem einen oder anderen nebenseitigen
Punkt doch gern mehr Informationen hätten, dann finden Sie dort die Details. Wie
dem auch sei: die vom Anschein her bescheidenere Variante wäre die mit Endnoten
anstelle von Fußnoten - auch wenn der Leser dafür den Preis des
Hin-und-Her-Blätterns zahlen muß: Ohne Blättern keine Belege oder Zusatzinfos.}.

Aber egal, ob in Form von Fuß- oder Endnoten\footnote{Um die Unterschiede direkt
erkennbar zu machen, habe ich diesen Text in zwei Versionen erzeugt, einmal
konfiguriert für einen Fußnotenapparat (fodinaClassicalScholar\ldots{}pdf),
einmal für einen Endnotenapparat (fodinaHumanities\ldots{}pdf). Diese
Unterschiede zu kennen, ist auch insofern wichtig, als man seinen Text
nachträgliche leider nur dann rein konfigurativ umstellen kann, wenn eine
Kleinigkeit vorab beachtet: nur $\backslash$footcite\{\ldots\} aus jurabib
verwenden, insbesondere keine (mehrfachen) $\backslash$cite\{\ldots\} innerhalb
von $\backslash$footnote\{\ldots\}, das verträgt bei einer Ersetzung von
\textit{footnote} durch \textit{endnote} $\backslash$endnote\{\ldots\} nicht.},
es ist der Anmerkungsapparat und seine expliziten bibliographischen Angaben, die
mir den Forschungskontext aufspannen. Schärfer noch: Ich möchte, dass eine
Quelle beim ersten Nachweis bibliographisch vollständig aufgeschlüsselt
wird\footnote{\cite[vgl.][195ff]{RueStaFra1980a} - dieses Werk bewahre ich aus
zwei Gründen auf: Zum einen bietet es auf den genannten Seiten eine griffige
Zusammenfassung der Regeln. Zum anderen zeigt der Rest der Seiten, wie man es
auf keinen Fall machen darf, wenn man spannende Wissenschaftslektüre schreiben
möchte: Pädagogisierte Texte sind nicht per se schon kunden- oder
leserorientiert. }. Ich möchte neben allen Angaben zu allen Autoren und allen
Titelfeinheiten auch die Auflage erfahren können, auf Übersetzungen und Reihen
hingewiesen werden und editorische Sonderfälle erkennen können\footcite[wie z.B.
bei][]{Covey2006a}. Und ja, ich stehe der Moderne mitnichten ablehnend
gegenüber: mittlerweile erleichtert die Angabe der ISBN das Wiederfinden des
richtigen Buches erheblich. Also wünsche ich mir auch deren Nennung\footnote{Als
Web-Mensch kann ich mich natürlich einem weiteren Zugeständnis von
\textit{jurabib} an die Moderne nicht verschließen: Wir müssen heute auch
Netz-Dokumente über URLs zitieren, wohl wissend, dass diese unter der Hand
geändert werden können. Mit den neuen BibTeX-Feldern 'url' und 'urldate' bietet
\textit{jurabib} die Möglichkeit, auf solche Dokumente Bezug zu nehmen, und zwar
nicht nur unter der Angabe der URL (Unified Resource Locator), sondern auch
unter Angabe des Abrufdatums\ldots was letztlich nicht mehr besagt, als dass der
jeweilige Autor um die Volatilität seiner Quellen weiß. Ich selbst gehe sogar
noch weiter: ich vermerke im BibTeX-Feld 'note', ob ich ein Werk effektiv in der
Hand gehabt habe (\textit{Print}), ob ich es im bibliothekseigenen Netz als PDF
etc. eingesehen habe (\textit{BibWeb/PDF}) oder ob ich es dem ganz
'unbeständigen' Internet entnommen habe (\textit{FreeWeb/PDF} oder
\textit{FreeWeb/HTML} oder sonstiges)}. Erst wenn im Laufe der Argumentation
erneut auf dieselbe Quelle zurückgegriffen wird\footcite[vgl.
dazu][194]{RueStaFra1980a} reicht mir ein verkürzter Beleg\footcite[wie z.B.
jetzt wieder bei][]{Covey2006a}, der mich zwanglos auf die mnemotechnisch
richtige Bahn bringt, ohne mir eine verklausulierte Geheimsprache aufzunötigen:
denn \emph{kurz} meint schließlich nicht \emph{esoterisch}\footnote{Man sieht,
dass es funktioniert: Den \emph{R\"uckriem/Stary/Franck} habe ich gerade
bibliographisch ebenso ausführlich ausgewiesen, wie den \emph{Covey}.
Anschließend habe ich erneut auf beide Quellen verwiesen, nun aber über die
Kurzform \emph{Autor: Titel, Jahr}. Dies alles ermöglicht das Heilmittel
\emph{Jurabib}( \cite[vgl. dazu][]{Berger2004a}). An einer Stelle blieb jedoch
ein Desiderat, das ich auf spezielle Weise umgehen musste: In den Kurzverweisen
wünsche ich mir neben Autor und Kurztitel immer auch das Jahr. Dies erleichtert
mir die Rezeption der Forschungs\emph{geschichte}. Jurabib bietet dieses
Schmankerl (noch) nicht an und liefert stattdessen nur den \emph{(short)author}
und den \emph{(short)title}. Allerdings kostet es ja nicht viel, an den eh
manuell zu erstellenden Kurztitel im Bibtex-Feld einfach auch noch das
Erscheinungsjahr anzuhängen. Schon hat man/ich, was man/ich will.}.

Immerhin: Wenn der Autor und ich uns im Haupttext auf eine verfeinerte Analyse
fremder Gedankengängen einlassen\footcite[vgl. etwa][32]{Allen2001a} und wenn
wir dabei mehrfach\footcite[vgl.][139]{Allen2001a} verschiedene
Passagen\footcite[vgl.][139]{Allen2001a} desselben Werkes referieren bzw.
rezipieren müssen, ja dann reichen die kleinen Hinweise, dass es sich wieder um
dieselbe Passage oder wieder um dasselbe Werk - wenn auch mit anderer Passage -
oder wieder um denselben Autor handelt: Nehmen wir an, wir zitierten zunächst aus
der Kritik der Urteilskraft\footcite[vgl.][9]{KantKdU1974} und anschließend aus
der Einleitung der Kritik der reinen Vernunft\footcite[vgl.][45]{KantKdV1974},
um sofort danach erst auf den 1. Absatz aus dem Kapitel \emph{Transzendentale
Ästhetik} verweisen zu müssen\footcite[vgl.][69]{KantKdV1974}, bevor auf den 3
Absatz derselben Seite eingehen können\footcite[vgl.][69]{KantKdV1974}, dann
müßte unser Anmerkungsapparat zunächst die beiden kompletten Belege auflisten,
gefolgt von einem \emph{ders., a.a.O. + neue Seite} wiederum gefolgt von einem
\emph{ders., ebda} erzeugen. Gingen wir nun zurück auf die \emph{Kritik der
Urteilskraft}\footcite[vgl.][9]{KantKdU1974}, dürfte unser Apparat sicher nicht
mehr mit dem Kürzel \emph{ders., a.a.O} arbeiten, sondern könnte allenfalls das
Kürzel \emph{ders.: Titel} anbieten oder sollte wenigstens - wie hier dann auch
geschehen - über den erneut genannten Autor + Kurztitel + Jahr einen neuen
expliziten Aufsatzpunkt verwenden\footnote{Wenn Sie der Demonstration gefolgt
sind, werden Sie sofort gesehen haben, dass in den wiederaufnehmenden
Anmerkungen eben nicht 'ders.' steht, sondern 'ds.' - was selbstverständlich
(vorerst noch) keine korrekte Abkürzung ist. Der Grund für diese 'Ersetzung' ist
einfach: Manchmal haben Werke mehrere Autoren. Dann dürfte im Text nicht
{\itshape ders.} erscheinen, sondern der Plural {\itshape dies.} Und oft genug
sind Bücher ja auch von Frauen geschrieben. In unseren heutigen Zeiten darf das
weibliche Geschlecht aber nicht mehr unter dem grammatischen 'bloß' mitgedacht
werden; Autorinnen können und werden zurecht auf einem {\itshape dies.}
bestehen, wenn schon ein {\itshape ders.} benutzt wird. Aus diesem Dilemma
könnte uns nur ein LaTeX befreien, dass aus der vorhergehenden Fußnote und dem
Original den Plural oder das Geschlecht der Autoren ableitet. Kurz:
Flektionsbewusstsein wäre nötig. LaTeX bietet das m.W. nicht. Also habe ich zu
einer anderen Lösung gegriffen, ich habe mir eine kollektive Abkürzung für die
einzelnen Abkürzungen definiert: {\itshape ds.} stehe für {\itshape ders.} und
{\itshape dies.}, so, wie {\itshape dies.} selbst schon für {\itshape dieselbe}
und {\itshape dieselben} steht.

Jurabib bietet von sich aus eine andere Lösung aus dem Dilemma: erst lässt die
Referenz auf den/die Autor(in/en) schlicht weg und verwendet anstelle von
{\itshape ders., a.a.O. + neue Seite} resp. {\itshape ders., ebda} schlicht
{a.a.O. + neue Seite} resp. {\itshape a.a.O.}. Das kann man machen. Ich
persönlich bevorzuge das durchgehaltene Muster 'Autor' : 'Buch', es erleichtert
mir das Lesen.}.

Auch die innere Struktur der bibliographischen Angaben wollte ich auf diesen Stil
ausgerichtet sehen. So wünschte ich mir bei kollektiv erarbeiteten Werken, dass
zwar der erste Autor mit dem Nachnamen zuerst genannt wird, dass alle folgenden
Autoren jedoch "`in natürlicher Weise"' erwähnt werden, also zuerst mit den
Vornamen und dann mit den Nachnamen\footnote{ wie \cite[hier bei:][]{Woods1991a}
oder \cite[hier bei:][]{RusNor2004a} oder ...} Außerdem wollte ich den jeweils
letzten Autor durch eine Konjunktion eingebunden sehen\footcite[... hier
bei:][]{SegEvaTay2009a}

Doch nicht nur Bücher, sondern auch Artikel\footcite[vgl. z.B.][]{Hays1985a}
sollten diesem Muster folgen, sei es solche aus Sammlungen\footcite[s.
etwa][]{Brachman1985a} oder solche aus Zeitschriften\footcite[s.
etwa][]{McCarthy1980a}: Immer wollte ich deren Form bruchlos als Erst- oder
Wiederholungszitat erkennen können, beim Sammlungsartikel\footcite[vgl.
erneut][]{Hays1985a} genauso, wie beim Zeitschriftenartikel\footcite[s.
nochmals][]{McCarthy1980a}. Und ich wollte dieses Muster auch dann so haben,
wenn es ökonomischer wäre, die bibliographischen Daten einer enthaltenden
Sammlung nicht als gesonderten Eintrag im Bibtex-File aufzunehmen, sondern sie -
sozusagen inline - in denen des enthaltenen Artikels einzubauen: Auch dann
sollte erst die Langversion erscheinen\footcite[vgl.][23]{RotCum2011a} und beim
erneuten Zitat die Kurzversion, sei nach der
'ds.ebda'-Mimik\footcite[vgl.][23]{RotCum2011a} oder - nach einem dazwischen
geschobenen anderen Zitat\footcite[vgl.][9]{KantKdU1974} wieder mit der
Kurztitelform\footcite[vgl.][23]{RotCum2011a}. Nur in der Bibliographie sollte
die Sammlung eben nicht als eigener Eintrag erscheinen, sondern mit dem Artikel
so verwoben sein, wie er auch im Text erschiene.



% fodina humanities 'for being included' snippet template
%
% (c) Karsten Reincke, Frankfurt a.M. 2010, 2011, ff.
%
% This LaTeX-File is licensed under the Creative Commons Attribution-ShareAlike
% 3.0 Germany License (http://creativecommons.org/licenses/by-sa/3.0/de/): Feel
% free 'to share (to copy, distribute and transmit)' or 'to remix (to adapt)'
% it, if you '... distribute the resulting work under the same or similar
% license to this one' and if you respect how 'you must attribute the work in
% the manner specified by the author ...':
%
% In an internet based reuse please link the reused parts to www.fodina.de and
% mention the original author Karsten Reincke in a suitable manner. In a
% paper-like reuse please insert a short hint to www.fodina.de and to the
% original author, Karsten Reincke, into your preface. For normal quotations
% please use the scientific standard to cite.


%% use all entries of the bibliography
%\nocite{*}

\section{'Form Follows Function' - oder: Wozu das Ganze?}

Offensichtlich habe ich genaue Vorstellungen von dem, wie Forschungsliteratur
für mich aussehen sollte. Und ich wünsche mir dieses Aussehen, damit ich es
bequem habe: es möge mir die Rezeption erleichtern, die der Argumentation und
die der Forschungsgeschichte; und ja: dieses Erscheinungsbild soll mir
insbesondere auch die Verifikation des Inhalts erleichtern. Autoren sind immer
auch Zuarbeiter für mich, gute Autoren sogar Diener.

Allerdings schweben noch ungestellten Frage über uns: Ist diese Form die
einzige? Und wenn sie nicht einmal die vorherrschende ist, warum bevorzugen wir
sie trotzdem? Und wozu zitieren wir überhaupt?

Beginnen wir mit dem einfachen. Ich sehe vier Gründe, Aussagen anderer zu
zitieren\footcite[vgl. dazu auch][187. Die Autoren beschreiben die
Funktionen ähnlich, legen aber andere Schwer\-punk\-te: So läuft
das, was ich als affirmatives Zitat bezeichnen, bei ihnen als
'Bestätigung wissenschaftlicher Thesen durch anerkannte
Autoritäten oder Arbeiten', während das, was ich als
'konfrontatives Zitat' bezeichne, bei Ihnen nicht vorkommt]{RueStaFra1980a}:

\begin{enumerate}
  \item Jemand anderes hat einen Befund geliefert, dessen Wahrheit, Gültigkeit
  oder Relevanz ich in Zukunft voraussetze. Ich referiere diesen Befund über
  \emph{affirmative Zitate} und baue meine Argumentation darauf auf. Und ich
  belege diese Zitate, damit ich den Schritt weg von der blossen Behauptung hin
  zum verifizierbaren Argument tue.
  \item Jemand anderes hat einen Befund geliefert, dessen Wahrheit oder
  Gültigkeit ich bestreiten will. Über \emph{konfrontative Zitate} referiere und
  widerlege ich diesen Befund. Und ich belege diese Zitate, damit meine
  Argumentation überprüfbar wird.
  \item Jemand anderes hat einen Begriff oder ein Wort benutzt, das ich
  übernehmen will. Damit delegiere ich die Arbeit der Definition an diesen
  anderen und referiere seine Ergebnisse über \emph{adaptive Zitate}. Und
  natürlich ich belege diese, um bei Rückfragen zu 'unterschlagenen' Details auf
  den eigentlichen Schöpfer verweisen zu können.
  \item Ich gebe - grosso modi - Hinweise auf konkurrierende Positionen, andere
  Aspekte oder erweiterte Kontexte. Und ich belege diese Hinweise über
  \emph{abweisende Zitate}, um überprüfbar zu machen, ob diese Positionen,
  Aspekte und Kontexte wirklich 'abseitig' sind. Denn genau das habe ich ja
  dadurch getan, dass ich nur grosso modi auf sie verwiesen habe.
\end{enumerate}

Man sieht\footcite[vgl. dazu auch][\nopage hier werden im Abschnitt
'Wissenschaft' drei Funktionen aufgelistet. Das, was ich als 'affirmatives
Zitat' bezeichne, läuft - unter dem Schlagwort 'auf den Schultern von Riesen' -
als Redundanzreduktion, gepaart mit der Überprüfbarkeit. Außerden wird die Moral
ins Feld geführt]{Wikipedia2011a}: es ist bei jeder dieser Zitatfunktionen in
meinem ureigenen Interesse, meine Quellen nicht nur 'irgendwie' anzugeben,
sondern sie leicht wiederfindbar zu machen - womit ich es meinen Lesern
zusätzlich und unter der Hand auch bequem mache, ja ihnen diene. Erschwere ich
es ihnen hingegen, schwäche ich meine Argumentation, schwäche ich mich. Denn
dann könnten sie bestenfalls über mich sagen: 'nun gut, er hat es zumindest
behauptet, aber ob's stimmt, wer weiß? - wir konnten es jedenfalls nicht
wirklich gut nachprüfen'.

Den Funktionen des Zitats stehen seine inhaltlichen Formen gegenüber. Meist
unterscheidet man zwischen \glqq{}wörtlichem\grqq{} oder
\glqq{}nicht-wörtlichem\grqq{} Zitat und meint damit die wortgetreue bzw. die
\glqq{}[\ldots] sinngemäße Übernahme oder Wiedergabe schriftlicher oder
mündlicher Äußerungen anderer\grqq{}\footcite[vgl.][187f - ohne Frage, dieses
ist ein sinngemäßes und kein wörtliches Zitat. Und es ist
affirmativ]{RueStaFra1980a}. Aus der Schulzeit kenne ich diese Formen noch als
\emph{direktes} bzw. \emph{indirektes Zitat}; so werden sie noch heute im Netz
spezifiziert\footcite[vgl.][\nopage]{WisArbOrgZitate}. Persönlich würde ich hier
noch feiner unterscheiden und diesen beiden Formen das \emph{begriffliche Zitat}
zur Seite stellen:
\begin{description}
  \item[direktes Zitat] :- die wort- und zeichengetreue Wiedergabe (mindestens)
  eines Satzes (Aussage), ggfls. durch markierte Auslassungen 'konzentriert'.
  Der Zitator erhebt den Anspruch, exakt wiedergegeben zu
  haben\footcite[vgl.][187f]{RueStaFra1980a}. Auf die Quelle wird am Ende des in
  Anführungszeichen eingeschlossenen Textes direkt verwiesen, also ohne
  modifizierende Partikel wie {\itshape vgl.}, {\itshape s.}, {\itshape ähnlich}
  etc.
  \item[indirektes Zitat] :- eine Paraphrase, die (mindestens) einen Satz
  (Aussage) sinngemäß wiedergibt. Sie darf einzelne Termini oder Satzteile aus
  dem Original entnehmen, sofern sie diese mit Anführungszeichen markiert. Auf
  die Quelle wird am Ende der Paraphrase mit Hilfe von {\itshape vgl.}
  verwiesen. Dieses signalisiert den Anspruch des Zitators, die Aussage als
  Ganzes sinngemäß, aber nicht wörtlich wiedergegeben zu
  haben\footcite[vgl.][\nopage letzter Absatz aus Abschnitt 'Grenzen der
  Zitierpflicht']{Wikipedia2011a}. \item[begriffliche Zitat] :- die wort- und
  zeichengetreue Übernahme eines Wortes bzw. einer Satzkonstituente als ein
  Begriff. Dieser übernommene Begriff wird in Anführungszeichen gesetzt, auf die
  Quelle wird unmittelbar nach dem Wort mit Hilfe von {\itshape vgl.} verwiesen.
  Der Zitator beansprucht damit, die Definition von jemand anderem übernommen zu
  haben, die Aussage, in die das Übernommene eingebettet ist, aber selbst zu
  verantworten.
\end{description}

Damit können wir die einfache Frage stellen, welche Zitatformen für welche
Zitatfunktionen dienlich sind. Und aus der antwortenden Tabelle folgt
unmittelbar, dass die Art wissenschaftlicher Texte, wie wir sie uns wünschen,
dem objektiven Sinn des Ganzen gerecht wird:

\begin{center}
\begin{tabular}{|r||c|c|c|}
\hline
& {Direktes Zitat}
& {Indirektes Zitat}
& {Begriffliches Zitat}
\\
\hline \hline 
\emph{affirmative Zitate}& \checkmark &  \checkmark & \\
\hline 
\emph{konfrontative Zitate}&  \checkmark &  \checkmark & \\
\hline
\emph{adaptive Zitate}&  & \checkmark & \checkmark\\ 
\hline
\emph{abweisende Zitate}&  & \checkmark & \\
\hline
\end{tabular}
\end{center}

Bliebe zusätzlich noch zu fragen, ob dies der einzige 'wissenschaftliche
(Schreib)\-Stil' ist. Die Antwort ist leicht zu erraten: ist er natürlich nicht.

Der größte Unterschied dürfte einem begegnen, wenn man natur- oder
geisteswissenschaftliche Forschungstexte liest, die dem anglo-amerikanischen
Forschungsraum verpflichtet sind. Einer der führenden Rat- und Richtungsgeber
dafür ist sicherlich das 'MLA Handbook for Writers of Research Papers', das sich
selbst als 'The Authorative Guide' bezeichnet
\footcite[vgl.][\nopage Buchcover]{ModLanAss2009a}. Es bietet - neben vielem
anderen - auch eine klare Argumentation: Zunächst erläutert es, was Plagiate
sind und was sie für die Forschung
bedeuten\footcite[vgl.][52ff]{ModLanAss2009a}, sodann erklärt es, wie
die Zitattexte korrekt erstellt werden\footcite[vgl.][92ff]{ModLanAss2009a}, um
anschließend die Form des zugehörigen
Beleges\footcite[vgl.][126ff]{ModLanAss2009a} und die dafür konstitutive
\glqq{}List of Works Cited\grqq{}, die Literaturliste zu
beschreiben\footcite[vgl.][126ff]{ModLanAss2009a}:

Beeindruckend ist der Anspruch, den das MLA Handbuch formuliert:

\begin{quote}\glqq{}They [the responsible writers; KR] specify when they
refer to another author's ideas, facts, and words, whether they want to
agree with, object to, or analyze the source. This kind of documentation
not only recognizes the work writers do; it also tends to discourage the
circulation of error, by inviting readers to determine for themselves
wether a reference to another text presents a reasonable account of what
the text says.\grqq{}\footcite[][52]{ModLanAss2009a}
\end{quote}

Zentral ist, dass Leser dazu \textit{eingeladen} (und nicht: daran gehindert)
werden sollen, Aussagen anderer Autoren, die im gerade gelesenen Text zitiert
oder paraphrasiert worden sind, \textit{eigenhändig zu überprüfen}, und zwar
nicht nur, ob sie korrekt wiedergegeben sind (das ist 'nur' eine notwendige
Voraussetzung), sondern ob sie in die Argumentation \textit{valide eingebunden
sind} und diese stützen. Und zu dieser Forderung an Autoren konstatiert das
Handbuch schlicht:

\begin{quote} \glqq{}Plagiarists undermine these important public value.
Once detected, plagiarism in a work provokes skepticism and even outrage
among readers, whose trust in the author has been
broken.\grqq{}\footcite[][52f]{ModLanAss2009a}
\end{quote}

Ein solcher Schaden - so das Handbuch - entstehe sogar durch 'unbeabsichtigte
Plagiate'\footcite[vgl.][55 - im Original \glqq{}unintenional
plagiarism\grqq{}]{ModLanAss2009a}. Und diese können leichter 'entstehen', als
der unbedarfte Autor anzunehmen geneigt ist. Denn sol gelte z.B.:

\begin{quote}\glqq{}Presenting an author's wording without marking it as
quotation is plagiarism, even if you cite the source.\grqq{}\footcite[][55
(herv.KR.)]{ModLanAss2009a}
\end{quote}

Warum diese Kleinigkeitskrämerei? Weil es zum Wesen der Wissenschaft gehöre, an
Vorarbeiten anzuknüpfen. Der Zweck eines Forschungspapieres \glqq{}[\ldots] is
to synthesize previous research and scholarship with your ideas on the
subject\grqq{}. Und wenn das 'Borgen' intentional schon dazugehöre, dann dürfe
\glqq{}[\ldots] the material you borrow [\ldots] not be presented as if it were
your own creation\grqq{}\footcite[vgl.][55]{ModLanAss2009a}. Klar, dass das
unmarkierte Zitat diese Regel verletzt. Denn wie sollte aus der bloßen
Quellenangabe geschlossen werden können, welche Wörter übernommen und welche
eigene Zutat sind? Eigentlich also kaum noch erwähnenswert, weil implizit
unabdingbar, ist dann noch die folgende ergänzende Regel

\begin{quote} \glqq{}[\ldots] you must document everything that you borrow - not
only direct quotations and paraphrases but also information and
ideas.\grqq{}\footcite[][52f]{ModLanAss2009a}
\end{quote}

Man sieht unmittelbar, dass Anspruch des MLA Handbuches und meine Wünsche an
wissenschaftliche Texte kaum divergieren. Wenn ein Unterschied besteht, dann
also in der Form. Und in der Tat gibt es eine zentrale Anweisung des MLA
Handbuches, deren Interpretation zu gravierenden Differenzen im Erscheinungsbild
führen:

\begin{quote} \glqq{}[\ldots] A citation in MLA style contains only enough
information to enable the readers to find the source in the works-cited
list.\grqq{}\footcite[][127]{ModLanAss2009a}
\end{quote}

Das hat radikale Konsequenzen: Wenn man in seinem 'Erzähltext' beispielsweise
den Namen des Autors erwähne und von diesem nur ein Werk zitiere, dann reiche es
aus, im Erzähltext nach dem Zitat die bloße Seitenzahl anzugeben. Erst wenn es
mehrere Werke seien, müsse zusätzlich zur Seitenzahl ein so gekürzter Titel im
laufenden Text eingefügt werden, dass man das Werk in der Literaturliste
wiederfinde\footcite[vgl.][127]{ModLanAss2009a}. Fairerweise erwähnt das
Handbuch, dass der \glqq{}[\ldots] MLA is not the only way to document
sources\grqq{}\footcite[vgl.][127]{ModLanAss2009a}. Eine Alternative sei der
'APA style', bei dem im laufenden Text Autor, Jahr und Seitenzahl angegeben
werden und als Muster in das Literaturverzeichnis
verweisen\footcite[vgl.][127f]{ModLanAss2009a}.

Dieser MLA-Stil ist konsequent minimalistisch. Und er erfüllt die selbst
gesetzten Ziele. Trotzdem werde ich nicht mit ihm warm:

\begin{itemize}
  \item Zum ersten wird der Lesefluss, das 'gleichmäßige' Gleiten des Blickes
  über die Zeilen durch meistenteils eben doch längliche Zitatbelege
  unterbrochen.
  \item Zum zweiten muss ich mir die Informationen aus dem Kontext
  'zusammenklauben', wenn ich ein Zitat überprüfen will. Wo stand noch gleich
  der Autorname? Welche Seitenangabe bezog sich jetzt grad noch auf sein Werk?
  Mich lädt diese Art nicht ein, das Vorgetragene zu überprüfen.
  \item Und zum dritten und entscheidenden: In diesem Stil können mir Autoren
  nicht nebenbei die Forschungssgeschichte vermitteln. Das geht einfach nicht,
  eben weil der Stil auf Minimalismus ausgelegt ist, der die
  forschungsgeschichtlichen Zusatzhinweise und Markanten wie Verlag, Auflage
  oder Jahr, wie Name der Zeitschrift oder der Serie etc. etc. einfach beiseite
  lassen muss.
\end{itemize}

Zuerst ein 'gutes', weil korrektes Beispiel, das diesen Stil verdeutlicht: Wenn
man den schon zitierten Artikel 'Intellectualism as Cognitive Science' liest,
findet man genau dieses unruhige Lese-Bild, dass zwar korrekt ist, aber stolpern
lässt: Werke werden im Text innerhalb von Klammern nach dem 'Schema Autoren,
Jahr, Seite' zitiert\footcite[vgl.][25]{RotCum2011a}. Und man stolpert gleich
zweifach: zum ersten unterbrechen die Klammern den Lesefluss. Und dann muss man
auch noch auf die Bibliographieseiten\footcite[vgl.][38f]{RotCum2011a} blättern,
um den Titel des Werkes und damit die engste Zusammenfassung des Inhalts
kennenzulernen. Noch unangenehmer wird dieser Stil jedoch, wenn der Verfasser
Stellen in Werken nur noch über 'Autor und Jahr' referieren und auf Seitenzahlen
ganz verzichten. Damit wäre die Überprüfbarkeit nicht nur stilistisch erschwert,
sondern gänzlich verloren gegangen\footcite[vgl. z.B.][151]{Bechtel2011a}.

Zugegeben, dieser Stil behandelt seine Leser auf Augenhöhe. Er geht unter der
Hand davon aus, dass dem Leser die zitierten Werke im Prinzip bekannt sind. Bei
dieser minimalisierten Art zu zitierten, diskutiert der ausgewiesene Experte
'Autor' mit dem Experten 'Leser'.

Darin liegt allerdings auch eine gehörige Portion Arroganz: Nur der Experte ist
der intendierte Adressat, einfachere, un\-(aus)\-ge\-bil\-de\-te\-re Leser
müssen erst noch zu Experten werden. Ehrlich, da lese ich doch lieber Werke, die
mir auf den ersten Blick etwas angeberisch erscheinen, die mir aber in der
Praxis sehr viel leichter ein Thema erschließen, und zwar gründlich.


%\bibliography{../bib/literature}


%fodina humanitied 'for being included' snippet template
%
% (c) Karsten Reincke, Frankfurt a.M. 2010, 2011, ff.
%
% This LaTeX-File is licensed under the Creative Commons Attribution-ShareAlike
% 3.0 Germany License (http://creativecommons.org/licenses/by-sa/3.0/de/): Feel
% free 'to share (to copy, distribute and transmit)' or 'to remix (to adapt)'
% it, if you '... distribute the resulting work under the same or similar
% license to this one' and if you respect how 'you must attribute the work in
% the manner specified by the author ...':
%
% In an internet based reuse please link the reused parts to www.fodina.de and
% mention the original author Karsten Reincke in a suitable manner. In a
% paper-like reuse please insert a short hint to www.fodina.de and to the
% original author, Karsten Reincke, into your preface. For normal quotations
% please use the scientific standard to cite.
%


%% use all entries of the bibliography
%\nocite{*}
\section{Function Supports Form: Wie geht das mit LaTeX?}

Mein Hauptwunsch aber war und ist, dass der altphilologisch
geisteswissenschaftliche Schreib- und Argumentierstil LaTeX-like ermöglicht
wird: ein simpler Befehl für den Zitatbeleg, und der Rest sollte sich von allein
ergeben: gerne über Zuladung von Paketen gesteuert, über Konfigurationen
verfeinert und mittels BibTeX\footcite[vgl.][]{BibtexOrgDe} erst ermöglicht.
Aber bitte ohne das wiederholte lästige Tippen all dieser fitzligen
Kleinigkeiten eines bibliographischen Nachweises.

Glücklicherweise gibt es mittlerweile die kooperierenden Retter
\emph{Jurabib}\footcite[vgl.][]{Berger2004a} und
\emph{KOMA-Script}\footcite[vgl.][]{Kohm2008a}: Der erste bewahrt die
altphilologische Zitierweise (selbst wenn er - wie sein Name unterstreicht - zu
anderem Zwecke geschaffen worden ist); der zweite sorgt für einen europäisch
klassischen Seitenspiegel. Ich hätte meine Selbstversuche viel früher einstellen
und erneut auf die Suche nach fertigen Paketen gehen sollen. Dann wäre mir diese
Lösung gewiss schon eher über den Weg gelaufen. Doch wie las ich neulich doch so
schön: \emph{Es ist nie zu spät für alles}\footnote{Das ist der Titel eines
Buches von Kajsa Ingemarsson, das ich nicht gelesen habe und darum auch nicht
zitieren darf}.

Und was heißt das nun konkret? Nun, das Ergebnis zeigt dieser spezielle
"`Blindtext"' an sich und aus sich selbst heraus. Die verwendete Technik möge
direkt dem LaTeX-Quelltext entnommen werden. Hier einige zentralen Punkte, wie
sie im LaTeX-File erscheinen:

Zuerst wird die KOMA-Dokumentenklasse als Basis festgesetzt, das Ganze auf
Deutsch mit utf8 Input ausgerichtet und der Absatz ohne Einrückungen aber mit
Abstand formatiert. Danach wird das Jurabib-Paket aktiviert und die ausgelagerte
Jurabib-Konfigurationsdatei hinzugeladen. Danach beginnt das eigentliche
Dokument, an dessen Ende auf die entsprechende Bibliographie referiert wird.
\small
\begin{verbatim}
\documentclass[DIV=calc,BCOR=5mm,11pt,smallheadings,oneside,
                                 abstract=true,toc=bib]{scrartcl}
\usepackage[utf8]{inputenc}
\usepackage[]{a4,ngerman}
\usepackage[english,ngerman]{babel}
\selectlanguage{ngerman}
\setlength{\parindent}{0pt}
\setlength{\parskip}{1.5ex plus 0.5ex minus 0.5ex}
% !JURABIB!
\usepackage[see]{jurabib}
\bibliographystyle{jurabib}
% !Hinzuladen der ausgelagerten Konfiguration
% mycsrf German jurabib configuration include module file 
%
% (c) Karsten Reincke, Frankfurt a.M. 2012, ff.
%
% This text is licensed under the Creative Commons Attribution 3.0 Germany
% License (http://creativecommons.org/licenses/by/3.0/de/): Feel free to share
% (to copy, distribute and transmit) or to remix (to adapt) it, if you respect
% how you must attribute the work in the manner specified by the author(s):
% \newline
% In an internet based reuse please link the reused parts to mycsrf.fodina.de
% and mention the original author Karsten Reincke in a suitable manner. In a
% paper-like reuse please insert a short hint to mycsrf.fodina.de and to the
% original author, Karsten Reincke, into your preface. For normal quotations
% please use the scientific standard to cite.

% the first time cite with all data, later with shorttitle
\jurabibsetup{citefull=first}

%%% (1) author / editor list configuration
%\jurabibsetup{authorformat=and} % uses 'und' instead of 'u.'
% therefore define your own abbreviated conjunction: 
% an 'and before last author explicetly written conjunction

% for authors in citations
\renewcommand*{\jbbtasep}{\ u.\ } % bta = between two authors sep
\renewcommand*{\jbbfsasep}{,\ } % bfsa = between first and second author sep
\renewcommand*{\jbbstasep}{\ u.\ }% bsta = between second and third author sep
% for editors in citations
\renewcommand*{\jbbtesep}{\ u.\ } % bta = between two authors sep
\renewcommand*{\jbbfsesep}{,\ } % bfsa = between first and second author sep
\renewcommand*{\jbbstesep}{\ u.\ }% bsta = between second and third author sep

% for authors in literature list
\renewcommand*{\bibbtasep}{\ u.\ } % bta = between two authors sep
\renewcommand*{\bibbfsasep}{,\ } % bfsa = between first and second author sep
\renewcommand*{\bibbstasep}{\ u.\ }% bsta = between second and third author sep
% for editors  in literature list
\renewcommand*{\bibbtesep}{\ u.\ } % bte = between two editors sep
\renewcommand*{\bibbfsesep}{,\ } % bfse = between first and second editor sep
\renewcommand*{\bibbstesep}{\ u.\ }% bste = between second and third editor sep

% use: name, forname, forname lastname u. forname lastname
\jurabibsetup{authorformat=firstnotreversed}
\jurabibsetup{authorformat=italic}

%%% (2) title configuration
% in every case print the title, let it be seperated from the 
% author by a colon and use the slanted font
\jurabibsetup{titleformat={all,colonsep}}
%\renewcommand*{\jbtitlefont}{\textit}

%%% (3) seperators in bib data
% separate bibliographical hints and page hints by a comma
\jurabibsetup{commabeforerest}

%%% (4) specific configuration of bibdata in quotes / footnote
% use a.a.O if possible
\jurabibsetup{ibidem=strict}
% replace ugly a.a.O. by ders., a.a.O. resp. ders., ebda.
% but if there are more than one author or girl writers?
\AddTo\bibsgerman{
  \renewcommand*{\ibidemname}{Ds.,\ a.a.O.}
  \renewcommand*{\ibidemmidname}{ds.,\ a.a.O.}
}
\renewcommand*{\samepageibidemname}{Ds.,\ ebda.}
\renewcommand*{\samepageibidemmidname}{ds.,\ ebda.}

%%% (5) specific configuration of bibdata in bibliography
% ever an in: before journal and collection/book-titles 

\renewcommand*{\bibjtsep}{in:\ }
\renewcommand*{\bibbtsep}{in:\ }

% ever a colon after author names 
\renewcommand*{\bibansep}{:\ }
% ever a semi colon after the title 
\renewcommand*{\bibatsep}{;\ }
% ever a comma before date/year
\renewcommand*{\bibbdsep}{,\ }

% let jurabib insert the S. and p. information
% no S. necessary in bib-files and in cites/footcites
\jurabibsetup{pages=format}

% use a compressed literature-list using a small line indent
\jurabibsetup{bibformat=compress}
\setlength{\jbbibhang}{1em}

% which follows the design of the cites and offers comments
\jurabibsetup{biblikecite}

% print annotations into bibliography
\jurabibsetup{annote}
\renewcommand*{\jbannoteformat}[1]{{ \itshape #1 }}

%refine the prefix of url download
\AddTo\bibsgerman{\renewcommand*{\urldatecomment}{Referenzdownload: }}

% we want to have the year of articles in brackets
\renewcommand*{\bibaldelim}{(}
\renewcommand*{\bibardelim}{)}

%Umformatierung des Reihentitels und der Reihennummer
\DeclareRobustCommand{\numberandseries}[2]{%
\unskip\unskip%,
\space\bibsnfont{(=~#2}%
\ifthenelse{\equal{#1}{}}{)}{, [Bd./Nr.]~#1)}%
}%

%Umformatierung Referenzverweises
\usepackage{xpatch}
\AfterFile{dejbbib.ldf}{%
  \xapptocmd{\bibsgerman}{%
     \def\inname{\ifjboxford in:\else\ifjbchicago in:\else in:\fi\fi}%
    \def\incollinname{\ifjboxford in:\else\ifjbchicago in:\else in:\fi\fi}%
  }{}{}%
}

% [...]
\begin{document}
% [...]
\bibliography{bibfiles/fodinaHumanitiesExDe}
\end{document}
\end{document}
\end{verbatim}

\normalsize
Und in der Jurabib-Konfigurationsdatei \emph{fodinaHumanitiesJbibCfgDeInc}
werden dann die Details festgelegt:
\small
\begin{verbatim}
% the first time cite with all data, later with shorttitle
\jurabibsetup{citefull=first}

%%% (1) author / editor list configuration
% for authors in citations
\renewcommand*{\jbbtasep}{ u. } % bta = between two authors
\renewcommand*{\jbbfsasep}{, } % bfsa = between first and second author
\renewcommand*{\jbbstasep}{ u. }% bsta = between second and third author
% for editors in citations
\renewcommand*{\jbbtesep}{ u. } % bta = between two authors
\renewcommand*{\jbbfsesep}{, } % bfsa = between first and second author
\renewcommand*{\jbbstesep}{ u. }% bsta = between second and third author

% for authors in literature list
\renewcommand*{\bibbtasep}{ u. } % bta = between two authors
\renewcommand*{\bibbfsasep}{, } % bfsa = between first and second author
\renewcommand*{\bibbstasep}{ u. }% bsta = between second and third author
% for editors  in literature list
\renewcommand*{\bibbtesep}{ u. } % bte = between two editors
\renewcommand*{\bibbfsesep}{, } % bfse = between first and second editor
\renewcommand*{\bibbstesep}{ u. }% bste = between second and third editor

% use: name, forname, forname lastname u. forname lastname
\jurabibsetup{authorformat=firstnotreversed}
\jurabibsetup{authorformat=italic}

%%% (2) title configuration
% in every case print the title, let it be seperated from the 
% author by a colon and use the slanted font
\jurabibsetup{titleformat={all,colonsep}}
%\renewcommand*{\jbtitlefont}{\textit}

%%% (3) seperators in bib data
% separate bibliographical hints and page hints by a comma
\jurabibsetup{commabeforerest}

%%% (4) specific configuration of bibdata in quotes / footnote
% use a.a.O if possible
\jurabibsetup{ibidem=strict}

% replace the ugly a.a.O. by ders., a.a.O. resp. ders., ebda.
% but if there are more than one author or girl writers?
\AddTo\bibsgerman{
  \renewcommand*{\ibidemname}{Ds., a.a.O.}
  \renewcommand*{\ibidemmidname}{ds., a.a.O.}
}
\renewcommand*{\samepageibidemname}{Ds., ebda.}
\renewcommand*{\samepageibidemmidname}{ds., ebda.}

%%% (5) specific configuration of bibdata in bibliography
% ever an in: before journal and collection/book-tiltes 
\renewcommand*{\bibbtsep}{in: }
%\renewcommand*{\bibjtsep}{in: }

% ever a colon after author names 
\renewcommand*{\bibansep}{: }
% ever a semi colon after the title 
\renewcommand*{\bibatsep}{; }
% ever a comma before date/year
\renewcommand*{\bibbdsep}{, }

% let jurabib insert the S. and p. information
% no S. necessary in bib-files and in cites/footcites
\jurabibsetup{pages=format}

% use a compressed literature-list using a small line indent
\jurabibsetup{bibformat=compress}
\setlength{\jbbibhang}{1em}

% which follows the design of the cites and offers comments
\jurabibsetup{biblikecite}

\AddTo\bibsgerman{\renewcommand*{\urldatecomment}
                                       {Referenzdownload: }}

% we want to have the year of articles in brackets
\renewcommand*{\bibaldelim}{(}
\renewcommand*{\bibardelim}{)}

% print annotations into bibliography
\jurabibsetup{annote}
\renewcommand*{\jbannoteformat}[1]{{ \itshape #1 }}

\end{verbatim}
\normalsize

%fodina humanitied 'for being included' snippet template
%
% (c) Karsten Reincke, Frankfurt a.M. 2010, 2011, ff.
%
% This LaTeX-File is licensed under the Creative Commons Attribution-ShareAlike
% 3.0 Germany License (http://creativecommons.org/licenses/by-sa/3.0/de/): Feel
% free 'to share (to copy, distribute and transmit)' or 'to remix (to adapt)'
% it, if you '... distribute the resulting work under the same or similar
% license to this one' and if you respect how 'you must attribute the work in
% the manner specified by the author ...':
%
% In an internet based reuse please link the reused parts to www.fodina.de and
% mention the original author Karsten Reincke in a suitable manner. In a
% paper-like reuse please insert a short hint to www.fodina.de and to the
% original author, Karsten Reincke, into your preface. For normal quotations
% please use the scientific standard to cite.
%
%% use all entries of the bibliography
%\nocite{*}
\section{Support Forms Wishes - oder: Geht es noch besser?}
Ist die Welt jetzt in Ordnung? Nun, ein paar Kleinigkeiten fehlen mir
eigentlich noch - jedenfalls, wenn ich ganz pingelig bin:

So sähe ich zunächst bei der initialen Nennung eines Sammlungs- oder
Zeitschriftenartikels schon im Anmerkungsapparat gern auch die begrenzenden
Seitenzahlen, ganz wie im zum Literaturverzeichnis. Die konkret
intendierte Belegseite könnte in diesen (und nur in diesen) Fällen einfach nach
dem Muster {\itshape XYZ. In. ZYX, S. 24-42, {\bfseries hier S. 28}} angehängt.

Desgleichen würde ich gerne im initialen Zitat gern auch die Sammlung, die einen
Artikel enthält, mit all ihren Angaben abgedruckt sehen, zumindest, wenn sie
selbst das erste Mal genannt wird.

Zudem würde ich LaTeX natürlich gerne 'flektierend kontextsensitiv'
sehen, sodass meine neue Abkürzung {\itshape ds.} überflüssig würde.

Ferner würde ich Sammlungen, die nur Herausgeber haben, einzig über ihre Titel
mit angehängtem {\itshape hrsg. v.} eingeordnet sehen. Ginge das grundsätzlich
nicht, wünschte ich mir im initialen Quellennachweis für einen Sammlungsartikel,
dass die Herausgeber auch als Herausgeber ausgewiesen werden \footfullcite[wie
hier eben nicht geschehen:][]{Hays1985a}

Und schließlich wünschte ich mir, auch innerhalb einer Endnote mit dem normalen
$\backslash$cite auch Inline-Verweise innerhalb dieser Endnote generieren zu
können. Bei $\backslash$footnote ist das möglich, bei $\backslash$endnote leider
nicht.

Aber zugegeben - besonders relevant ist das alles nicht, vielleicht wäre
letztlich sogar störend. Sollte ich also meine Wünsche überdenken?

%\bibliography{../bib/literature}


% fodina humanitied 'for being included' snippet template
%
% (c) Karsten Reincke, Frankfurt a.M. 2010, 2011, ff.
%
% This LaTeX-File is licensed under the Creative Commons Attribution-ShareAlike
% 3.0 Germany License (http://creativecommons.org/licenses/by-sa/3.0/de/): Feel
% free 'to share (to copy, distribute and transmit)' or 'to remix (to adapt)'
% it, if you '... distribute the resulting work under the same or similar
% license to this one' and if you respect how 'you must attribute the work in
% the manner specified by the author ...':
%
% In an internet based reuse please link the reused parts to www.fodina.de and
% mention the original author Karsten Reincke in a suitable manner. In a
% paper-like reuse please insert a short hint to www.fodina.de and to the
% original author, Karsten Reincke, into your preface. For normal quotations
% please use the scientific standard to cite.
%

%% use all entries of the bibliography
%\nocite{*}

\section{Fulfilled Wishes Evoke Thanks: Wer ist schon allein auf der Welt?}

Fassen wir zusammen: Verglichen mit dem numerischen Verweisen oder kryptischen
Schlüsselreferenzen innerhalb des Lesetextes, ja selbst verglichen mit stark
verkürzendem Autor-Jahr-Schema bietet uns \emph{Jurabib} - entsprechend
konfiguriert - eine lese- und lernbegünstigende Alternative: Der
Anmerkungsapparat bedient seine immanente Aufgabe, Zitate zu belegen. Und
zugleich kann er zum forschungshistorischen Dienst werden. Er breitet vor dem
Leser vertrackte Aspekte der Wissenschaftsgeschichte aus und reicht damit die
schmerzliche Detailarbeit des Autors uneigenützig an die Leser weiter. Wissen
ist hier nicht mehr Macht, Gelehrsamkeit nicht mehr Klientel stabilisierendes
Herrschaftswissen, sondern schlichter \emph{Dienst am Kunden}.

Bliebe nur noch zu gestehen, dass mein Anteil an dieser Lösung bestenfalls im
genauen Lesen und Anwenden der Vorarbeit anderer besteht: Den zentralen, geradezu
erlösenden Hinweis auf das Jurabib-Paket habe ich dem LaTeX-Begleiter
entnommen\footcite[vgl.][741ff]{MitGoo2005a}, die Einzelheiten zu seiner
Nutzung natürlich auch dem entsprechenden Handbuch\footcite[vgl.][]{Berger2004a}.
Und Basis meiner LaTeX-Kenntnis bildet bis heute die LaTeX-Einführung von Helmut
Kopka\footcite[vgl.][]{Kopka2000a}. 

Und bei allen Feinheiten sollten wir nicht vergessen, dass wir es hier mit
freier Software zu tun haben: LaTeX ist frei, Jurabib ist frei, Koma-Script ist
frei und Texlipse, mein bevorzugtes LaTeX-Plugin für Eclipse ist frei. Es gehört
sich mithin so, wenn auch ich meine Arbeit freigebe: 

\begin{itemize}
  \item Aus dem anfänglichen Dokument über die Erstellung
  geisteswissenschaftlicher Texte mit \textit{jurabib} ist mittlerweile ein
  ganzes Framework namens \textit{mycsrf} entstanden, das - zu diesem Stil
  passend -
  \begin{itemize}
    \item die Suche und Evaluation von Sekundärliteratur unterstützt
    \item die Pflege der bibliographischen Daten vereinfacht
    \item die Erstellung dazu passender 'Abstracts' und 'Extracts' ermöglicht
    \item und das schließlich auch das Schreiben der eigentlichen Arbeit
    erleichtert
  \end{itemize}
  Dieses \textit{mycsrf-framework} ist unter der \textit{Creative Commons
  3.0 Germany License} veröffentlicht\footnote{Weitere Infos und Download unter
  \texttt{http://github.com/kreincke/mycsrf/}.}.
  \item Das Dokument jedoch, was sie gerade lesen, ist - davon unabhängig -
  unter der \textit{Creative Commons Attribution-ShareAlike 3.0 Germany License}
  veröffentlicht. Auch dazu können sie sich das Quellcode-Paket
  herunterladen\footnote{s. dazu
  \texttt{http://github.com/kreincke/mycsrf}.}.
\end{itemize}

Damit ist die Sache ganz einfach: auf dem Framework können Sie ihre eigenen
Arbeiten frei aufsetzen und vertreiben. Wenn sie jedoch an diesem Text über
\textit{einen besonderen Dienst am Leser} weiterarbeiten, geben Sie ihn bitte
unter derselben Lizenz weiter\footnote{Die Details zur Lizenzerfüllung entnehmen
Sie bitte in beiden Fällen der lizenzierenden Anmerkung und der zugehörigen
Creative Commons Lizenz im Netz}.



\small

% insert the nomenclature here

% mycsrf Deutsch Nomenclation Tokens Include Module 
%
% (c) Karsten Reincke, Frankfurt a.M. 2012, ff.
%
% This file is licensed under the Creative Commons Attribution 3.0 Germany
% License (http://creativecommons.org/licenses/by/3.0/de/): 
% For details see teh file LICENSE in the top directory

% specific abbreviations
\abbr[utb]{UTB}{Uni-Taschenbuch}
\abbr[stw]{stw}{suhrkamp taschenbuch wissenschaft}% mycsrf  Deutsch Nomenclation Tokens Include Module 

% general abbreviations
\abbr[vgl]{vgl.}{vergleiche}
\abbr[aaO]{a.a.O.}{am angegebenen Ort}
\abbr[ds]{ds.}{kollektiv für ders., dies., \ldots}
\abbr[ebda]{ebda.}{ebenda}
% \abbr[id]{id.}{idem = latin for 'the same', be it a man, woman or a group\ldots}
% \abbr[ibid]{ibid.}{ibidem = latin for 'at the same place'}
\abbr[ifross]{ifross}{Institut für Rechtsfragen der Freien und Open Source
Software}
% \abbr[lc]{l.c.}{loco citato = latin for 'in the place cited'}
\abbr[wp]{wp.}{webpage = Webdokument ohne innere Seitennummerierung}
% mycsrf English Nomenclation Tokens Include Module 
%
% (c) Karsten Reincke, Frankfurt a.M. 2012, ff.
%
% This file is licensed under the Creative Commons Attribution 3.0 Germany
% License (http://creativecommons.org/licenses/by/3.0/de/): 
% For details see teh file LICENSE in the top directory
%

\abbr[afda]{AfdA}{Anzeiger für deutsches Altertum}
%\abbr[zfda]{ZfdA}{Zeitschrift für deutsches Altertum und deutsche Literatur [ISSN: 00442518]}
%\abbr[zfaw]{}{Zeitschrift für Allgemeine Wissenschaftstheorie / Journal for General Philosophy of Science [ISSN: 0044-2216]}

\printnomenclature

% insert the bibliographical data here
\bibliography{bib/literature}

\end{document}
