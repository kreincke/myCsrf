%fodina humanitied 'for being included' snippet template
%
% (c) Karsten Reincke, Frankfurt a.M. 2010, 2011, ff.
%
% This LaTeX-File is licensed under the Creative Commons Attribution-ShareAlike
% 3.0 Germany License (http://creativecommons.org/licenses/by-sa/3.0/de/): Feel
% free 'to share (to copy, distribute and transmit)' or 'to remix (to adapt)'
% it, if you '... distribute the resulting work under the same or similar
% license to this one' and if you respect how 'you must attribute the work in
% the manner specified by the author ...':
%
% In an internet based reuse please link the reused parts to www.fodina.de and
% mention the original author Karsten Reincke in a suitable manner. In a
% paper-like reuse please insert a short hint to www.fodina.de and to the
% original author, Karsten Reincke, into your preface. For normal quotations
% please use the scientific standard to cite.
%
%% use all entries of the bibliography
%\nocite{*}
\section{Bib\TeX\ und Jabref - aufeinander abgestimmt abgestimmt}

Um in und mit \LaTeX\ Literaturverweise möglichst geschickt zu managen, muss man
sich die Konzepte klargemacht haben, die dafür benötigt werden:

\begin{enumerate}
  \item \LaTeX\ selbst stellt Fußnoten oder -- mit geeigneten Zusatzpaketen --
  Endnoten zur Verfügung. Das ist auch schon alles. Und es ist eigentlich auch
  alles, was man braucht: für jeden Quellennachweis könnte man jetzt händisch in
  eine Fußnote die entsprechenden bibliographischen Angaben
  eintragen\footnote{wie u.a. vorgeführt in Fakeautor, Class: Über geünsteltes
  Schreiben, Hamburg 2019, S.66}, die lange Form, wenn die Quelle zum ersten Mal
  auftaucht, die kürzere, wenn sie schon erwähnt worden ist. Und das
  Literaturverzeichnis würde man dann am Ende auch noch händisch tippen. Nur wäre es nicht
  besonders geschickt, wenn man so verführe. Denn Computer können und sollen uns
  wiederkehrende Arbeiten erleichtern -- hier: die fehlerträchtige Eingabe immer
  derselben Daten.
  \item Um das zu automatisieren, existiert mit Bib\TeX\ eine Abmachung, wie man
  die bibliographischen Angaben in einer Textdatei strukturiert abspeichert.
  Allerdings verlangen verschiedene Literaturtypen unterschiedliche Angaben: zu
  einem Zeitschriftenartikel muss die Zeitschrift angegeben werden, zu einem
  Internetartikel nicht unbedingt. Und Bücher brauchen ganz andere
  Informationen. Deshalb wird jeder Datensatz in der Bib\TeX-Datei auch
  typisiert.
  \item Allerdings ist man sich in der Forschungsgemeinschaft durchaus nicht
  einig, bei welchem Typ welche bibliographischen Angaben in welchem Kontext
  erscheinen müssen. Das ist von Profession zu Profession und gelegentlich auch
  von Professor zu Professor und Forscher zu Forscher verschieden. Es ist -- bis
  zu einer gewissen Grenze\footnote{An dieser Stelle haben wir unsere eigene
  dezidierte Meinung: Es sollte alles angegeben werden, was die Überprüfung
  erleichtert. Und es darf nichts fehlen, was die Überprüfung systematisch oder
  vom angemessenen Aufwand her verhindert: wer als Nachweis für ein Zitat nur
  auf ein 1000 seitiges Werk verweist, ohne die Seite zu nennen, erwartet von
  seinem Leser implizit, dass er das ganze Werk durchgeht, um zu verifizieren,
  dass es das Zitat enthält. Solch eine Erwartung ist unzumutbar. Sie behindert
  und verhindert letztlich die Überprüfbarkeit. Sie unterläuft das
  \emph{Abgrenzungskriterium} und evoziert Unwissenschaftlichkeit!} -- eine
  Stilfrage. Deshalb erlaubt \LaTeX\ das Einbinden verschiedener Zitierstile.
  Diese legen fest, was für welche Literaturtypen an welcher Stelle ausgegeben
  wird.
\end{enumerate}

Dies alles muss aufeinander abgestimmt sein. Der Zitierstil muss die Typen
bedienen, die in der Bib\TeX-Datei verwendet werden. Und die Bib\TeX-Datensätze
müssen die Informationen bereitstellen, die der Zitierstil für vernüftige
Angaben benötigt. Und das Verbindungsglied ist der Nutzer. Er muss wissen,
welche Typen sein Zitierstil verwenden kann und welche Angaben dabei jeweils
mindestens notwendig sind. Dabei unterstützen ihn Bib\TeX-Frontends. Besonders
ausgereift ist \emph{Jabref}\footcite[vgl.][\nopage wp]{Jabref2019a}. Wenn es
entsprechend konfiguriert ist, signalisiert es fehlende Angaben.

In unserem Fall haben wir den Zitierstil \emph{Jurabib} 'umkonfiguriert'.  Wer
den so entstandenen aplphilologisch-geisteswissenschaftlichen Stil verwenden
möchte, muss also offensichtlich noch erfahren, welche realen Literaturtypen er
auf welche Bib\TeX-Typen abbilden kann, damit er die versprochenen Ergebnisse
bekommt. Es sind viele Typen im Umlauf. Selbst die reduziertere
Wikipedia-Klassifikation\footnote{$\rightarrow$
\lnkb{https://de.wikipedia.org/wiki/BibTeX}{2019-01-24}} kann noch verwirren.
Die Lösung bringt ein einfacher Gedanke: Es ist nicht wichtig, der
Bib\TeX-Typologie gerecht zu werden. Es ist nur wichtig, konsistent den
Literaturtyp aus der realen Welt auf einen Bib\TeX-Typ abzubilden, aus dem der
Zitierstil die richtigen Angaben erzeugt.

Mit diesem Ansatz kann man die Möglichkeiten auf ein sehr überschaubares Maß
reduzieren. Und man kann Jabref entsprechend konfigurieren:

\subsection{Definition der Literaturtypen}

Allen Typen gemeinsam sind die optionalen Felder \textsf{Annote,
Note, Language, Url, Urldate}. Ansonsten unterscheiden sie sich durch folgende
Kombinationen\footnote{\emph{InProceedings} versus \emph{Inbook} dürften ebenso
etwas schwieriger zu unterscheiden sein wie \emph{Misc} versus \emph{Www}.
Deshalb hier die Auflösung:
\begin{itemize}
  \item \emph{InProceedings} verweist nur auf den Sammlungsband per
  \emph{Crossref}. \emph{Inbook} nimmt die Daten des Sammlungsbandes dagegen in
  sich selbst auf. Einen Sammlungsartikelso zu erfassen, macht Sinn, wenn aus
  der Sammlung nur diesen einen Artikel benötigt.
  \item Viele Seiten im Internet haben sehr wohl einen Autor und ggfls. sogar
  ein Erscheinungsjahr, obwohl sie eben nicht als Teil einer Sammlung, eines
  Buches oder einer Zeitschrift veröffentlicht worden sind. Solche Werke erfasst
  \emph{mycsrf} im Typ \texttt{Misc}. Der Typ \texttt{Www} meint dann einen bloßen
  Link auf eine Seite ohne Autor oder Jahr.
\end{itemize}}:

\subsubsection{Buch (mit Autor(en)) = Bib\TeX-Typ \texttt{Book} } 

  \begin{description}
  \item[obligatorisch] :- \textsf{BibtexKey, Author, Title, Address, Year,  Shorttitle}
  \item[fakultativ] :- \textsf{TitleAddon, Editor, Edition, Publisher, ISBN, Series, Volume,
  Volumes (+ generelle Felder)}
  \end{description}

\subsubsection{Zeitschriftenartikel = Bib\TeX-Typ \texttt{Article} }

  \begin{description}
  \item[obligatorisch] :- \textsf{BibtexKey, Author, Title, Journal, Year, Pages, Shorttitle}
  \item[fakultativ] :- \textsf{TitleAddon, Volume, Number, ISSN (+ generelle Felder)}
  \end{description}

\subsubsection{Sammlung  Bib\TeX-Typ \texttt{Proceedings}}

  \begin{description}
  \item[obligatorisch] :- \textsf{BibtexKey, Editor, Title, Address, Year, Shorttitle}
  \item[fakultativ] :- \textsf{TitleAddon, Publisher, ISBN, Volume, Volumes, Series,
  Number (+ generelle Felder)}
  \end{description}

\subsubsection{Sammlungsartikel mit Referenz = Bib\TeX-Typ \texttt{Inproceedings} }

  \begin{description}
  \item[obligatorisch] :- \textsf{BibtexKey, Author, Title, Crossref, Pages, Shorttitle}
  \item[fakultativ] :- \textsf{TitleAddon, (generelle Felder)}
  \end{description}
  
\subsubsection{Sammlungsartikel mit Buchangaben = Bib\TeX-Typ \texttt{Incollection | Inbook}   }

  \begin{description}
  \item[obligatorisch] :- \textsf{BibtexKey, Author, Title, Pages, Shorttitle, booktitle, address, year}
  \item[fakultativ] :- \textsf{Titleaddon, Bookauthor, Booktitleaddon, Editor, Edition, Volume, Series, Number,  Isbn  (+ generelle Felder)}
  \end{description}  
  

\subsubsection{Internetwerk (mit Autor(en)) = Bib\TeX-Typ \texttt{Misc} }

  \begin{description}
  \item[obligatorisch] :- \textsf{BibtexKey, Author, Title, Year, Shorttitle}
  \item[fakultativ] :- \textsf{Titleaddon, (generelle Felder)}
  \end{description}

\subsubsection{Internetseite Bib\TeX-Typ \texttt{Www} }

  \begin{description}
  \item[obligatorisch] :- \textsf{BibtexKey, Author=\texttt{anonymous}, Title, Shorttitle}
  \item[fakultativ] :-  \textsf{Titleaddon, Year (+ generelle Felder)}
  \end{description}



\subsection{Definition der Datensatzfelder}

Die Semantik der einzelnen Felder ergibt sich wie folgt:

\begin{description}
  \item[Address] :- Erscheinungsort\footnote{Wird mittlerweile als 'deprecated'
  gewertet, informiert bei älteren Werken aber über über Aspekte der
  Forschungshistorie.}
  \item[Annote] :- Kommentar für ein kommentiertes Literaturverzeichnis.
  \item[Author] Name(n) de(s|r) Autor(s|en) -- falls unbekannt, anonymous  
  \item[Bibtexkey] eindeutiger Schlüssel, am besten 
  \textit{Autor}$\cup$\textit{Jahr}$\cup$\texttt{\{a, b, c, \ldots\}}. Bei
  mehreren Autoren von den ersten drei jeweils die ersten drei Buchstaben - capitalized.
  \item[Crossref] :- Bei InProccedings oder InBook Bibtexkey des Proceedings / Books
  \item[Edition] : reine Auflagenzahl
  \item[Editor] :- Herausgeber  
  \item[ISBN] :- ISBN-Nummer
  \item[ISSN] :- ISSN-Nummer
  \item[Journal] :- Zeitschriftentitel
  \item[Language] :- \{english,
  german\footnote{Ja! -- muss kleingeschrieben werden. Bestimmt u. gewissen
  Umständen die Zeichensetzung}, \ldots \} 
  \item[Note] :- \{ \texttt{Print} , ( \texttt{\{FreeWeb, BibWeb\}} $\times$ \}
  \texttt{\{HTML, PDF, \ldots\}} )) 
  \item[Number] :- Nummer / Jhrg einer Zeitschrift
  \item[Pages] :- \emph{Anfangsseite} \texttt{-} \emph{Endseite}
  \item[Publisher] :- Verlag
  \item[Series] :- Reihename
   \item[Shorttitle] Kurztitel der Form
  \emph{Titel}\texttt{,}\emph{Jahr}\footnote{Dies ist bei \emph{mycsrf} ein
  zentraler Datensatzteil. Bei anderen Zitaten aus dem Werk wird nur noch der
  Kurztitel ausgegeben. Leider muss man den Kurztitel immer selbst bilden und 
  manuell eingegeben. Eine automatisierte Ableitung hätte zu tiefe Eingriffe in
  Jurabib erfordert.}
  \item[Title] Werktitel und Untertitel, letzterer mit Punkt abgetrennt
  \item[Titleaddon] : Zusatzangaben wie \emph{aus dem Amerikanischen übers.v.}
  oder \emph{aus dem Amerikanischen übers.v.} oder \emph{3.verb. u. erweit. Aufl.}
  \item[Url] :- weist bei eResourcen  auf
  die Quelle. Sollte / muss bei \texttt{\{FreeWeb, BibWeb\}} gesetzt sein.
  \item[Urldate] :- das Bezugsdatum
  \item[Volume] :- Band in einer Reihe
  \item[Volumes] :- Anzahl der Bände
  \item[Year] Erscheinungsjahr\footnote{Jabref hält dies für 'deprecated'. Wir sehen das
  anders. ISBN und Erscheinungsort und -jahr erleichtern das Auffinden.
  Erscheinungsort- und vor allem: -jahr sind forschungsgeschichtlich wichtige Angaben.}
 \end{description}


\subsection{Jabref Konfiguration}

Mit dieser Typologie sollten man alle realen Werktypen abbilden können. Fehlt
ein Feld, darf man die Informationen im Feld 'Titleaddon' ablegen. \emph{mycsrf}
bietet in der Datei  \texttt{cfg/jabref.prefs.xml} eine Konfiguration, die diese
Typologie umsetzt. Sie müssen sie nur in Jabref laden\footnote{$rightarrow$
Preferences/Import}. Wird sie Ihren Anforderungen nicht gerecht, konfigurieren
Sie \emph{Jabref} einfach um. Aber probieren Sie aus, ob auch wirklich alle
Informationen von unser \emph{jurabib}-Version entsprechend Ihrer Vorstellungen
ausgegeben werden. Ist dem nicht so, müssen Sie selbst an die Konfiguration
'ran'.

Und als Referenz werde ich jetzt noch einmal für jeden Typ ein neues, möglichst
umfangreiches Beispiel geben:

\begin{description}
  \item[Buch (mit Autor(en))] : \cite[][]{GarSpr2018a}
  \item[Zeitschriftenartikel] : \cite[][59]{Lewin1992a} (Hinweis: Seitenbereich erscheint nur im Literaturverzeichnis)
  \item[Sammlung] : \cite[][]{Cysarz1978a}
  \item[Sammlungsartikel mit Referenz] :  \cite[][]{Dach1978a}
  \item[Sammlungsartikel mit Buchangaben] : \cite[][]{Arndt1811a} (Hinweis: Seitenbereich erscheint nur im Literaturverzeichnis)
\end{description}
%\bibliography{../bib/literature}
